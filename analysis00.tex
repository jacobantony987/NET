%Elementary set theory, finite, countable and uncountable sets, Real number system as a complete ordered field, Archimedean property, supremum, infimum.
%Sequences and series, convergence, limsup, liminf.
%Bolzano Weierstrass theorem, Heine Borel theorem.
%Continuity, uniform continuity, differentiability, mean value theorem.
%Sequences and series of functions, uniform convergence.
%Riemann sums and Riemann integral, Improper Integrals.
%Monotonic functions, types of discontinuity, functions of bounded variation, Lebesgue measure, Lebesgue integral.
%Functions of several variables, directional derivative, partial derivative, derivative as a linear transformation, inverse and implicit function theorems.
%Metric spaces, compactness, connectedness. Normed linear Spaces. Spaces of continuous functions as examples

\chapter{Assorted Analysis Results}
\section{Sets}
	\begin{definition}
		Ordered pair $(x,y) = \{ x,\{x,y\} \}$\\
		n-tuple, $(x_1, x_2, \cdots, x_n) = \{ \{x_1\}, \{x_1, x_2\}, \cdots,\{x_1, x_2, \cdots, x_n\}\}$
	\end{definition}
	\begin{remark}A few results\\
		$A \subset B \iff \forall x \in A,\ x \in B$\\
		$A = B \iff A \subset B \ \wedge \ B \subset A$\\
		$A \subset B \implies A \cup B = B,\ A \cap B = A,\ A-B = \phi$\\
		$A = B \iff (A-B=\phi) \ \wedge (B-A=\phi) \qquad \because A \subset B \implies A-B=\phi$\\
		$A-B = A \cap \neg B$\\
		$A-(A-B) = A \cap B$\\
		$\forall X,\ A \subset B \implies X-B \subset X-A$\\
		$\forall X,\ A \subset B \implies A-X \subset B-X$\\
		$(A \cup B) - C = (A-C) \cup (B-C)$\\
		$(A \cap B) - C = (A-C) \cap (B-C)$
	\end{remark}
\section{Functions}
	\begin{definition}
		identity function, $id_A:A \to A,\ \forall x \in A,\ id_A(x) = x$
	\end{definition}
	\begin{remark}
		$f:\mathbb{N}\times\mathbb{N}\to\mathbb{N},\ f(m,n)=2^m3^n$ is injective.\cite{bear1}
	\end{remark}
	\begin{definition}
		The graph of a function $f:A \to B$ is the subset $\{(x,f(x)) : x \in A \}$ of the cartesian product $A \times B$.
	\end{definition}
	\begin{definition}
		A function $f:A \to B$ on ordered sets is monotonic/isotonic/order-preserving if partial order is preserved by images under f.\\
		A function $f:A \to B$ on ordered sets is strictly monotonic/isotonic/order-preserving if strict order is preserved by images under f.
	\end{definition}
	\begin{definition}Order Equivalence\cite{josh1}\\
		A bijection $f:A \to B$ and its inverse $f^{-1}:B \to A$ are both order preserving, then f is an order isomorphism/equivalence.\\
		$\exists$ order-isomorphism $f:A \to B$, then A,B are order-equivalent.\\
		Order type of a poset is the class of all posets order-equivalent to it.
	\end{definition}
\section{POSET}
	\begin{remark}Let $f:A \to A$,\\
		The graph of f is reflexive iff $f = id_A$\\
		The graph of f is symmetric iff \\
		The graph of f is antisymmetric iff \\
		The graph of f is transitive iff \\
		The graph of f is an order on A iff
	\end{remark}
	\begin{remark}
		Every connex relation is reflexive.\\
		symmetric $\wedge$ transitive $\not\!\!\!\implies$ reflexive\\
		every total order is a partial order.\\
		partial order - diagonal = strict order\\
		A subset T of poset S,R is a chain/nest/tower if $R$ is a total order in $T$.
	\end{remark}
	\begin{definition}\cite{josh1}
		$R[x]$ is the equivalence class under R containing x.\\
		The quotient set $S/R$ is the set of all equivalent classes of S under R.\\
		Projection/Quotient function, $p:S \to S/R$ such that $p(x) = R[x]$.
	\end{definition}
\section{Infimum and Supremum}
\section{Cardinality}
	\begin{definition}
		Cardinality, $card(A)$ is the number* of elements in $A$ ?
	\end{definition}
	\begin{remark}
		$card(X)=n \iff card(P(X))=2^n$\\
		Cantor set, $C = \{ e_n : e_n \in \{0,1\} \},\ card(C)=c$
	\end{remark}
	\begin{remark}True or False ?
		\begin{enumerate}
		 \item[F] $\exists x \in R,\ \forall y \in R,\ x+y > 0$
		 \item[T] $\forall x \in R,\ \exists y \in R,\ x+y > 0$
		 \item[F] $\exists y \in R,\ \forall x \in R,\ y^2 > x$
		\end{enumerate}
	\end{remark}
	\begin{definition}
		$x \in \mathbb{R}$ is algebraic if $\exists n \in \mathbb{Z}^+, \forall i \leq n,\ a_i \in \mathbb{Z}\ \wedge \ \exists i \le n$, such that $a_i \neq 0\ \wedge \ a_0 + a_1 x + a_2 x^2 + \cdots + a_n x^n = 0$.
		$x \in \mathbb{R}$ is transcendental if it is not algebraic.
	\end{definition}
	\begin{remark}
		The set of all algebraic numbers is countable. Thus the set of all transcendental numbers is uncountable since $\mathbb{R}$ is uncountable.
	\end{remark}
	\begin{definition}
		$x \in X$ is a accumulation point of $A \subset X$, if $\forall U \in \mathcal{N}_x$, $A\cap U$ is infinite.\\
		$x \in X$ is a condensation point\cite{khar2} of $A \subset X$, if $\forall U \in \mathcal{N}_x$, $A\cap U$ is uncountable.
	\end{definition}
\section{Field}
	\begin{remark}
		$\cup,\cap,-,\Delta$ are binary operations on the family of all subsets of a given set.
	\end{remark}
	\begin{remark}For any field $F,\ \forall x,y \in F$\\
	$0 < 1$\\
		$x < y \iff x-y < 0$\\
		$0 < x \iff -x < 0$\\
		$0 < x \iff 0 < x^{-1}$\\
		$1 < x \iff x^{-1} < 1,\ x^{-1} \neq 0$
	\end{remark}
\section{Real Field}
	\begin{remark}
		is there any finite, ordered field ?\\
		$|x+y| \le |x|+|y| \qquad \& \qquad |x-y| \le |x-z|+|z-y|$
	\end{remark}
	\begin{theorem}
		$\not\!\exists$ smallest positive real number ?.
	\end{theorem}
	\begin{theorem}
		$\forall x \in \mathbb{R},\ \exists n \in \mathbb{Z}$ such that $n < x < n+1$
	\end{theorem}
	\begin{remark}
		$r \in \mathbb{R}, r \not\in \mathbb{Q}, \inf\{ m+nr : m,n \in \mathbb{Z} \} = 0$
	\end{remark}
	\begin{remark}
		$\forall x \in \mathbb{R},\ (x=x^{-1}) \iff (x = 1) \vee (x = -1)$
	\end{remark}
\section{Sequences}
	\begin{theorem}
		$\{x_n\} \to x \implies \{|x_n|\} \to |x|$
	\end{theorem}
	\begin{remark}
		$\{|x_n|\} \to |x| \implies \{x_n\} \text{ has atmost two limit points } x, -x$
	\end{remark}
	\begin{remark}\cite{kham1}
		$\forall n \in \mathbb{N},\ x_n \le y_n$, sequence $\sequence{x_n}$ is increasing, and sequence $\sequence{y_n}$ is decreasing, then sequences $\sequence{x_n},\sequence{y_n}$ are convergent and $\lim_{n \to \infty} x_n \le \lim_{n \to \infty} y_n$\\
		Sequence $\sequence{y_n-x_n} \to 0 \iff \lim_{n \to \infty} x_n = \lim_{n \to \infty} y_n$
	\end{remark}
	\begin{remark}
		Sequence $\sequence{x_n}$, where $x_n = \sum_{k=1}^n \frac{1}{n}$ is divergent.
	\end{remark}
	\begin{remark}\cite{kham1}
		$$x_n = \int_1^n \frac{\cos t}{t^2}dt, \text{ then } \sequence{x_n} \text{ is cauchy ?}$$
	\end{remark}
	\cite{kham1} 3.10 onwards pending
\section{Number Theory}
	\begin{theorem}[binomial theorem]
		$$(a+b)^n = \sum_{k=0}^n \binom{k}{n} a^k b^{n-k},\quad \binom{k}{n} = \frac{n!}{k!(n-k)!}$$
	\end{theorem}

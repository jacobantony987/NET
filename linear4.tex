\chapter{Canonical Forms}
\section{Diagonalisable Matrices}
\begin{note}
	For every square matrix $A$ has a conjugate matrix of the Jordan normal form which unique upto block permutations.
\end{note}

\begin{definition}[diagonalisable]
	A diagonalisable matrix has a diagonal matrix as its Jordan normal form.
\end{definition}

\begin{note}
	Jordan normal form determines the minimal polynomial.
	The set of all polynomials that annihilate $A$ form a principal ideal domain in $\mathbb{C}[x]$ with minimal polynomial as its generator.
\end{note}

\begin{definition}[multiplicity]
	Algebraic multiplicity of an eigenvalue $\alpha$ of $A \in M_n(F)$ is the degree of $(\lambda-\alpha)$ in its characteristic equation.
	Geometric multiplicity of $\alpha$ is the number of blocks in Jordan normal form with diagonal entry $\alpha$.
\end{definition}

\subsection{Properties of Diagonalisable Matrices}
\begin{enumerate}
	\item If $A \in M_n(F)$ with Jordan normal form $J = P^{-1}AP$, then $A^n = PJ^nP^{-1}$.
	\item Eigenvalues, their algebraic and geometric multiplicities, characteristic polynomial, minimal polynomial, trace, determinant, rank and nullity are invaint under conjugation.
	\item A matrix is normal if and only if its diagonalisable by a unitary matrix.
		Thus, real symmetric matrices are diagonalisable over $\mathbb{R}$.
		And hermitian, skew-hermitian matrices are diagonalisable over $\mathbb{C}$.
	\item real skew-symmetric matrices are not diagonalisable over $\mathbb{R}$.
	\item Rotation matrices are non-diagonalisable over $\mathbb{R}$ but diagonalisable over $\mathbb{C}$.
	\item Non-zero nilpotent matrices are non-diagonalisable over any field $F$.
	\item Sum of diagonalisable matrices need not be diagonalisable.
\end{enumerate}


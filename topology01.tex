%basis, dense sets, subspace and product topology, separation axioms, connectedness and compactness

\section{Metric Space}
	\begin{definition}
		A function $d : X \times X \to \mathbb{R}$ is a metric on $X$ if it satisfies
		\begin{enumerate}
			\item $d(x,y) \ge 0,\ d(x,y) = 0 \iff x = y$
			\item $d(x,y) = d(y,x)$
			\item $d(x,y) = d(x,z) + d(z,y)$
		\end{enumerate}
	\end{definition}
	\begin{definition}
		A set $X$ together with a metric on $X$ is a metric space.
	\end{definition}
	\begin{description}
		\item[neighbourhood] A subset $N$ is a neighbourhood of a point $x$ if there exists a positive real number $r$ such that $y \in N$ for every $y$ satisfying $d(x,y) < r$.
		\item[limit point] A point $x$ is a limit point of a subset $A$ if every neighbourhood of $x$ has some another point from the subset $A$.
		\item[isolated point] A point which is not a limit point of the subset $A$.
		\item[closed] A subset $A$ of $X$ is closed if it has every limit point of it. $\overline{A} = A$
		\item[interior point] A point $x$ is an interior point the subset $A$ if there exists some neighbourhood of $x$, which is contained in the subset $A$. $x \in A^0$
		\item[open] A subset $A$ is open if each point of it is an interior point in it.
		\item[complement] The subset of all points which are not in the subset $A$ is the complement of $A$.
		\item[perfect] A subset $A$ is perfect if it is closed and has each point of it as its limit point.
		\item[bounded] A subset $A$ is bounded if there is a point $x$ and a positive real number $r$ such that for each point $y \in A$, $d(x,y) < r$.
		\item[dense] A subset $A$ is dense if every point of $X$ is either in $A$ or is a limit point of $A$.
		\item[boundary] A point $x$ is a boundary point of subset $A$ of $X$ if every neighbourhood of $x$ has some point from $A$ as well as $X-A$. $x \in \partial(A)$
	\end{description}
	\begin{remark}
		\begin{enumerate}
			\item $\overline{A} \cap \overline{B} \ne \overline{A \cap B}$ eg. $\overline{(1,2)} \cap \overline{(2,3)} = \{ 2 \},\ \overline{(1,2)\cap(2,3)} = \{ \}$
			\item There exists countable, dense subsets of $\mathbb{R}$ with empty interior. eg. $\mathbb{Q}$
			\item There exists uncountable, dense subsets of $\mathbb{R}$ with empty interior.
			\item $A \times B$ is open(or closed) iff both $A,B$ are open(or closed).
			\item For Cantor set $\mathcal{C}$, $\mathcal{C}^0 = \phi,\ \overline{\mathcal{C}} = \mathcal{C},\ \partial(\mathcal{C}) = \mathcal{C}$
			\item $\mathcal{C}$ can't be expresses as countable union of closed intervals.
			\item $\mathbb{R}-\mathcal{C}$ can be expresses as countable union of open intervals.
		\end{enumerate}
	\end{remark}
	\begin{remark}
		\begin{enumerate}
			\item For any bounded subset $A$, there exists a point $x$ such that $d(x,y) < r$ for every $y \in A$. Then for any point $z \in A$, $d(z,y) < 2r$ for every $y \in A$.
		\end{enumerate}
	\end{remark}
	\begin{definition}
		\begin{description}
			\item[radius] The radius of a bounded subset $A$ is the smallest real number $r$ such that for a particular $x \in X$, $d(x,y) < r$ for every point $y \in A$.
		\end{description}
	\end{definition}
\section{Topological Space}
	\begin{definition}
		A set $X$ together with a family $\mathcal{T}$ of subsets of $X$is a topological space if it satisfies
		\begin{enumerate}
			\item $\mathcal{T}$ is closed under arbitrary unions
			\item $\mathcal{T}$ is closed under finite intersections
		\end{enumerate}
	\end{definition}
	\begin{description}
		\item[usual] $d(x,y) = |x-y|$
		\item[discrete] $d(x,y) = \delta_{xy}$
		\item[taxicab] $d(x,y) = |x| + |y|$
%		\item[box] $d(x,y) = \max\{|x|,|y|\}$
	\end{description}
	\begin{definition}
		A set is separable if it has a countable dense subset. 
	\end{definition}

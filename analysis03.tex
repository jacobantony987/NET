%Bolzano Weierstrass theorem, Heine Borel theorem
\section{Bolzano-Weierstrass Theorem}
	\begin{theorem}
		Every sequence $\{x_n\}$ in $\mathbb{R}$ has subsequences converging to $\varliminf \{x_n\}$ and $\varlimsup \{x_n\}$.
	\end{theorem}
	\begin{remark}
		A sequence $\{x_n\}$ is convergent iff $\varliminf \{x_n\} = \varlimsup \{x_n\}$
	\end{remark}
	\begin{theorem}
		Every bounded sequence in $\mathbb{R}$ has a convergent subsequence.\footnote{Every bounded sequence in $\mathbb{R}$ has subsequences converging to $\varliminf$ and $\varlimsup$. \cite{aliprant}}
	\end{theorem}
	\begin{theorem}
		Every bounded sequence $\{x_n\}$ in $\mathbb{R}$ converges iff $\varliminf \{ x_n \} = \varlimsup \{ x_n \}$.
	\end{theorem}
	\begin{definition}
		A sequence $<x_k>$ is Cauchy, if $\forall \epsilon > 0,\ \exists N \in \mathbb{N}$ such that $\forall n,m > N,\ |a_n-a_m|<\epsilon$.
	\end{definition}
	\begin{theorem}[Cauchy]
		A sequence in $\mathbb{R}$ is convergent iff it is a Cauchy sequence.
	\end{theorem}
	\begin{remark}
		For a sequence $\{ x_n \}$ in $\mathbb{R}$ following statements are equivalent :
		\begin{enumerate} 
			\item $\{ x_n \}$ converges to $x$
			\item Every subsequence of $\{ x_n \}$ has a subsequence converging to $x$
			\item $\{ n_k \}, \{ m_k \}$ be sequences in $\mathbb{N}$, then $\lim x_{n_k} = \lim x_{m_k} = x$
		\end{enumerate}
	\end{remark}
	\begin{remark}
		$\lim \{ x_n \} = x$ iff $\lim \{ x_{2n} \} = \lim \{ x_{2n+1}\} = x$
	\end{remark}
	\begin{theorem}[Bolzano-Weierstrass]
		Every bounded, infinite subset $A$ of $\mathbb{R}^k$ has a cluster point.
	\end{theorem}
	\begin{remark}
		Cluster points of $A \cup B$ are either cluster points of $A$ or of $B$.
	\end{remark}

\section{Heine-Borel Theorem}
	\begin{theorem}
		A subset of $\mathbb{R}^p$ is compact iff closed and bounded.
	\end{theorem}
	\begin{remark}
		Applications
		\begin{enumerate}
			\item Cantor Intersection Theorem
			\item Lebesgue Covering Theorem
			\item Nearest Point Theorem
			\item Circumscribing Contour Theorem
		\end{enumerate}
	\end{remark}
	\begin{theorem}[Cantor intersection]
		Let $F_1, F_2, \cdots$ be non-empty, closed, bounded subsets of $\mathbb{R}^p$ such that $F_1 \supset F_2 \supset \cdots$. Then there exists a point $y$ such that $y \in F_k,\ \forall k$.
	\end{theorem}
	\begin{theorem}[Lebesgue covering]
		Let $K$ be a compact subset of $\mathbb{R}^p$ and $\mathcal{U}$ be a cover of $K$. There exists a real number $r > 0$ such that $\forall x \in K,\ B(x,r) \subset U$ for some $U \in \mathcal{U}$.
	\end{theorem}
	\begin{theorem}[Nearest Point]
		Let $K$ be a compact subset of $\mathbb{R}^p$ and $x \not\in K$, then there exists $y \in K$ such that $|x-y| \le |x-z|,\ \forall z \in K$.
	\end{theorem}
	\begin{theorem}[Circumscribing contour]
		Let $K$ be a compact subset of $\mathbb{R}^2$ and $G$ be an open set containing $K$. Then there exists a closed curve contained in $G$ made up of arcs of finite number of circles in $G$ such that $K$ is contained in it. 
	\end{theorem}

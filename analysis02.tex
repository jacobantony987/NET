%Sequences and series, convergence, limsup, liminf

\chapter{Sequences and Series}
\section{Sequences}
	\begin{definition}[sequence]
		A sequence $\sequence{x_n}$ in $\mathbb{R}$ is a function $x: \mathbb{N} \to \mathbb{R}$ where the $n^{th}$ term of the sequence, $x_n = x(k)$.
	\end{definition}

	\begin{definition}
		The range of a sequence $\sequence{x_n}$ is the set $\{ x_k : k \in \mathbb{N} \}$.
	\end{definition}

	\begin{remark}
		The range of any sequence is countable.
	\end{remark}

	\begin{definition}[subsequence]
		Given a sequence $\sequence{x_k},\ x:\mathbb{N} \to X,\ x_k = x(k)$ and a monotone function $N:\mathbb{N} \to \mathbb{N},\ n_k = n(k)$, then the sequence $\sequence{y_k} : \mathbb{N} \to X,\ y_k = x_{n_k} = (x \circ n)(k)$ is a subsequence of $\sequence{x_k}$.
	\end{definition}

	\begin{axiom}[Dependent Choice]
		Let $\le$ be a relation on $X$ such that every element $x \in X$ is related to some element of $X$, then there exists a sequence for each element $x \in X$ such that $x_1 = x$ and $x_k \le x_{k+1}$ for every integer $k \in \mathbb{N}$.\cite{kharazishvili}
	\end{axiom}

	\begin{theorem}[Recursive Definition]
		Given a function $f:X \to X$, for every $x \in X$ there exists a unique sequence $\sequence{x_n}$ such that $x_1 = x,\ x_{k+1} = f(x_k),\ \forall k \in \mathbb{N}$
	\end{theorem}

	\begin{corollary}[Generalised Recursive Definition]
		Given a sequence of functions $f_n : X^n \to X$, for every $x \in X$ there exists a unique sequence $\sequence{x_n}$ such that $x_1 = x,\ x_k+1 = f_k(x_1,x_2,\cdots,x_k),\ \forall k \in \mathbb{N}$
	\end{corollary}

\section{Convergence of Sequences}
	\begin{definition}[convergence]
		A sequence $\sequence{x_n}$ converges to $x \in \mathbb{R}$ if for every $\epsilon > 0$, there exists $N \in \mathbb{N}$ such that $|x-x_n| < \epsilon$ for every $k \ge N$. The real number $x$ is the limit of the sequence $\sequence{x_n}$. ie, $\sequence{x_n} \to x$. A sequence $\sequence{x_n}$ is divergent if it is not convergent. ie, $\sequence{x_n} \to \infty$.
	\end{definition}
	
	\begin{remark}Examples,\\
		Convergent sequence, $\sequence{x_n},\ x_n = \frac{1}{n}$, $\sequence{x_n} \to 0$\\
		Divergent sequence, $\sequence{x_n},\ x_n = -1^n$, $\sequence{x_n} \to \infty$
	\end{remark}

	\begin{remark}[$\epsilon$-neighbourhood of $x$]
		A sequence $\sequence{x_n} \to x \iff \forall \epsilon > 0,\\ \exists N \in \mathbb{N} \text{ such that } \forall n \ge N,\ x_n \in (x-\epsilon,x+\epsilon)$.
	\end{remark}

	\begin{definition}[bounded]
		A sequence $\sequence{x_n}$ is bounded if $\exists M >0$ such that $\forall n \in \mathbb{N},\ |x_n| \le M$. ie, $\forall n \in \mathbb{N},\ x_n \in [-M,M]$.
	\end{definition}

	\begin{remark}
		Every convergent sequence is bounded.
	\end{remark}

	\begin{theorem}[algebraic limit]
		Suppose $\sequence{x_n} \to x,\ \sequence{y_n} \to y$, then
		\begin{enumerate}
			\item $\forall n \in \mathbb{N},\ x_n \le y_n \implies x \le y$
			\item $\sequence{\alpha x_n + \beta y_n} \to \alpha x + \beta y$
			\item $\sequence{x_n y_n} \to xy$
			\item $\sequence{\frac{x_n}{y_n}} \to \frac{x}{y}$, provided $y \ne 0$.
		\end{enumerate}
	\end{theorem}

	\begin{remark}Application 1\\
		Let $\sequence{x_n},\ x_n = \frac{n+3}{2n} \implies \forall n \in \mathbb{N},\ x_n = \frac{1}{2} + \frac{3}{2n}$\\
		Thus, $x_n = 0.5y_n + 1.5z_n$, where $\forall n \in \mathbb{N},\ y_n = 1,\ z_n = \frac{1}{n}$\\
		We have, $\sequence{y_n} \to 1$ and $\sequence{z_n} \to 0$. Thus, $\sequence{x_n} \to 0.5$
	\end{remark}

	\begin{remark}Application 2\\
		Let $\sequence{x_n}$ is convergent and given recursive definition, $x_1 = c,\ x_{n+1} = f(x_n)$. Then $\sequence{x_n} \to x$ where $x = f(x)$. In case of multiple solutions, use $x_1$.\\
		For example, $x_1 = 3,\ x_{n+1} = \frac{1}{4-x_n} \implies x = \frac{1}{4-x} \implies x = 2 \pm \sqrt{3},\ x \ne 4$.\\
		However, $\forall n \ge 3,\ x_n < 1$. Thus $x = 2 - \sqrt{3}$.
	\end{remark}

	\begin{theorem}[order limit]
		Suppose $\sequence{x_n} \to x,\ \sequence{y_n} \to y$, then
		\begin{enumerate}
			\item $\forall n \in \mathbb{N},\ x_n \ge 0 \implies x \ge 0$.
			\item $\forall n \in \mathbb{N},\ x_n \ge y_n \implies x \ge y$.
			\item $\exists c \in \mathbb{R},\ \forall n \in \mathbb{N},\ x_n \ge c \implies x \ge c$.
			\item $\exists c \in \mathbb{R},\ \forall n \in \mathbb{N},\ x_n \le c \implies x \le c$.
		\end{enumerate}
	\end{theorem}

	\begin{theorem}[Squeeze]
		Suppose $\sequence{x_n},\ \sequence{y_n},\ \sequence{z_n}$ are sequences. Then \\$\forall n \in \mathbb{N},\ x_n \le z_n \le y_n,\ \sequence{x_n} \to k,\ \sequence{y_n} \to k \implies \sequence{z_n} \to k$.
	\end{theorem}

	\begin{remark}[shuffled sequence]
		Let $\sequence{x_n},\sequence{y_n}$ be sequences and $\sequence{z_n}$ be a shuffled sequence given by $\sequence{ x_1,y_1,x_2,y_2,\cdots }$, then\\ $\sequence{z_n} \to k \iff \sequence{x_n} \to k,\ \sequence{y_n} \to k$.
	\end{remark}

	%This result can be generalised for a shuffled(without any rearrangement) sequence obtained from a finite family of sequences converging to a common point.

	\begin{definition}[absolute convergence]
		A sequence $\sequence{x_n}$ is absolutely convergent if the sequence $\sequence{|x_n|}$ is convergent.
	\end{definition}

	\begin{remark}
		Every convergent sequence is absolutely convergent.\\
		$\sequence{x_n} \to x \implies \sequence{|x_n|} \to |x|$
	\end{remark}

	\begin{remark}
		Suppose $\sequence{x_n}$ is bounded and $\sequence{y_n} \to 0$, then $\sequence{x_n y_n} \to 0$
	\end{remark}
	
	\begin{remark}
		$\sequence{x_n} \to 0$ and $\forall n \in \mathbb{N},\ |y_n-y| \le x_n$, then $\sequence{y_n} \to y$.
	\end{remark}

	\begin{definition}[Cesaro Means]
		The sequence $\sequence{y_n}$ where $y_k = \frac{1}{k} \sum_{n=1}^k x_k$ is the sequence of the averages of a sequence $\sequence{x_n}$.
	\end{definition}

	\begin{remark}Let $\sequence{y_n}$ be the sequence of averages of the sequence $\sequence{x_n}$,
		\begin{enumerate}	
			\item $\sequence{y_n} \to \infty \implies \sequence{x_n} \to \infty$
			\item $\sequence{x_n} \to k \implies \sequence{y_n} \to k$
		%	\item $\sequence{y_n}$ converges, then $\sequence{x_n}$ need not converge. \footnote{ $\sequence{x_n}$ such that $x_n = -1^n$}
		\end{enumerate}
	\end{remark}

	\begin{definition}[cluster point]
		A real number $x$ is a cluster point of the sequence $\sequence{x_n}$ if for every $\epsilon > 0$ and every integer $N \in \mathbb{N}$, there exists an integer $k > N$ such that $|x_n - x| < \epsilon$.
	\end{definition}

	\begin{remark}
		A real number $x$ is a cluster point of the sequence $\sequence{x_n}$ iff there is a subsequence converging to $x$.
	\end{remark}
	

	\begin{definition}[cauchy]
		A sequence is cauchy if for every $\epsilon > 0$, there exists an integer $N \in \mathbb{N}$ such that for every $n,m > N$, $|x_n-x_m| < \epsilon$.
	\end{definition}

	\begin{remark}
		A sequence of real numbers converges iff cauchy.
	\end{remark}

\section{Monotone Convergence Theorem}
	\begin{theorem}
		Every bounded monotone sequence $\sequence{x_n}$ in $\mathbb{R}$ converges.
	\end{theorem}
	
	\begin{remark}
		If monotone decreasing and bounded below, $\sequence{x_n} \to \inf \sequence{x_n}$.\\
		If monotone increasing and bounded above, $\sequence{x_n} \to \sup \sequence{x_n}$.
	\end{remark}

	\begin{remark}
		Every convergent sequence is bounded.
	\end{remark}

	\begin{remark}
		A sequence of real numbers can have atmost one limit.
	\end{remark}

	\begin{definition}.\\
		limit inferior, $\varliminf \sequence{x_n}= \sup \sequence{\inf \sequence{x_m : m \ge n} }$\\
		limit superior, $\varlimsup \sequence{x_n}= \inf \sequence{\sup \sequence{x_m : m \ge n} }$
	\end{definition}

	\begin{remark}
		Let $\sequence{x_n}$ be a bounded sequence, then limit superior and limit inferior are the largest and smallest limit points of it.\cite{aliprantis}\\
		$\varliminf \sequence{x_n} \le \varlimsup \sequence{x_n}$
	\end{remark}

	\begin{remark}
		For $\sequence{x_n},\ \sequence{y_n}$ such that $\forall n \in \mathbb{N},\ x_n \le y_n$,
		\begin{enumerate}
			\item $\varlimsup \sequence{x_n} \le \varlimsup \sequence{y_n}$
			\item $\varliminf \sequence{x_n} \le \varliminf \sequence{y_n}$
		\end{enumerate}
	\end{remark}

\section{Convergence Test by Sandwitch Lemma}
	Convergence of sequences can be tested using the Sandwitch Lemma.	
	\begin{remark}
		\begin{align*}
			\varliminf \sequence{x_n} + \varliminf \sequence{y_n} & \le \varliminf \sequence{x_n + y_n}\\ & \le \varlimsup \sequence{x_n + y_n}\\ & \le \varlimsup \sequence{x_n} + \varlimsup \sequence{y_n}
		\end{align*}
	\end{remark}

	\begin{remark}
		Let $\sequence{x_n}$ such that  $\sqrt{n} = (1+x_n)^n$.\\
		By Bernouli's inequality, $\sqrt{n} = (1+x_n)^n \ge 1+nx \implies 0<x<\frac{1}{\sqrt{n}}$.
		$$\varliminf \sequence{\frac{x_n+1}{x_n}} \le \varliminf \sequence{\sqrt[n]{n}} \le \varlimsup \sequence{\sqrt[n]{n}} \le \varlimsup \sequence{\frac{x_n+1}{x_n}}$$
		And $\lim \frac{x_n+1}{x_n} = 1$. Thus $\lim\sqrt[n]{n} = 1$
	\end{remark}

	\begin{remark}
		Let $\sequence{y_n}$ be the sequence of averages of the sequence $\sequence{x_n}$, then $\varliminf x_n \le \varliminf y_n \le \varlimsup y_n  \le \varlimsup x_n$
	\end{remark}

	\begin{remark}Results by Average Sequence\\
		\begin{enumerate}
			\item $x_{n+1}-x_n \to x \implies \frac{x_n}{n} \to x$
			\item If $\sequence{x_n}$ is bounded and $$2x_n \le x_{n+1}+x_{n-1} \implies \sequence{x_n+1 - x_n}$$ is monotone increasing to 0. pending pp.28 exr 12b\cite{alip2}
			\item $ 0 < x_1 < 1,\ x_{n+1} = 1-\sqrt{1-x_n}$, then $x_n$ monotone decreasing to 0 and $\frac{x_{n+1}}{x_n}$ convergent to $\frac{1}{2}$.
			\item $\sequence{x_n},\ x_n = \left( 1 + \frac{1}{n} \right)^n$ is convergent
			\item $\sequence{x_n},\ |x_n-x_{n-1}| \le \alpha |x_{n+1}-x_n|,\ 0 < \alpha < 1$ is convergent
			\item $x_1 = 1,\ x_{n+1} = \frac{1}{3+x_n}$ is convergent to ?
			\item $x_1 = 1,\ x_{n+1} = 1+ \frac{1}{1+x_n}$ is convergent to $\sqrt{2}$
			\item $x_1 = 1,\ x_{n+1} = \frac{1}{2} \left( x_n + \frac{2}{x_n} \right)$ is convergent to $\sqrt{2}$
		\end{enumerate}
	\end{remark}

\section{Series}
	\begin{definition}[series]
		A series is given by $\series{x_n}{n}$.
	\end{definition}

	\begin{definition}[convergence]
		A series $\series{x_n}{n}$ is convergent if the sequence of partial sums $\sequence{y_k}$ such that $y_k = \sum_{n=1}^k x_n$ is convergent. And the limit of the sequence of partial sums is the sum of the series. ie, $\sequence{y_k} \to y$, then $\series{x_n}{n} = y$.
	\end{definition}

	\begin{remark}
		$\series{\frac{1}{n^2}}{n} = r < 2$
	\end{remark}

	\begin{remark}[harmonic series]
		$\series{\frac{1}{n}}{n} = \infty$
	\end{remark}

	\begin{theorem}[cauchy condensation test]
		Suppose $\sequence{x_n}$ is decreasing and $x_n \ge 0$.\\
		$\series{x_n}{n}$ converges iff $\series{2^{n-1}x_{2^{n-1}}}{n}$ converges.
	\end{theorem}

	\begin{corollary}
		$\series{\frac{1}{n^p}}{n}$ converges iff $p > 1$.
	\end{corollary}
	
%	\begin{remark}
%		Every series is not rearrangement invariant.\\
	%		$S = \series{\frac{-1^{n+1}}{n}}{i} = 1 - \frac{1}{2} + \frac{1}{3} + \cdots = 0.69\cdots$ is not rearrangement invariant.\\
%		$S + \frac{1}{2}S = S^* = 1 + \frac{1}{3} - \frac{1}{2} + \frac{1}{5} -\frac{1}{4} + \frac{1}{7} + \cdots$
%	\end{remark}

	\begin{definition}[rearrangement invariant]
		A series $\series{x_n}{n}$ is rearrangement invariant if for every bijection $\sigma : \mathbb{N} \to \mathbb{N},\ \series{x_{\sigma_k}}{k}$ converges and is invariant.
	\end{definition}

	\begin{theorem}
		A series $\series{x_n}{n}$ is rearrangement invariant if for every integer $n \in \mathbb{N}$, $x_n \ge 0$.
	\end{theorem}

	\begin{remark}
		Series that are rearrangement invariant are unconditionally convergent.
	\end{remark}

\section{Double Series}
	\begin{definition}[double series]
		A double series $$\doubleseries{x_{n,m}}{m}{n} = \lim_{k \to \infty} \sum_{n=1}^k \series{x_{n,m}}{m}$$
	\end{definition}

	\begin{theorem}[rearrangement invariance]
		$\forall n,m \in \mathbb{N},\ a_{n,m} \ge 0$, $$\doubleseries{x_{n,m}}{m}{n} = \doubleseries{x_{n,m}}{n}{m}$$
	\end{theorem}

	\begin{theorem}[rearrangement invariant double series into single series]
		$\forall n,m \in \mathbb{N},\ x_{n,m} \ge 0,\ \forall \text{ bijection }\sigma : \mathbb{N} \to \mathbb{N}\times\mathbb{N}$, $$\series{x_{\sigma_k}}{k} = \doubleseries{x_{n,m}}{m}{n}$$
	\end{theorem}

	\begin{corollary}
		$\forall n,m \in \mathbb{N},\ x_{n,m} \ge 0,\ \forall \text{ bijection }\sigma : \mathbb{N}\times\mathbb{N} \to \mathbb{N}\times\mathbb{N}$, $$\doubleseries{x_{\sigma_{j,k}}}{k}{j} = \doubleseries{x_{n,m}}{m}{n}$$
	\end{corollary}

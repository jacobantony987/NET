\section{Number Theory}
\begin{lemma}[Euclid]
	Let $p$ be a prime. If $p$ divides $ab$, then either $p$ divides $a$ or $p$ divides $b$.
\end{lemma}

\paragraph{Greatest Common Divisor}
\begin{enumerate}
	\item B\'ezout`s Identity : If $\gcd(n,m)= d$, then $\exists s,t \in \mathbb{Z}$ such that $d = sn+tm$.
	\item Euclid`s Division Algorithm : If $b > 0$, then $\forall a \in \mathbb{Z}$, $\exists q \in \mathbb{Z}$ and $\exists r \in \mathbb{Z}$ such that $a = qb+r$ where $0 \le r < b$.
	\item Euclid`s Algorithm : $\gcd(a,b) = \gcd(b,r) = \dots = \gcd(d,1)$ where $a = bq+r$.
	\item The linear equation $ax+by=c$ has integer solutions if $\gcd(a,b)$ divides $c$.\\ If $(x,y)$ is a solution, then $(x-b/d,y-a/d)$ is also a solution.
	\item Chinese Remainder Theorem : Let $x \cong a_j \pmod{n_j}$ be a system of congruences where $\gcd(n_j,n_k) = 1,\ (j \ne k)$. Then there exists a solution.\\ If $x_1,x_2$ is are two solutions, then $x_1 \cong x_2 \pmod{N}$ where $N = \prod n_j$.
		$$x \cong \sum a_j M_j N_j \pmod{N} \text{ where } N_j = \frac{N}{n_j} \text{ and } M_j \cong N_j^{-1} \pmod{n_j}$$
\end{enumerate}

\paragraph{Congruences}
\begin{definition}
	The congruence is a relation on $\mathbb{Z}$ defined by
	$$a \cong b \pmod{n} \iff n | (a-b)$$
\end{definition}
\begin{enumerate}
	\item The relation $\cong$ is an equivalence relation.
	\item $a \cong b \pmod{n} \implies \forall k,\ a^k \cong b^k \pmod{n}$.
	\item If $\gcd(a,n) = 1$, then $a^{-1} \pmod{n}$ exists.
	\item Linear congruence equation $ax \cong b \pmod{n}$ has a solution if $\gcd(a,n)$ divides $b$.
\end{enumerate}

\paragraph{Euler`s phi function}
	The function $\phi : \mathbb{N} \to \mathbb{N}$ is defined as $\phi(n) = $ the cardinality of the set $\{k \in \mathbb{N} : k \le n,\ \gcd(n,k)=1\}$.

	\begin{enumerate}
		\item $\phi$ is multiplicative. That is, $\phi(mn) = \phi(m)\phi(n),\ \gcd(m,n)=1$.
		\item $\phi(p^n) = p^n-p^{n-1}$ where $p$ is a prime.
		\item $\phi(n)$ is even for $n > 2$.
		\item The sum of $\phi(d)$ for all divisors of $n$ is $n$.
		\item The sum of all natural numbers $k \le n$ that are relatively prime to $n$ is $n\phi(n)/2$.
	\end{enumerate}

\begin{theorem}[Fermat]
	$a^p \cong a \pmod{p}$
\end{theorem}
\begin{definition}
	A number $x$ such that $a^x \cong a \pmod{x}$ is a (fermat) \textbf{pseudoprime} for base $a$ where $\gcd(a,x) = 1$.
\end{definition}
Number $341$ is the smallest pseudoprime for base $2$.

\begin{definition}
	A number $x$ is a \textbf{Carmichael} number if $a^x \cong a \pmod{x}$ whenever $\gcd(a,x)=1$.
\end{definition}

\subsection{Arithmetical Functions}
\begin{definition}
	A function $f : \mathbb{N} \to \mathbb{C}$ is an \textbf{arithmetical} (number theoretic) function.
\end{definition}

\begin{definition}
	An arithmetical function $f$ is multiplicative iff $f(mn) = f(m)f(n)$ whenever $\gcd(m,n) = 1$.
	And completely multiplicative iff $f(mn) = f(m)f(n)$ always.
\end{definition}

\begin{definition}
	The \textbf{Dirichlet convolution}
	$$ f \ast g = \sum_{d|n} f(d)g\left(\frac{n}{d}\right)$$
\end{definition}
	Clearly, Dirichlet convolution is commutative and associative.\\
	And Dirichlet convolution of multiplicative functions in multiplicative. However, Dirichlet convolution of completely multiplicative functions is not completely multiplicative.
\begin{definition}
	Every artithmetical function $f$ with $f(1) \ne 0$ has a unique \textbf{Dirichlet inverse} $f^{-1}$.
	$$ f^{-1}(n) = \begin{cases} \frac{1}{f(1)} & n = 1 \\ \frac{-1}{f(1)} \displaystyle \sum_{\substack{d|n \\ d < n}} f(n/d)f^{-1}(d) & n > 1 \end{cases} $$
\end{definition}

	Clearly, $(f \ast g)^{-1} = g^{-1} \ast f^{-1}$ provided $f^{-1}$ and $g^{-1}$ exists.

\begin{theorem}
	Let $f$ be multiplicative.
	Then $f$ is completely multiplicative iff $f^{-1} = \mu f$.
\end{theorem}
\paragraph{Arithmetical Functions and their Dirichlet products}
\begin{enumerate}
	\item \textbf{Identity function}, $I(n) = \left[\frac{1}{n}\right]$ vanishes everywhere except at $n = 1$, $I(1) = 1$.
	Clearly, $I$ is completely multiplicative.
	\item \textbf{M\"obius function}, $\mu(n)$ gives the parity of the number of prime factors of a square free number and vanishes for numbers which are contains a square.\\
	For example, $\mu(1) = 1,\ \mu(30) = -1,\ \mu(12) = 0$.
	Clearly, $\mu$ is multiplicative.
\item \textbf{Riemann Zeta function}, $\zeta(n) = 1$ is completely multiplicative.\\
	Thus $\zeta^{-1} = \mu \zeta = \mu$.
	%M\"obius function is the inverse of Riemann zeta function. 
	%Apostol $\zeta(n) = u(n)$.
	\item \textbf{Power function}, $N^\alpha(n) = n^\alpha$ is completely multiplicative.\\
	Thus, $(N^\alpha)^{-1} = \mu N^\alpha$. And $N^0 = \zeta$.
	\item \textbf{Characteristic function}, $\chi_S$ is the membership indicator function.
	$$\chi_S(n) = \begin{cases} 1 & n \in S \\ 0 & n \notin S \end{cases}$$
	$\chi_S$ is not multiplicative.
	\item \textbf{Euler totient function}, $\phi(n)$ gives the number of positive integers less than $n$ which are relatively prime to $n$. And \textcolor{red}{$\phi = \mu \ast N$}. Thus, $\phi^{-1} = \zeta \ast \mu N$.
	\item \textbf{Liouville function} $\lambda(n)$ gives the parity of sum of prime powers of $n$.\\
	For example, $\lambda(1) = 0,\ \lambda(30)=-1,\ \lambda(12)=-1$.\\
	Clearly, $\lambda$ is completely multiplicative and $\lambda^{-1} = \mu \lambda$.\\
	And $\lambda = \mu \ast \chi_{Sq}$ where $Sq$ is the set of all squares.
	\item \textbf{Divisor function} $\sigma_\alpha(n)$ is the sum of $\alpha$th powers of divisors of $n$.\\ Clearly, $\sigma_\alpha = \zeta \ast N^\alpha$. And $\sigma_\alpha^{-1} = \mu \ast \mu N^\alpha$.
	\item $\tau(n)$ gives the number of divisors of $n$.And $d(n)$ gives the sum of divisors of $n$.\\
		Clearly, $\tau = \sigma_0 = \zeta \ast \zeta$. And $d = \sigma = \sigma_1 = \zeta \ast N$.\\
	%\tau(n) = d(n)$
	We have, $\sigma \ast \phi = \zeta \ast N \ast \mu \ast N =  N \ast N = N \tau$ since,
		$$ N \ast N(n)= \sum_{d|n} N(d)N(n/d) = \sum_{d|n} n = N(n) \tau(n)$$
	and $\tau \ast \phi = \zeta \ast \zeta \ast \mu \ast N = \zeta \ast N = \sigma$
	\item $\omega(n)$ gives the number of distinct prime factors of $n$.\\
	Clearly $\omega = \zeta \ast \chi_\mathbb{P}$ where $\mathbb{P}$ is the set of all primes.
	\item $\Omega(n)$ gives the number of prime factors of $n$ counted with multiplicity.
	Clearly, $\Omega = \zeta \ast \chi_\mathcal{P}$ where $\mathcal{P}$ is the set of all prime powers 
	\item $p$-adic valuation $\nu_p(n)$ is the exponent of highest power of prime $p$ that divides $n$.
	$$ \omega(2^n3^m) = 2,\ \Omega(2^n 3^m)=n+m,\ \nu_2(2^n3^m)=n$$
	$$\nu_p(n!) = \left[\frac{n}{p}\right] + \left[\frac{n}{p^2}\right] + \dots$$
\end{enumerate}
\paragraph{Strange Functions}
\begin{enumerate}
	\item $\sin : \mathbb{N} \to [-1,1]$ is an injection since $\sin (x) = \sin (y) \implies 2\pi | (x-y)$.
\end{enumerate}

\section{Functions}
\begin{enumerate}
	\item If $f(x_n) \to L$ as $x_n \to a$, then $\displaystyle \lim_{x \to a} f(x) = L$.
	\item Criteria for Continuity
		\subitem Sequential Criteria : $f$ is continuous at $x_0$ if $f(x_n)$ should converge to $f(x_0)$ for every sequence $\sequence{x_n}$ converging to $x_0$.
		$$ \lim_{x \to x_0} f(x) = f(\lim_{x \to x_0} x) = f(x_0)$$
		\subitem Neighbourhood Criteria : A function is continuous at $x_0$ if every neighbourhood of $f(x_0)$ contains the image of a neighbourhood of $x_0$.
		$$ \forall \varepsilon > 0,\ \exists \delta > 0,\ \forall x \in (x_0 - \delta,x_0+\delta),\ f(x) \in (f(x_0)-\varepsilon,f(x_0)+\varepsilon)$$
	\item Types of Discontinuity
		\subitem First Kind : Removable Discontinuity $f(x+) = f(x-) \ne f(x)$.
		\subitem Second Kind : Jump Discontinuity $f(x+) \ne f(x-)$
		\subitem Third Kind : Essential Discontinuity\footnote{Classification of essential discontinuties is the work of John Klippert, 1989} $f(x+)$ or $f(x-)$ does not exist.
	\item For a function $f$, every discontinuity except possible essential discontinuity of first kind(where both limits do not exist) are countable.
	\item If $f,g$ are continuous functions from $A$ to $\mathbb{R}$. Suppose $c \in A$. Then \begin{enumerate*} \item $f+g$ \item $f-g$ \item $fg$ \item $bf$ \item $f/g$ provided $g(x) \ne 0,\ \forall x \in A$ \end{enumerate*} are continuous at $c$.
	\item $f \circ g$ continuous $\nimplies f,g$ continuous.
\end{enumerate}

\section{Properties of Continuity}
\begin{enumerate}
	\item The set of discontinuities is an $F_\sigma$ set and the points where continuous is a $G_\delta$ set.\footnote{In a metrix space $X$, the locus of continuity of $f : X \to \mathbb{R}$ is a countable union of open balls.}
	\item Froda's theorem : The set of discontinuities of a monontone function is countable.
	\subitem Discontinuities of a monotone function are jump discontinuities.
	\item Lebesgue-Vitali theorem : A bounded function $f$ is Riemann integrable on $I=[a,b]$ if and only if the set of discontinuities has zero measure.\footnote{A bounded function is Riemann integrable if and only if the essential discontinuity of first kind has Lebesgue measure zero.}
	\subitem Dirichlet function, $\chi_\mathbb{Q}$ is discontinuous everywhere. The disconuities are essential discontinuities of first kind.
	\subitem Characteristic function of Cantor set, $\chi_C$ is Riemann integrable, since $\mu(C)=0$.
	\subitem By Baire's Category theorem, there does not exists a function which continuous exactly on $\mathbb{Q}$.
	\item Continuous image of a compact set is compact.
	\subitem Continuous function on a bounded interval is bounded. ??
	\subitem Continuous image of a closed interval is closed.
	\item Location of root theorem : If $f$ is continuous on $[a,b]$ and $f(a),f(b)$ are of different sign, then there exists $c \in (a,b)$ such that $f(c) = 0$.
	\item Intermediate Value theorem : If $f$ is continuous on $[a,b]$ and $f(a) \ne f(b)$, then $f$ assumes every value between $f(a)$ and $f(b)$.
	\subitem Converse : If $f$ is $1-1$ and satisfies intermediate value property, then $f$ is continuous.
\end{enumerate}

\section{Exercise}
\begin{enumerate}
	\item Constant, Identity Functions are continuous.
	\item Check continuity of $x\sin 1/x$ at $x =0$
	\begin{proof}[Solution]
		$$ \lim_{x \to 0} -x \le \lim_{x \to 0} \frac{\sin \frac{1}{x}}{\frac{1}{x}} \le \lim_{x \to 0} x $$
	\end{proof}
	\item $f(x) = 1/x$ is not continuous at $0$.
	\item Signum Function is continuous only at $0$.
		$$ sgn(x) = \begin{cases} -1 & x < 0 \\ 0 & x = 0 \\ 1 & x>1 \end{cases} $$
	\item There exists a function which is continuous only at $a$.
		$$f(x) = \begin{cases} x-a & x \in \mathbb{Q} \\ 0 & x \in \mathbb{Q}^c \end{cases} $$
	\subitem There exists function which is continuous only at a finite number of points.
		$$f(x) = \begin{cases} (x-a_1)(x-a_2)\dots(x-a_n) & x \in \mathbb{Q} \\ 0 & x \in \mathbb{Q}^c \end{cases} $$
	\item There exists a function which is discontinuous only at finite number of points. (Jump discontinuities) 
		$$ f(x) = \frac{1}{(x-a_1)(x-a_2)\dots(x-a_n)} $$
	\item Thomae's Function is continuous on $\mathbb{Q}^c$ and discontinuous on $\mathbb{Q}$. The discontinuities are removable.
		$$ f : \mathbb{R} \to \mathbb{R},\ f(x) = \begin{cases} 0 & x \in Q^c \\ \frac{1}{q} & x = \frac{p}{q} \in \mathbb{Q} \end{cases} $$
	\subitem There exists Function continuous only on $\mathbb{Z},\mathbb{N}$. 
\end{enumerate}

\section{Fixed Points}
\begin{enumerate}
	\item Let $f : A \to B$. Then $x$ is fixed point of $f$ if $f(x) = x$.
\end{enumerate}

\section{Properties of Fixed points}
\begin{enumerate}
	\item A continuous function on closed interval to itself will have a fixed point.
		\subitem Continuous function on a compact set to a subset of it.
\end{enumerate}

\section{Exercise}
\begin{enumerate}
	\item Find fixed points of $f(x) =2x$ ? $x = 0$.
	\item Find fixed points of $f : (0,1) \to (\frac{1}{2}, 1)$ where $f(x) = \frac{x+1}{2}$ ? No fixed points.
\end{enumerate}

\section{Differentiability}
\begin{enumerate}
	\item Let $f : I \to \mathbb{R}$ where $I$ is an interval. Then $f$ is differentiable at $c \in I$ if
		$$ \forall \varepsilon > 0,\ \exists \delta > 0,\ |x-c| < \delta \implies \left| \frac{f(x)-f(c)}{x-c} \right| < \varepsilon $$
		$$ f^\prime(c) = \lim_{x \to c} \frac{f(x)-f(c)}{x-c} $$
		$$ \lim_{h \to 0} \frac{f(c+h)-f(c)}{h} = \lim_{h \to 0} \frac{f(c)-f(c-h)}{h} = \lim_{h \to 0} \frac{f(c+h)-f(c-h)}{2h} $$
	\item Local extrema(minima/maxima) is a point $x_0$ such that $f(x_0) \le f(x)$ or $f(x_0) \ge f(x)$ for every $x$ in a neighbhoourhood of $x_0$.
	\item Absolute/Global extrema is a point $x_0$ such that $f(x_0) \ge f(x)$ or $f(x_0) \le f(x)$ for every $x$ in its domain.
	\item A function $f$ is convex if $f^{\prime\prime}(x) > 0,\ \forall x \in I$. And concave if $f^{\prime\prime}(x) < 0,\ \forall x \in I$. If $f^{\prime\prime}(x) = 0$, then $x$ is a point of inflection.
	\item A function $f$ has derivative zero at $x_0$, then $x_0$ is a point of extrema.
	\item Every differentiable function is continuous. But, there exists non-differentiable continuous functions. 
		\subitem Weierstrass monster function is continuous, but nowhere differentiable.
	\item A function $f$ is increasing if $f(x)\le f(y)$ whenever $x<y$. And decreasing if $f(x) \ge f(y)$ whenever $x<y$. Function $f$ is monotonic if it is either increasing or decreasing. Function $f$ is strictly monotone (increasing/decreasing) if the inequality is strict.
\end{enumerate}
\section{Properties of Derivatives}
\begin{enumerate}
	\item
\end{enumerate}

\section{Exercise}
\begin{enumerate}
	\item Check differentiability of $f(x) = x^2 \sin 1/x$ at $x = 0$ ?
	\begin{proof}[Solution]
		Diferentiable and $f^\prime(0) = 0$ since $x\sin 1/x \to 0$.
	\end{proof}
	\item 
	\subitem $|x-10|^2$ is differentiable everywhere.
	\subitem $\sin (x-3)$ is not differentiable at $x = 3$.
	\item If $f^{\prime\prime} > 0,\ f(0) = 0, f'(0) < 0$, then $f$ has a positive solution.
\end{enumerate}

\section{Uniform Continuity}
\begin{enumerate}
	\item A function $f : X \to \mathbb{R}$ is uniformly continuous if 
		$$ \forall \varepsilon > 0,\ \exists \delta > 0,\ \forall x_0 \in X,\ |x-x_0|<\delta \implies |f(x)-f(x_0)| < \varepsilon $$
		$$ \forall \varepsilon > 0,\ \exists \delta > 0,\ |x_1-x_2| < \delta \implies |f(x_1)-f(x_2)| < \varepsilon $$
	\item A function $f : X \to \mathbb{R}$ is Lipschitz if 
		$$ \exists k > 0,\ \forall x,y \in X,\ |f(x)-f(y)| \le k|x-y|$$
	\item A function $f : X \to \mathbb{R}$ satisfies Holder's condition if $|f(x)-f(y)| \le k|x-y|^\alpha,\ k,\alpha > 0$.
\end{enumerate}

\section{Properties of Uniform Continuity \& Tests}
\begin{enumerate}
	\item Uniformly continuous functions are continuous.
	\item If a function $f$ is uniformly continuous on $X$, then it is uniformly continuous on every subset of $X$.
	\item A function $f : X \to \mathbb{R}$ is not uniformly continuous if there exists two sequences $x_n,y_n$ such that $x_n-y_n \to 0 \nimplies |f(x_n)-f(y_n)| \to 0$.
	\item Continuous function on a compact set is uniformly continuous.
		\subitem $f$ is continuous on an interval and endpoint limits exists, then is $f$ is uniformly continuous
%	\subitem $f$ is continuous on $[a,b]$ and $\displaystyle \lim_{x \to a+} f(x)$ \& $\displaystyle \lim_{x \to b-} f(x)$, then $f$ is uniformly continuous.
%	\subitem $f$ is continuous on $[a,\infty]$ and $\displaystyle \lim_{x\to \infty} f(x)$ exists, then $f$ is uniformly continuous.
%	\subitem $f$ is continuous on $(-\infty,b]$ and $\displaystyle \lim_{x \to -\infty} f(x)$ exists, then $f$ is uniformly continuous.
%	\subitem $f : [-\infty,\infty]$ is continuous and $\displaystyle \lim_{x \to \infty} f(x), \lim_{x \to -\infty} f(x)$ exists, then $f$ is uniformly continuous.
	\item $\alpha$-H\"older continuous functions are uniformly continuous. 
	\subitem If $\alpha > 1$, then constant function. 
	\subitem If $\alpha = 1$, then Lipschitz continuous.
	\item Lipschitz functions are uniformly continuous.
	$$\text{Analytic} \subsetneq \text{Lipschitz} \subsetneq\text{$\alpha$-H\"older} \subsetneq \text{Uniformly Continuous} \subsetneq \text{Continuous}$$
	\subitem Function $f$ is Lipschitz if and only if $f$ is differentiable and derivative is bounded.
	\item Continuous periodic functions are uniformly continuous.
\end{enumerate}

\section{Exercise}
\begin{enumerate}
	\item Check whether $f(x) = \sin x$ is uniformly continuous ?
	\begin{proof}[Solution]
		$$ |\sin x_1 - \sin x_2| = \left|2\cos\left(\frac{x_1+x_2}{2}\right)\sin\left(\frac{x_1-x_2}{2}\right)\right| \le 2\sin(\delta/2) \le \delta $$
	\end{proof}
	\item Check whether $f(x) = \frac{1}{x}$ is uniformly continuous on $(0,\infty)$ ?
	\begin{proof}[Solution]
		As $x_0 \to 0$, $|f(x)-f(x_0)| \to \infty$.
		$0$ is a bad point for $f$ as it has a suddent variation there. Since $0$ is a limit point of its domain the function is not uniformly continuous.\\

		Alternately, $f(x) = \frac{1}{x}$ is Lipschitz if $\frac{1}{xy}$ is bounded since $|\frac{1}{x}-\frac{1}{y}| = | \frac{y-x}{xy} | \le |\frac{1}{xy}| |x-y|$. That is, $f$ is Lipschitz if $x,y$ are bounded away from zero. Therefore, $f(x) = \frac{1}{x}$ is Uniformly continuous if the domain of $f$ is bounded away from zero.\\

		When domain of $f$ is bounded away from origin, both the end point limits exists and $f(x)=\frac{1}{x}$ is continuous in its domain. Therefore, $f$ is uniformly continuous.
	\end{proof}
	\item Check whether $f(x) = x^2$ is uniformly continuous on $(0,\infty)$ ?
	\begin{proof}[Solution]
		Bad points are $\pm \infty$ and they are limit points of $f$. Thus, $f$ is not uniformly continuous on any unbounded subset of $\mathbb{R}$.\\

		Alternately, consider sequences $\sqrt{n+1}, \sqrt{n}$. Now $\sqrt{n+1} - \sqrt{n} = \frac{1}{\sqrt{n+1}+\sqrt{n}} \to 0$. However, $|n+1-n| = 1 \to 1$.\\

		When the domain of $f(x)=x^2$ is bounded, then both end point limits exists and $f(x)$ is continuous everywhere. Therefore $f(x)=x^2$ is uniformly continuous on every bounded set.
	\end{proof}
	\item Check whether $f(x) = \sqrt{x}$ is Lipschitz on $[0,1]$ ?
	\begin{proof}[Solution]
		Consider $x = 1/n^2$ and $y = 0$. Then $|f(x)-f(y)| = |1/n| \le c|x-y| = c/n^2$ which is not possible.
		Therefore, $f(x) = \sqrt{x}$ is not Lipschitz. However, being a continuous function on a compact interval, $f$ is uniformly continuous.
	\end{proof}
	\item Function $f(x) = \sin x$ is differentiable and derivative is bounded, therefore uniformly continuous.
	\item Function $f(x) = \sin x^2$ is differentiable and derivative is unbounded, therefore, $\sin x^2$ is not Lipschitz.\\

	$\pm \infty$ are bad points of $f(x) = \sin x^2$. Thus, $f$ is uniformly continuous if and only if it is defined on a bounded set.\\

	Consider $\sqrt{2n\pi+\frac{1}{n}}, \sqrt{2n\pi+\frac{\pi}{2}}$. Then $x_n - y_n \to 0$, but $|f(x_n) - f(y_n)| \to 1 = 0$. Therefore, $f$ is not uniformly continuous.
	\item $f(x) = x\sin x$ is not uniformly continuous.
		Consider $(2n\pi+\frac{1}{n}), (2n\pi)$. Then $x_n - y_n \to 0$, but $|f(x_n) - f(y_n)| = |2n\pi\sin (1/n) + \frac{1}{n} \sin (1/n)| \to 1 = 0$.

	\item $f(x) = \sin x^3, x^2 \sin x$ is uniformly continuous on bounded sets.
	\item $f(x) = \sin (1/x)$ is uniformly uniformly continuous on sets bounded away from zero.
	\item $f(x) = x \sin (1/x)$ is uniformly uniformly continuous on $\mathbb{R}$ as both end point limits exists.
	\item $f(x) = x^2$ is differentiable but not H\"older continuous on $\mathbb{R}$.\\
		$f(x) = |x|$ is H\"older continuous but not differentiable.
	\item $f(x)$ is a polynomial of degree $4$ and $f(0) = 0$, $f'(0) = 0$, $f(-1) = 0$ and $f(1) = 1$. Characterise $f^{\prime\prime\prime}(x)$ ?\\
		Let $f(x) = a + bx + cx^2 + dx^3 + ex^4$. $f(0) = 0 \implies a = 0$. Then $f'(x) = b + 2cx + 3dx^2 + 4ex^3$. $f'(0) = 0 \implies b = 0$. Now $f(x) = cx^2 + dx^3 + ex^4$. $f(-1) = c - d + e = 0$ and $f(1) = c+d+e = 1$. Thus, $d = 0.5$ and $e = 0.5-c$. Thus, $f(x) = cx^2 +0.5x^3 + (0.5-c)x^4$. $f^{\prime\prime\prime}(x) = 3 + 24(0.5-c)x$. $f^{\prime\prime\prime}(0) = 3$.
	\item $f(x) = \frac{1}{\sqrt{x}}$ is uniformly continuous of $[r,\infty)$. As $\lim_{x \to \infty} f(x) = 0$ exists.
	\item $f : \mathbb{Q} \to \mathbb{R}$, $f(0) = 0$, $f(r) = \frac{p}{10^q}$. Characterise $f$ ? (injective,surjective)\\
		There does not exists a surjection between two sets of different cardinality. And $f$ is not injective since $f(\frac{1}{2}) = f(\frac{10}{3}) = 0.01$.
\end{enumerate}

\section{Properties of Limit of a Function}
\begin{enumerate}
	\item Limit is algebraic.
		Suppose $\displaystyle \lim_{x \to a} f(x),\ \lim_{x \to a} g(x)$ exists, then
	\begin{align}
		\lim_{x \to a} cf(x) & = c\lim_{x \to a} f(x) \\
		\lim_{x \to a} f(x) \pm g(x)  & = \lim_{x \to a} f(x) \pm \lim_{x \to a}g(x) \\
		\lim_{x \to a} f(x)g(x) & = \lim_{x \to a}f(x) \ \lim_{x \to a}g(x) \\
		\lim_{x \to a} \frac{f(x)}{g(x)} & = \frac{\displaystyle \lim_{x \to a}f(x)}{\displaystyle \lim_{x \to a}g(x)} \\
		\lim_{x \to a} f(x)^{g(x)} & = \lim_{x \to a}f(x)^{\displaystyle \lim_{x \to a}g(x)}
	\end{align}
		with a few exceptions, where $ \frac{0}{0}, \frac{\pm \infty}{\pm \infty},\ 0 \pm \infty,\ \infty - \infty,\ 0^0,\ \infty^0,\ 1^{\pm\infty} $.
\end{enumerate}

\section{Functions of Bounded Variation}
\begin{enumerate}
	\item A function $f$ is of bounded variable on $[a,b]$ if the sum of variations is bounded for any partition of $[a,b]$.
	\item Total variation of $f$ on $[a,b]$ is the supremum of bounded variations.
		$$ V_f(a,b) = \sup_{P \in \mathscr{P}[a,b]} \!\!\! V(P,f) \text{ where } V(P,f) = \!\!\!\!\!\! \sum_{(x_{i-1},x_i) \in P}\!\!\!\!\!\! |f(x_i) - f(x_{i-1})| $$
\end{enumerate}

\section{Properties of Bounded Variation}
\begin{enumerate}
	\item Functions of bounded variation on $[a,b]$ are bounded on $[a,b]$.
	\item Monotonic function $f$ are of bounded variation with total variation 
		$$ V_f(a,b) = |f(b)-f(a)| $$
		\subitem If $f(x) = \sin x$, then $V_f(a,b) = \sum |\sin x_i - \sin x_{i-1}| \le \sum |x_i - x_{i-1}| = b-a$.
	\item Continuous functions on $[a,b]$ and Functions of bounded variation on $[a,b]$ are non-comparable.
	\item Lipschitz functions are of bounded variation on $[a,b]$.
		\subitem Function $f$ is differentiable and derivative is bounded on $[a,b]$, then $f$ has bounded variation on $[a,b]$.
	\item If $f$ has bounded variation on $[a,b]$, then $|f|$ has bounded variation on $[a,b]$.
	\item Function $f$ have bounded variation if and only if $f$ is difference of two increasing functions.
	\item If an increasing function $f$ have bounded variation on $[a,b]$, then it is sum of two increasing functions.
	\begin{proof}[Solution]
		$$ f = g-h \implies g = f+h $$
	\end{proof}
\end{enumerate}

\section{Exercise}
\begin{enumerate}
	\item Monotonic function on a bounded interval is not necessarily have bounded variation. For example, $f(x) = \frac{1}{x}$ on $(0,1)$ have unbounded variation.
	\item If $f$ is monotonic non-decreasing function. Then $g(x) = e^{-f(x)}$ is not necessarily bounded as $x \to -\infty$, $g \to \infty$.
	%Does there exists a monotonic functions which does not have one of the limits at an interior point.
\end{enumerate}
%83019 78727 Minitek

%example problems for each exceptions above can be found at
%http://sites.science.oregonstate.edu/math/home/programs/undergrad/CalculusQuestStudyGuides/SandS/lHopital/index.html
\section{Properties of Limit}
\begin{enumerate}
	\item L'Hospital/Bernouli Theorem
	\[ \lim_{x \to a} \frac{f(x)}{g(x)} = \lim_{x \to a} \frac{f'(x)}{g'(x)} \]
	\item \[ \lim_{x \to 0} (2+x)^\frac{1}{x} = \lim_{x \to 0} e^{\frac{1}{x} \log (2+x)} = e^{\displaystyle \lim_{x \to 0} \frac{\log(2+x)}{x}} = e^{\displaystyle \lim_{x \to 0} \frac{1}{2+x}} = \sqrt{e} \]
	\item Squeeze Theorem :
	Suppose $f(x) \le g(x) \le h(x)$ for each $x$ in an open interval containing $a$ (except $a$).
	If $\displaystyle \lim_{x \to a} f(x) = \lim_{x \to a} h(x) = L$,  then
	\begin{equation}
		\lim_{x \to a} g(x) = L
		\label{equ:lim_squeeze}
	\end{equation}
	\item Chain Rule :
	Suppose $\displaystyle \lim_{x \to a} g(x) = b$ and $f$ is continuous at $b$, then
	\begin{equation}
		\lim_{x \to a} f(g(x)) = f(\lim_{x \to a} g(x)) = f(b) = c
	\end{equation}
\end{enumerate}

\section{Limit of a Set}
\begin{definition}
	\[ \liminf X = \inf \{ \text{limit points} \} \]
	\[ \limsup X = \sup \{ \text{limit points} \} \]
\end{definition}

\section{Sequence of Sets}
\begin{definition}
	\[ \liminf X_n = \bigcup_{n = 1}^\infty \bigcap_{m = n}^\infty X_n \]
	\[ \limsup X_n = \bigcap_{n = 1}^\infty \bigcup_{m = n}^\infty X_n \]
\end{definition}

\section{Limit Superior/Inferior of a Function} What is this ?
\begin{enumerate}
	\item \[ \limsup_{x \to a} f = \lim_{\varepsilon \to 0} \sup_{x \in B(a,\varepsilon)^\ast} \{ f(x) \} = \inf_{\varepsilon > 0} \sup_{x \in B(a,\varepsilon)^\ast} \{ f(x) \} \]
	\[ \liminf_{x \to a} f = \lim_{\varepsilon \to 0} \inf_{x \in B(a,\varepsilon)^\ast} \{ f(x) \} = \sup_{\varepsilon > 0} \inf_{x \in B(a,\varepsilon)^\ast} \{ f(x) \} \]
\end{enumerate}

\section{Topology}
\subsection{Metric Space}
\begin{definition}[distance function]
	A distance function $d : X \times X \to \mathbb{R}^+$ on a set $X$ is a function which satisfies
	\begin{enumerate}
		\item $d(x,y) \ge 0,\qquad \forall x,y \in X$
		\item $d(x,y) = 0 \iff x=y$
		\item $d(x,y) = d(y,x)$
		\item $d(x,y) \le d(x,z) + d(z,y), \qquad x,y,z \in X$
	\end{enumerate}
\end{definition}

\subsection{Convergence}
\begin{definition}[metric]
	A sequence $x_n$ converges to $x$ if there exists $N \in \mathbb{N}$ such that $\forall n > N,\ d(x_n,x) < \varepsilon$.
\begin{equation}
	\forall \varepsilon > 0,\ \exists N \in \mathbb{N},\ \forall n > N,\ d(x_n,x) < \varepsilon
\end{equation}
\end{definition}

\subsection{Cauchy Criterion}
\begin{definition}[metric]
	A sequence $x_n$ is Cauchy if there exists $N \in \mathbb{N}$ such that $\forall n,m > N,\ d(x_n,x_m) < \varepsilon$.
\end{definition}

\subsection{Topological Space}
\begin{definition}[topological space]
	A topological space $\entity{X,\mathcal{T}}$ where $\mathcal{T} \subset \mathcal{P}(X)$ satisfies
	\begin{enumerate}
		\item $\phi,X \in \mathcal{T}$.
		\item $\mathcal{T}$ is closed under finite intersections.
		\item $\mathcal{T}$ is closed under arbitrary unions.
	\end{enumerate}
	Let $G \in \mathcal{T}$.
	Then $G$ is an open set in $\entity{X,\mathcal{T}}$.
	And $X-G$ is a closed set.
\end{definition}

\begin{definition}[clopen]
	A clopen set is both open and closed.
\end{definition}

\begin{definition}[dense]
	A dense set $A$ intersects every non-trivial open set in $\entity{X,\mathcal{T}}$.
\end{definition}
\begin{note}
	A dense set has no proper closure.
	If $A$ is dense in $X$, then $\bar{A} = X$.
	If $A$ is dense in $X$ and $x \in X$, then every neighbourhood of $x$ has an element of $A$.
\end{note}

\begin{definition}[neighbourhood]
	A \textbf{neighbourhood} $N$ of a point $x \in X$ contains an open set containing $x$.
	Then $x$ is an \textbf{interior point} of $N$.
\end{definition}
\begin{definition}[neighbourhood system]
	The \textbf{neighbourhood system} of $x$, $\mathcal{N}_x$ is the family of all neighbourhoods of $x$.
\end{definition}
\begin{definition}[interior]
	The set of all interior points of $N$ is the \textbf{interior} of $N$, $N^\circ$.
\end{definition}
\begin{definition}[exterior]
	The interior of $X-N$ is the \textbf{exterior} of $N$.
\end{definition}
\begin{definition}[boundary]
	The \textbf{boundary} of $N$, $\partial N$ is the set of all points which are neither in its interior or exterior.
\end{definition}

\begin{definition}[derived set]
	A \textbf{limit point} $x$ of a set $A$ has every deleted neighbourhood $N-\{x\}$ intersecting $A$.
	The \textbf{derived set} $A'$ is the set of all limit points of $A$.
\end{definition}

\begin{note}
	A point $x$ is a limit point of $A$ if and onoy if there exists a non-eventual sequence in $A$ converging to $x$.
\end{note}

\begin{definition}[closure]
	The closure of $A$, $\bar{A} = A \cup A'$.
\end{definition}

\begin{note}
	The closure of $A$, $\bar{A}$ is the smallest closed set containing $A$.
	If $A$ is closed, then $\bar{A} \subset A$.
	If $C$ is closed and $A \subset C$, then $\bar{A} \subset C$.
\end{note}

\subsection{Convergence}
\begin{definition}[neighbourhood]
	A sequence $\sequence{x_n}$ \textbf{converges} to $x$ if any neighbourhood $N$ of $x$ contains all except finitely many $x_n$'s.
	Then $x$ is a \textbf{limit} of sequence $\sequence{x_n}$.
\end{definition}
\begin{note}
	Let $x_n \to x$ in $\entity{X,\mathcal{T}}$.
	Then $x_n$ is eventually in every neighbourhood of $x$.
	\begin{equation}
		\forall U \in \mathcal{N}_x,\ \exists N \in \mathbb{N},\ \forall n > N,\ x_n \in U
	\end{equation}
\end{note}

\begin{note}
	Sequences $\sequence{\frac{1}{n}}$, $\sequence{\frac{1}{2^n}}$ are eventually in every neighbourhood of $0$.
\end{note}


\paragraph{Results}
\begin{definition}[cyclotomic field]
	The $n$th cyclotomic field is $\mathbb{Q}(\alpha)$ where $\alpha$ is a primitive $n$th root of unity.
\end{definition}

\begin{definition}[cyclotomic polynomial]
	The $n$th cyclotomic polynomial $\Phi_n(x)$ is the monic irreducible polynomial with primitive $n$th roots of unity as its zeroes.
	$$ \Phi_n(x) = \prod_{\substack{1 \le k \le n \\ \gcd(k,n)=1}} \!\!\!\!\!\left(x-\zeta_k\right)$$
\end{definition}
	The $n$th cyclotomic polynomial has degree $\phi(n)$.

\begin{definition}[Euler Characteristic]
	$\chi = V - E + F$
\end{definition}
\begin{remark}
	Every convex polyhedron has Euler characteristic, $\chi = 2$.
\end{remark}

\begin{lemma}[B\'ezout]
	Let $gcd(a,b) = d$. Then there exists integers $x,y$ such that $ax+by = d$. And integers of the form $as+bt$ are exactly the multiples of $d$.
\end{lemma}

	The integers $x,y$ are the B\'ezout coefficients for $(a,b)$. B\'ezout coefficients are not unique.
	B\'ezout identity implies Euclid's lemma, and chinese remainder theorem.

\begin{lemma}[Euclid]
	Let $p$ be a prime. If $p$ divides $ab$, then $p$ divides either $a$ or $b$.
\end{lemma}
\begin{proof}
	By B\'ezout's identity or By induction using Euclidean algorithm.
\end{proof}

\begin{theorem}[chinese remainder theorem]

\end{theorem}

\begin{definition}[B\'ezout Domain]
	A B\'ezout Domain is an integral domain which satisfyies B\'ezout's identity.
\end{definition}
	Every PID is a B\'ezout Domain.

\begin{definition}[Gaussian Integers]
	Gaussian integers, $\mathbb{Z}[i]$ are complex numbers of the form $a+ib$, $a,b \in \mathbb{Z}$.
\end{definition}

	Let $x,y$ are Gaussian integers. $x$ divides $y$ if there exists a Gaussian integer $z$ such that $y = xz$.
	The Gaussian integers not divisible by any non-unit Gaussian integer is a Gaussian prime.
\paragraph{Properties}
	\begin{enumerate}
		\item $\mathbb{Z}[i]$ is a subring of $\mathbb{C}$
		\item $\mathbb{Z}[i]$ is an integral domain.
		\item $\mathbb{Z}[i]$ is a principal ideal domain (PID).
		\item $\mathbb{Z}[i]$ i s a Unique factorisation domain (UFD).
		\item $\mathbb{Z}[i]$ with norm $N(a+ib) = a^2+b^2$ is a Euclidean Domain.
		\item $\mathbb{Z}[i]$ is a B\'ezout Domain.
	\end{enumerate}

\paragraph{Previous Results}
\begin{enumerate}
	\item Every PID is a UFD.
	\item If $D$ is a UFD, then $D[x]$ is a UFD.
\end{enumerate}

\begin{definition}[Eisenstein Integers]
	Eisenstein Integers, $\mathbb{Z}[w]$ are complex numbers of the form $a+wb$, $a,b \in \mathbb{Z}$ and $w = e^{i2\pi/3}$.
\end{definition}
	The units in $\mathbb{Z}[w]$ are $\pm 1, \pm w, \pm w^2$.
	Let $x,y$ be Eisenstein integers. $x$ divides $y$ if there exists a Eisenstein integer $z$ such that $y = xz$.
	The Eisenstein integers not divisible by any other non-unit Eisenstein integer is an Eisenstein prime.
	\begin{enumerate}
		\item $\mathbb{Z}[w]$ is a subring of $\mathbb{Q}(w)$, the third cyclotomic field.
	\end{enumerate}
	The ring of Eisenstein integers form an Eulerian domain with norm $N(a+wb) = a^2-ab+b^2$.
	The quotient $\mathbb{C}/\mathbb{Z}[w]$ is a complex torus.

\begin{theorem}[Reciprocity(Gauss)]
	For $k = 2,3,4$, $\forall p,q \in \mathbb{P}$, there exists an integer $p^\ast$ such that $x^k \equiv q \pmod{p}$ is solvable iff $x^k \equiv p^\ast \pmod{q}$ is solvable.
\end{theorem}

\begin{definition}[Legedre Symbol]
	$$\left(\frac{a}{p}\right) = \begin{cases} 0 & a \equiv 0 \pmod{p} \\ 1 & a \not\equiv 0 \pmod{p}, \exists x, a \equiv x^2 \pmod{p} \\ -1 & a \not\equiv 0 \pmod{p}, \forall x, a \not\equiv x^2 \pmod{p} \end{cases}$$
\end{definition}
	Legendre symbol gives $p^\ast$ for $k=2$.
	Jacobi symbol and Hilbert symbol are extensions of Legendre symbol.

\paragraph{Amazing Results}
\begin{enumerate}
	\item If a soccer ball(Goldberg polyhedra) is made by stiching pentagonal and hexgonal pieces, then how many pentagonal pieces are used ? \textbf{Ans : 12} \\Reference Image : Adidas Telstar\\
		$V - E + F = \frac{5P+6H}{3} - \frac{5P+6H}{2} + P + H = 2 \implies P = 12$
\end{enumerate}

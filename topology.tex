\section{Topology}
\section{Metric Space}
\begin{definition}[distance function]
	A distance function $d : X \times X \to \mathbb{R}^+$ on a set $X$ is a function which satisfies
	\begin{enumerate}
		\item $d(x,y) \ge 0,\qquad \forall x,y \in X$
		\item $d(x,y) = 0 \iff x=y$
		\item $d(x,y) = d(y,x)$
		\item $d(x,y) \le d(x,z) + d(z,y), \qquad x,y,z \in X$
	\end{enumerate}
\end{definition}

\section{Convergence}
\begin{definition}[metric]
	A sequence $x_n$ converges to $x$ if there exists $N \in \mathbb{N}$ such that $\forall n > N,\ d(x_n,x) < \varepsilon$.
\begin{equation}
	\forall \varepsilon > 0,\ \exists N \in \mathbb{N},\ \forall n > N,\ d(x_n,x) < \varepsilon
\end{equation}
\end{definition}

\section{Cauchy Criterion}
\begin{definition}[metric]
	A sequence $x_n$ is Cauchy if there exists $N \in \mathbb{N}$ such that $\forall n,m > N,\ d(x_n,x_m) < \varepsilon$.
\end{definition}

\section{Topological Space}
\begin{definition}[topological space]
	A topological space $\entity{X,\mathcal{T}}$ where $\mathcal{T} \subset \mathcal{P}(X)$ satisfies
	\begin{enumerate}
		\item $\phi,X \in \mathcal{T}$.
		\item $\mathcal{T}$ is closed under finite intersections.
		\item $\mathcal{T}$ is closed under arbitrary unions.
	\end{enumerate}
	Let $G \in \mathcal{T}$.
	Then $G$ is an open set in $\entity{X,\mathcal{T}}$.
	And $X-G$ is a closed set.
\end{definition}

\begin{definition}[clopen]
	A clopen set is both open and closed.
\end{definition}

\begin{definition}[dense]
	A dense set $A$ intersects every non-trivial open set in $\entity{X,\mathcal{T}}$.
\end{definition}
\begin{note}
	A dense set has no proper closure.
	If $A$ is dense in $X$, then $\bar{A} = X$.
	If $A$ is dense in $X$ and $x \in X$, then every neighbourhood of $x$ has an element of $A$.
\end{note}

\begin{definition}[neighbourhood]
	A \textbf{neighbourhood} $N$ of a point $x \in X$ contains an open set containing $x$.
	Then $x$ is an \textbf{interior point} of $N$.
\end{definition}
\begin{definition}[neighbourhood system]
	The \textbf{neighbourhood system} of $x$, $\mathcal{N}_x$ is the family of all neighbourhoods of $x$.
\end{definition}
\begin{definition}[interior]
	The set of all interior points of $N$ is the \textbf{interior} of $N$, $N^\circ$.
\end{definition}
\begin{definition}[exterior]
	The interior of $X-N$ is the \textbf{exterior} of $N$.
\end{definition}
\begin{definition}[boundary]
	The \textbf{boundary} of $N$, $\partial N$ is the set of all points which are neither in its interior or exterior.
\end{definition}

\begin{definition}[derived set]
	A \textbf{limit point} $x$ of a set $A$ has every deleted neighbourhood $N-\{x\}$ intersecting $A$.
	The \textbf{derived set} $A'$ is the set of all limit points of $A$.
\end{definition}

\begin{note}
	A point $x$ is a limit point of $A$ if and onoy if there exists a non-eventual sequence in $A$ converging to $x$.
\end{note}

\begin{definition}[closure]
	The closure of $A$, $\bar{A} = A \cup A'$.
\end{definition}

\begin{note}
	The closure of $A$, $\bar{A}$ is the smallest closed set containing $A$.
	If $A$ is closed, then $\bar{A} \subset A$.
	If $C$ is closed and $A \subset C$, then $\bar{A} \subset C$.
\end{note}

\section{Convergence}
\begin{definition}[neighbourhood]
	A sequence $\sequence{x_n}$ \textbf{converges} to $x$ if any neighbourhood $N$ of $x$ contains all except finitely many $x_n$'s.
	Then $x$ is a \textbf{limit} of sequence $\sequence{x_n}$.
\end{definition}
\begin{note}
	Let $x_n \to x$ in $\entity{X,\mathcal{T}}$.
	Then $x_n$ is eventually in every neighbourhood of $x$.
	\begin{equation}
		\forall U \in \mathcal{N}_x,\ \exists N \in \mathbb{N},\ \forall n > N,\ x_n \in U
	\end{equation}
\end{note}

\begin{note}
	Sequences $\sequence{\frac{1}{n}}$, $\sequence{\frac{1}{2^n}}$ are eventually in every neighbourhood of $0$.
\end{note}


\section{Properties of a Convex Polyhedron}
\begin{definition}[Euler Characteristic]
	$\chi = V - E + F$
\end{definition}
\begin{remark}
	Every convex polyhedron has Euler characteristic, $\chi = 2$.
\end{remark}

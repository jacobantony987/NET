\chapter{Basics}
\section{Set Theory}
\textbf{Set} is a collection of points which satisfies ZFC-axioms.
And the points are the elements of $A$, $x \in A$.

\begin{definition}
	\textbf{Cardinality} $|A|$ is the number of elements of the set $A$.
\end{definition}

\begin{definition}
	A set $B$ is a \textbf{subset} of a set $A$, $B \subset A$ if $x \in B \implies x \in A$.
\end{definition}

\begin{definition}
	The \textbf{power set} $\mathcal{P}(A)$ of a set $A$ is the family of all subsets of $A$.
\end{definition}

\begin{definition}
	Two sets $A,B$ are \textbf{equal}, $A = B$ if $A \subset B$ and $B \subset A$.
\end{definition}

\begin{definition}
	The \textbf{union} of two sets $A,B$ is the set $A \cup B = \{ x : x \in A \text{ or } x \in B\}$.
\end{definition}

\begin{definition}
	The \textbf{intersection} of two sets $A,B$ is the set $A \cap B = \{ x : x \in A \text{ and } x \in B\}$.
\end{definition}

\begin{definition}
	The \textbf{complement} of a set $A$ with respect to a set $B$ is the set $A-B = \{ x \in A : x \notin B\}$.
\end{definition}

\begin{definition}
	The \textbf{symmetric difference} of two sets $A,B$ is the set $A \Delta B = (A-B) \cup (B-A)$.
\end{definition}

\subsection{Relation}
\begin{definition}
	The cartesian \textbf{product} of $A$ and $B$, $A \times B = \{ (a,b) : a \in A,\ b \in B\}$.
\end{definition}

\begin{definition}
	A \textbf{relation} from $A$ to $B$ is a subset of $A \times B$.
	And $xRy \implies (x,y) \in R \subset A \times B$.
	A relation on $A$ is $R \subset A \times A$.
\end{definition}

\begin{definition}
	A \textbf{reflexive} relation $R$ on $A$ satisfies $xRx,\ \forall x \in A$.
\end{definition}

\begin{definition}
	A \textbf{symmetric} relation $R$ on $A$ satisfies $xRy \iff yRx$.
\end{definition}

\begin{definition}
	An \textbf{antisymmetric} relation $R$ on $A$ satisfies $(x,y) \in R \implies (y,x) \notin R$.
\end{definition}

\begin{definition}
	A \textbf{transitive} relation $R$ on $A$ satisfies $xRy,\ yRz \implies xRz,\ \forall x,y,z \in A$.
\end{definition}

\begin{definition}
	An \textbf{equivalence} relation $R$ on $A$ is a reflexive, symmetric, and trasitive relation.
\end{definition}

\begin{definition}
	Let $x \in A$.
	An \textbf{equivalence class} of a set $A$ containining $x$ is the subset $\hat{x} = \{ y \in A : xRy \}$.
\end{definition}

\begin{definition}
	A \textbf{partition} $\{\hat{x},\hat{y},\dots\}$ of $A$ is a family of subsets $\hat{x}$ of $A$ which satisfies
	\begin{enumerate}
		\item $x \in \hat{x},\ \forall x \in A$.
		\item $\hat{x} \cap \hat{y} \iff \hat{x} = \hat{y}$.
		\item $A = \cup \{ \hat{x} : x \in A\}$.
	\end{enumerate}
\end{definition}

\begin{definition}
	A \textbf{total} relation $R$ on $A$ satisfies either $xRy \text{ or } yRx,\ \forall x,y \in A,\ (x \ne y)$.
\end{definition}

\begin{definition}
	A \textbf{function} from $A$ to $B$ is relation which has a unique element $(a,b)$ for every $a \in A$.
\end{definition}

\begin{definition}
	An \textbf{injection} $f : A \to B$ satisfies $f(x) = f(y) \implies x = y$.
\end{definition}

\begin{definition}
	A \textbf{surjection} $f : A \to B$ satisfies $y = f(x),\ \forall y \in B$.
\end{definition}

\begin{definition}
	A \textbf{bijection} $f : A \to B$ is both injective and surjective.
\end{definition}
	There exists a bijection from the set of all equivalence relations on $A$ to the set of all partitions of $A$.

\begin{definition}
	A set $A$ is \textbf{finite} if there exists a natural number $N$ and a bijection $f : A \to \{1,2,\dots,N\}$.
\end{definition}
	$A$ is finite if and only if there does not exists a bijection from $A$ into any proper subset of $A$.

\begin{definition}
	A set $A$ is \textbf{countably infinite} if there exists a bijection $f : A \to \mathbb{N}$.
\end{definition}

\paragraph{}
\begin{enumerate}
	\item Let $f:X \to Y$, $g:Y \to X$ and $g \circ f = id_X$. Then $f \circ g$ is idempotent.
\end{enumerate}


\chapter{Sequence of Functions}
\section{Uniform Convergence}
\begin{enumerate}
	\item Sequence of functions are pointwise convergent if for each $x_0 \in X$, the sequence $f_n(x_0)$ converges to $f(x_0)$.
	\begin{eqnarray}
		\text{(metric)} & \forall x \in X,\ \forall \varepsilon > 0,\ \exists N_{x,\varepsilon} \in \mathbb{N},\ \forall n > N_{x,\varepsilon},\ d(f_n(x),f(x)) < \varepsilon \\
		\text{(norm)} & \forall x \in X,\ \forall \varepsilon > 0,\ \exists N_{x,\varepsilon} \in \mathbb{N},\ \forall n > N_{x,\varepsilon},\  \|f_n(x),f(x)\| < \varepsilon\\
	\text{(nbd)} & \forall x \in X,\ \forall U \in \mathcal{N}_{f(x)},\ \exists N_{x,U} \in \mathbb{N},\ \forall n > N_{x,U},\  f_n(x) \in U
	\end{eqnarray}
	\item Sequence of functions are uniformly convergent  if for each $x \in X$, all the sequences $f_n(x)$ converges to $f(x)$ uniformly.
	\begin{eqnarray}
	\text{(metric)} & \forall \varepsilon > 0,\ \exists N_{\varepsilon} \in \mathbb{N},\ \forall x \in X,\ \forall n > N_{\varepsilon},\ d(f_n(x),f(x)) < \varepsilon \\
	\text{(norm)} & \forall \varepsilon > 0,\ \exists N_{\varepsilon} \in \mathbb{N},\ \forall x \in X,\ \forall n > N_{\varepsilon},\  \|f_n(x),f(x)\| < \varepsilon
		%\text{(nbd)} & \exists N_\varepsilon \in \mathbb{N},\ \forall U \in \mathcal{N}_{f(x)},\ \forall n > N_{x,U},\  f_n(x) \in U %I am not sure about this ?!
	\end{eqnarray}
	\item A sequence of functions are pointwise bounded if for each $x_0 \in X$, the sequence $f_n(x_0)$ is bounded.
		$$ \forall x \in X,\ \exists M_x \in \mathbb{R},\ |f_n(x)| < M_x $$
	\item A sequence of functions are uniformly bounded if they have a uniform bound.
		$$ \exists M \in \mathbb{R},\ \forall x \in X, |f_n(x)| < M $$
\end{enumerate}

\subsection{Properties of Uniform Convergence}
		Let $f_n,g_n$ converges uniformly. Then
\begin{enumerate}
	\item Subsequences of uniformly convergent sequence inherit uniform convergence.
	\item Uniform convergence is linear.
	$$f_n \pm g_n \underset{U}\to f \pm g \qquad kf_n \underset{U}{\to} kf$$
	\item Uniform convergence preserves continuity.
	$$\lim_{t \to x} \lim_{n \to \infty} f_n(t) = \lim_{n \to \infty} \lim_{t \to x} f_n(t)$$
\item The converse is not true. However, suppose a sequence continuous functions $f_n(x)$ on a compact set $K$ converges to a continuous limit function $f$. If \textcolor{red}{$f_n(x) \ge f_{n+1}(x)$}, then the convergence is uniform on $K$.%Rudin 7.13
	\item If both $f,g$ are bounded, then $f_ng_n \underset{U}{\to} fg$.
	\item If $f_n \underset{U}{\to} f$ and $f$ is bounded away from zero. Then $\frac{1}{f_n} \underset{U}{\to} \frac{1}{f}$.
	\item Uniform convergence preserves boundedness.
	%\item Composition of uniformly convergent sequences is uniformly convergent. $f_n \circ g_n \underset{U}{\to} f \circ g$ ?
	\item Let $\alpha$ be monotonically increasing on $[a,b]$. Let $\mathscr{R}(\alpha)$ be the family of functions which are Reimann-Stieljes integrable with respect to $\alpha$ on $[a,b]$.
	$$f_n \in \mathscr{R}(\alpha)[a,b], f_n \underset{U}{\to} f \implies f \in \mathscr{R}(\alpha)[a,b]$$
	\item Let $\sequence{f_n(x)}$ be a sequence of functions on $[a,b]$ which are differentiable on $[a,b]$ and $\exists x_0 \in [a,b],\ f_n(x_0) \to f(x_0)$.
	$$f_n' \underset{U}{\to} f \implies f_n \underset{U}{\to} f$$
\item \textcolor{red}{$\sum U_n(x) f_n(x)$ converges uniformly on $[a,b]$ if sequence $\sequence{f_n}$ is +ve, monotonically decreasing, converges to $0$ uniformly on $[a,b]$ and $\forall m \in \mathbb{N},\ \exists k > 0,\ \sum_{n=1}^m U_n(x) < k$.}
\end{enumerate}

\subsection{Tests for Uniform Convergence of Sequence of Functions}
\begin{enumerate}
	\item Suppose $\sequence{f_n}$ converges pointwise to $f(x)$. Let $M_n = \sup_{x \in E} |f_n(x) - f(x)|$. Then 
		$$M_n \to 0 \iff f_n \underset{U}{\to} f \text{ on } E$$
	\item Let $x_n \to x$ and $M_n = |f_n(x_n) - f(x_n)|$.
	$$\forall x_n \to x,\ M_n \to 0 \iff f_n \underset{U}{\to} f$$
	\item The uniform limit of continuous functions is continuous.
	\item Dini's Theorem : If limit of sequence of monotone continuous functions is continuous, then the convergence is uniform.\\
	Dini's theorem gives the sufficient condition for converse of the above.
	\item Weierstrass Approximation Theorem : If $f$ is continuous on $[a,b]$, then there exists a sequence of functions\footnote{It also suggest the existence of a dense set of nice functions which are sufficient to approximate any continuous function $f$.} $f_n$ which converges to $f$ uniformly.
\end{enumerate}

\subsubsection*{Exercise}
\begin{enumerate}
	\item $f_n(x) = x^n \underset{U}{\to} 0 \quad \forall x \in (-1,1)$.\\
	The convergence is not uniform on any neighbourhood of $\pm 1$ as the limit function $f$ is not continuous at $\pm 1$.
	\item $f_n(x) = \frac{x}{x+n} \underset{U}{\to} 0,\forall x \in [0,a]$ since
	$$M_n = \sup_x |f_n(x) - f(x)| = \sup_x \left|\frac{x}{x+n}\right| \le \sup_x \frac{x}{n} \to 0$$
	\item $f_n(x) = \frac{nx}{1+nx} \to 1 \quad \forall x > 0$
	\item $f_n(x) = \frac{nx}{1+n^2x^2} \underset{U}{\to} 0 \quad \forall x \in [a,\infty)$ 
	$$M_n = \sup_x |f_n(x) - f(x)| = \sup_x \left|\frac{nx}{1+n^2x^2} - 0\right| \le \sup_x \frac{1}{nx} \to 0$$
	It does not converge uniformly on $(0,a]$ as $M_n \not\to 0$.
	$$M_n = \sup_x |\frac{nx}{1+n^2x^2}| \ge \frac{1}{2}$$
	\item $f_n(x) = \frac{x^n}{1+x^n} \underset{U}{\to} 1$ if $x \in \mathbb{R} - (-1,1)$.
	$$M_n = \sup_x \left|\frac{x^n}{1+x^n} - 1\right| = \sup_x \left|\frac{1}{1+x^n}\right| \le \sup_x \frac{1}{x^n}$$
	\item $f_n(x) = \frac{x^n}{1+x^n} \underset{U}{\to} 0$ if $x \in (-1,1)$.
		$$M_n = \sup_x \left|\frac{x^n}{1+x^n} - 0\right| = \sup_x \left|\frac{x^n}{1+x^n}\right| \le \sup_x x^n \to 0$$
	This sequence is not uniformly convergent on any neighbourhood of $\pm 1$ as the limit function is discontinuous at $\pm 1$.
	\item $f_n(x) = x^n(1-x) \underset{U}{\to} 0$ if $x \in (0,1)$ since $f_n(x)$ are continuous, $f_n$ is monotone and $f$ is continuous.
	\item $f_n(x) = \frac{1}{1+(nx-1)^2} \to $.
	\item $f_n(x) = \frac{x}{x+n} \underset{U}{\to} 0$ if $x \in [a,\infty)$.
	\item $f_n(x) = e^{-nx} \underset{U}{\to} 0$ if $x \in [a,\infty)$.
	\item $f_n(x) = \frac{1}{n^2+x^2}$ on $[\frac{1}{2},1]$ is monotonically decreasing and $f_n \underset{U}{\to} 0$.
	\item $g_n(x) = \sin^2(x + \frac{1}{n})$ and $f_n(x) = \int_0^x g_n(t)\ dt$. Then $$f_n(x) = \int (1 - 2\cos 2(t + \frac{1}{n}))\ dt = \frac{x}{2} - \frac{\sin 2(x+\frac{1}{n})}{4} - \frac{\sin \frac{2}{n}}{2}$$ 
		Clearly, $f_n(x) \to \frac{x}{2} - \frac{\sin 2x}{4}$. It is enough to show that $|\sin (x+\frac{1}{n}) - \sin x| \to 0$.
		$$\sin (x+1/n) - \sin x = \sin x \cos (1/n) + \sin (1/n) \cos x - \sin x = \sin x (\cos (1/n) - 1) + \sin (1/n) \cos x \to 0$$
		Thus $f_n \underset{U}{\to} f$ on $[0,\infty)$. What about $(-\infty,0)$ ? %Q45 Real Mathematics
	\item \textcolor{red}{$f_n(x) = b_nx + c_nx^2 \underset{U}{\to} 0$ iff both $b_n$ and $c_n$ are eventually zero.}\\ %Q48 Real Mathematics
		Take $b_n = \frac{1}{n^2}$ and $c_n = 0$. Then $f_n \to 0$ but, $|f_n(x_n) - 0(x_n)| = 1$ when $x_n = n^2$. Thus, $f_n \underset{U}{\not \to} 0$. Therefore, $\sum |b_n| < \infty$ and $\sum |c_n| < \infty$ is not sufficient.
\end{enumerate}

\section{Series of Functions}
\begin{definition}
	A series $\sum f_n$ of function $f_n$ on $E$ converges if the sequence of partial sums converges for every $x \in E$.
\end{definition}

\begin{definition}
	Series of functions is pointwise/uniformly convergent if the respective sequence of partial sums is pointwise/uniformly convergent.
\end{definition}

\subsection{Properties of Uniform Convergence of Series of Functions}
\begin{enumerate}
	\item Suppose $\sum_{n=1}^\infty f_n(x) = f(x)$ uniformly on $[a,b]$. Then
	$$\int_a^b \sum_{n=1}^\infty f_n(x)\ d\alpha(x) = \sum_{n=1}^\infty \int_a^b f_n(x)\ d\alpha(x)$$
\end{enumerate}

\subsection{Test for Convergence of Series of Functions}
\begin{enumerate}
	\item Weierstrass M Test : Let $M_n$ be a convergent sequence such that $\forall x \in E,\ |f_n(x)| \le M_n$. T
	Then $\sum f_n$ is converges uniformly on $E$.
\end{enumerate}

\subsubsection*{Exerise}
\begin{enumerate}
	\item If $f_n(x) = \frac{nx}{1+n^2x^2} - \frac{(n-1)x}{1+(n-1)^2x^2}$.
	Then $\sum f_n \to 0$.
	\item Which of the following are uniformly convergent on $(\pi,\pi)$ ?\\ %Q41 Real Mathematics
	$$\sum \frac{e^{-n|x|}}{n^3} \qquad \sum \frac{\sin nx}{n^5} \qquad \sum \left(\frac{x}{n}\right)^n \qquad \sum \left(\frac{1}{(x+\pi)n}\right)^n$$
	$$\left|\frac{e^{-nx}}{n^3}\right| < \frac{1}{n^3} \qquad \left|\frac{\sin nx}{n^5}\right| < \frac{1}{n^5}$$
%	$$\left|\frac{x^n}{n^n}\right| < \frac{1}{n^2}, \forall n \ge 6$$ %Is what I had in my mind !
		$$M_n = \frac{\pi^n}{n^n} \text{ By Ratio Test : } \lim_{n \to \infty} \frac{M_{n+1}}{M_n} = \frac{\pi}{n(1+1/n)} \to 0 < 1 \implies \sum M_n \text{ converges.}$$
	Finding an upper bound $M_n$ for the given $f_n(x)$ is very hard. So we try to relay on convergence of a related sequence.
		Let $x_n = -\pi + 1/n$. We have $f_n(x) \to 0(x)$. And $M_n = |f_n(x_n) - f(x_n)| \to 1 \ne 0$.
	$$f_n \underset{U}{\not\to} 0 \implies \sum f_n  \text{ does not converges uniformly.}$$
\end{enumerate}


\chapter{Partial Differential Equation}
\begin{enumerate}
	\item \textbf{Partial differential equations} contains one or more partial derivatives. Usually variable $z$ is dependent on two indpendent variables $x,y$.
		$$ \frac{\partial z}{\partial x} + \frac{\partial z}{\partial y} = 0$$
	\item Partial differential equations must contain two independent variables.
	\item \textbf{Order} is the highest order of the derivative occuring in the equation.
		$$ x\frac{\partial z}{\partial x} + y \frac{\partial z}{\partial y} = z,\quad order = 1$$
		$$ \frac{\partial ^2z}{\partial x^2} + 3\frac{\partial^2 z}{\partial x \partial y} + \frac{\partial^2 z}{\partial y^2} = 0, \quad order = 2$$
	\item Standard Notations: (better not to use these letters for other purposes)
		$$ \frac{\partial z}{\partial x} = p,\quad \frac{\partial z}{\partial y} = q,\quad \frac{\partial^2 z}{\partial x^2} = r,\quad \frac{\partial^2 z}{\partial x \partial y} = s,\quad \frac{\partial^2 z}{\partial y^2} = t$$
		$$ pz+qy = z,\ r+3s+t = 0$$
	\item $$ \frac{\partial \phi(\psi(x,y,z))}{\partial x} = \phi'(\psi(x,y,z))\ \frac{\partial \psi(x,y,z)}{\partial x} $$
	\item A \textbf{first order} partial differential equation has only the first order partial derivatives, say $p,q$. A PDE is \textbf{linear} if its does not contain product of partial derivatives.
\end{enumerate}

\section{Formation of Partial Differential Equations}
\begin{enumerate}
	\item Elimination of arbitrary constants.\\
		Differential wrt $x,y$ and eliminate arbitrary constants.
	\item Elimination of arbitrary functions.\\
		If the equation contains only one arbitrary constant(function), then differentiate it wrt $x,y$. Otherwise find higher order derivatives to eliminate arbirary function.
\end{enumerate}

\section*{Exercise}
\begin{enumerate}
	\item $z = ax+by+ab \implies p = a, q=b \implies z = pz+qz+pq$
	\item $z = (x+a)(y+b) \implies p = (y+b),q = (x+a) \implies z = pq$
	\item $az+b = a^2x + y \implies ap = a^2,aq=1 \implies pq = 1$
	\item $(x-h)^2 + (y-k)^2 + z^2 = c^2$ where $c$ is a fixed constant, say $c = 5$.\\
		$\implies 2zp + 2(x-h) = 0, 2zq + 2(y-k) = 0 \implies z^2(p^2 + q^2 + 1) = c^2$.
	\hrule
\item $lx+my+nz = \phi(x^2+y^2+z^2)$ 
\begin{align*}
	& l+np = (2x+2zp)\phi'(x^2+y^2+z^2),\\
	& m+nq  = (2y+2zq)\phi'(x^2+y^2+z^2) \\
	\implies & (l+np)(y+zq)  = (m+nq)(x+zp) \\
	\implies & (ly-mx)+(ny-mz)p + (lz-nx)q + (nz-nz)pq = 0
\end{align*}
\item $z = y^2 + 2\phi(\frac{1}{x} + \log y)$ 
\begin{align*}
	& p = 2\phi'(\frac{1}{x}+\log y) \frac{-1}{x^2},\\
	& q = 2y + 2\phi'(\frac{1}{x} + \log y) \frac{1}{y}\\
	\implies & \frac{p}{q-2y} = \frac{-y}{x^2}\\
	\implies & px^2+qy-2y^2 = 0
\end{align*}
\item $z = \phi(x+iy)+\psi(x-iy)$ (two functions expect second order PDE)
\begin{align*}
	& p = \phi'(x+iy) + \psi'(x-iy)\\
	& q = i\phi'(x+iy) -i \psi'(x-iy)\\
	& r = \phi^{(2)}(x+iy) + \psi^{(2)}(x-iy) \\
	& s = i\phi^{(2)}(x+iy) -i \psi^{(2)}(x-iy) \\
	& t = -\phi^{(2)}(x+iy) -\psi^{(2)}(x-iy)\\
	\implies & r+t = 0
\end{align*}
\end{enumerate}

\section{Solving Pfaffian}
\begin{enumerate}
	\item Grouping Method : Convert into variable separable form.
	\item Multiplier Method : $P'P+Q'Q+R'R = 0 \implies \int P'dx + Q'dy + R'dz = 0$.
\end{enumerate}

\section{Solving Partial Differential Equations}
\begin{enumerate}
	\item Lagrange`s Equation - Linear PDE
	\begin{equation}
		Pp + Qq = R
	\end{equation}
		where $P,Q,R$ are functions of $x,y,z$.
	\begin{enumerate}
		\item Lagrange`s Equation : $Pp + Qq = R$
		\item Auxilary Equation : $\frac{dx}{P} = \frac{dy}{Q} = \frac{dz}{R}$
		\item Solve and find two solutions : $U(x,y,z) = c_1, V(x,y,z) = c_2$.
		\item General Solution : $\phi(U,V) = 0$ or $U = \phi(V)$ where $\phi$ is an arbitrary function.
	\end{enumerate}

	\item Charpit`s Equation - Non-Linear PDE First Order Partial Differential Equations\\
	\begin{equation}\frac{dx}{f_p} = \frac{dy}{f_q} = \frac{dz}{pf_p + qf_q} = \frac{dp}{-(f_x + pf_z)} = \frac{dq}{-(f_y+qf_z)}\end{equation}
	\begin{enumerate}
		\item Select to proper fractions so that the resulting integral is the simplest relation involving at least $p$ or $q$.
		\item Solve the simplest relation along with the given equation to determine $p,q$.
		\item Substitute $p,q$ in the equation $dz = pdx + qdy$ and solve.
\end{enumerate}
\end{enumerate}

\section*{Exercise}
\begin{enumerate}
	\item $px + qy = 3z$
	\begin{align*}
		& \frac{dx}{x} = \frac{dy}{y} = \frac{dz}{3z}\\
		& \frac{dx}{x} = \frac{dy}{y} \implies \log x = \log y + \log c \implies x/y = c_1 \\
		& \frac{dx}{x} = \frac{dz}{3z} \implies 3\log y = \log z + \log c \implies y^3/z = c_2
	\end{align*}
		General Solution : $\phi(x/y,y^3/z) = 0$.
	\item $2p + 3q = 1$.
		General Solution : $\phi(3x-2y,x-2z) = 0$.
	\item $p+q = z$.
		General Solution : $\phi(x-y,ze^{-x}) = 0$.
	\item $3p+4q = 2$.
		General Solution : $\phi(2x-3z,y-2z) = 0$.
	\item $yq - xp = z$.
		\begin{align*}
			& \frac{dx}{-x} = \frac{dy}{y} = \frac{dz}{z} \\
			& \frac{dx}{x} = \frac{-dy}{y} \implies \log x = -\log y + c \implies xy = c_1 \\
			& \frac{dx}{x} = \frac{dz}{z} \implies \log x = \log z + c \implies x/z = c_2
		\end{align*}
		General Solution : $\phi(xy,x/z) = 0$.
	\item $x^2p + y^2q = z^2$.
		General Solution : $\phi(\frac{1}{x}-\frac{1}{y},\frac{1}{x}-\frac{1}{z}) = 0$.
		\hrule
	\item $zp + yq = x$.
	\begin{align*}
		& \frac{dx}{z} = \frac{dy}{y} = \frac{dz}{x} \\
		& \frac{dx}{z} = \frac{dz}{x} \implies xdx = zdz \implies x^2/2 = z^2/2 +c \implies x^2-z^2 = c_1\\
		& \frac{dx+dz}{x+z} = \frac{dy}{y} \implies \log(x+z) = \log(y)+c \implies (x+z)/y = c_2
	\end{align*}
		General Solution : $\phi(x^2-z^2,(x+z)/y) = 0$.
	\item $\frac{y^2z}{x}p + xzq = y^2$
		\begin{align*}
			& \frac{xdx}{y^2z} = \frac{dy}{xz} = \frac{dz}{y^2} \\
			& \frac{xdx}{y^2z} = \frac{dz}{y^2} \implies xdx = zdz \implies x^2/2 = z^2/2 + c \implies x^2-z^2 = c_1 \\
			& \frac{xdx}{y^2z} = \frac{dy}{xz} \implies x^2dx = y^2dy \implies x^3/3 = y^3/3 + c \implies x^3-y^3 = c_2
		\end{align*}
		General Solution : $\phi(x^2-z^2,x^3-y^3) = 0$.
	\item $a(p+q) = z$
		General Solution : $\phi(x-y, ze^\frac{-x}{a}) = 0$.
	\item $\tan x p + \tan y q = \tan z$ 
		(hint : $\int \frac{dx}{\tan x} = \int \frac{du}{u} = \log \sin x$.)
		General Solution : $\phi\left(\frac{\sin x }{\sin y},\frac{\sin x}{\sin z}\right) = 0$.
		\hrule
	\item $zp = -x$
		(hint : $dy/0 \implies y = c_1$).
		General Solution : $\phi(y,x^2+z^2) = 0$.
		\hrule
	\item $y^2p - xyq = x(z-2y)$
		\begin{align*}
			& \frac{dx}{y^2} = \frac{dy}{-xy} = \frac{dz}{x(z-2y)} \\
			& \frac{dy}{-y} = \frac{dz}{z-2y} \\
		\end{align*}
		General Solution : $\phi(xy,-)= 0$.
	\item $(x^2+2yx)p - xyq = xz$.
		General Solution : $\phi(yz,-) = 0$.
		\hrule
	\item $(y^2+z^2-x^2)p - 2xyq + 2zx = 0$
		\begin{align*}
			& \frac{dx}{y^2+z^2-x^2} = \frac{dy}{-2xy} = \frac{dz}{-2zx} \\
			& \frac{dy}{y} = \frac{dz}{z} \\
			& \log y = \log z + c \\
			& y/z = c_1 \\
			& \frac{2xdx+2ydy+2zdz}{-2x(x^2+y^2+z^2)} = \frac{dy}{-2xy} \\
			& \frac{d(x^2+y^2+z^2)}{x^2+y^2+z^2} = \frac{dy}{y} \\
			& \log (x^2+y^2+z^2) = \log y + c \\
			& (x^2+y^2+z^2)/y = c_2
		\end{align*}
		General Solution : $\phi(y/z,(x^2+y^2+z^2)/y) = 0$.
	\item $xu_x + yu_y = u$.
		\begin{align*}
			& \frac{dx}{x} = \frac{dy}{y} = \frac{du}{u}
		\end{align*}
		General Solution : $\phi(x/y,x/z) = 0$.
	\item $(x^2-yz)p + (y^2 - zx)q = z^2-xy$
		\begin{align*}
			& \frac{dx}{x^2-yz} = \frac{dy}{y^2-zx} = \frac{dz}{z^2-xy} \\
			& \frac{dx-dy}{(x-y)(x+y+z)} = \frac{dy-dz}{(y-z)(y+z+x)}\\
			& \frac{d(x-y)}{x-y} = \frac{d(y-z)}{y-z} \\
			& \log (x-y) - \log (y-z) + c\\
			& (x-y)/(y-z) = c_1
			\intertext{Similarly,}
			& (x-z)/(y-z) = c_2
		\end{align*}
		General Solution : $\phi\left(\frac{x-y}{y-z},\frac{x-z}{y-z}\right) = 0$.
	\item $(y+z)p + (z+x)q = (x+y)$
		\begin{align*}
			& \frac{dx}{y+z} = \frac{dy}{z+x} = \frac{dz}{x+y} \\
			& \frac{dx-dy}{y-x} = \frac{dx-dz}{z-x} \\
			& \frac{d(x-y)}{x-y} = \frac{d(x-z)}{x-z} \\
			& \log (x-y) = \log (x-z) + c \\
			& (x-y)/(x-z) = c_1
		\end{align*}
		General Solution : $\phi\left(\frac{x-y}{x-z}, \frac{x-y}{y-z}\right) = 0$.
\end{enumerate}

\section{Solutions of a PDE}
Consider liner PDE \begin{equation} f(x,y,z,p,q) = 0 \end{equation}
\begin{enumerate}
	\item Complete Solution :
		A two parameter family \begin{equation}z = \psi(x,y,z,a,b) \end{equation}
	\item General Solution :
		A two parameter family \begin{equation}z = \psi(x,y,z,a(x,y),\phi(a)((x,y))) \end{equation} obtained from $f(x,y,z,a, \phi(a)) = 0$ and $\partial f/\partial a = 0$.
	\item Particular Solution :
		The particular solution is a curve in the two parameter family of complete solution satisfying certain conditions.
	\item Singular Solution : A solution \begin{equation}\phi(x,y,z) = 0\end{equation} obtained from complete solution by eliminating $a,b$ using $\partial \psi/\partial a = 0$ and $\partial \psi/\partial b = 0$ Or by eliminating $p,q$ using $\partial f/\partial p = 0$ and $\partial f/\partial q = 0$ in the Linear PDE.
\end{enumerate}
		
\section{Applications of PDE}
\begin{enumerate}
	\item The particular solution satisfying a relation  $b = \phi(a)$, is obtained by differentiating $z= f(x,y,z,a,\phi(a)) = 0$ wrt $a$ partially($x,y,z$ are indepedent of $a$) and substituting the value of $a$. This particular solution is the \textbf{envelope of the subfamily} $z = \eta(x,y,z,a,b)$ where $b=\phi(a)$.
\end{enumerate}
\section*{Exercise}
\begin{enumerate}
	\item Singular solution of $z = px + qy + p^2 + q^2$ is $z = -(x^2+y^2)/4$ obtained by substituting $p = -x/2$ and $q = -y/2$ since $\partial z/\partial p = x + 2p = 0$ and $\partial z/\partial q = y + 2q = 0$.
	\item $(x-a)^2 + (y-b)^2 + z^2 = 1$ is the complete integral of $z^2(p^2+q^2+1) = 1$. \\
	The envelope\footnote{An envelope is the curve which is tangent to every curve in the family at some point. The points of tangency together form the whole envelope.} of the subfamily $b = 2a$ is  the particular integral $(y-2x)^2+5z^2 = 5$.\\
	And $z = \pm 1$ are singular solutions of $z^2(p^2+q^2+1) = 1$.
\end{enumerate}

\section{Classification of First Order PDE}
Let $Pp + Qq = R$ with initial conditions $x = x_0(t)$, $y = y_0(t)$ and $z = z_0(t)$ has
\begin{enumerate}
	\item \textbf{unique solution} if $$\frac{P(x_0,y_0,z_0)}{dx_0/dt} \ne \frac{Q(x_0,y_0,z_0)}{dy_0/dt}$$
	\item \textbf{infinitely many solution} if $$\frac{P(x_0,y_0,z_0)}{dx_0/dt} = \frac{Q(x_0,y_0,z_0)}{dy_0/dt} = \frac{R(x_0,y_0,z_0)}{dz_0/dt}$$
	\item \textbf{no solution} if $$R \ne 0 \qquad \text{and}\qquad \frac{P(x_0,y_0,z_0)}{dx_0/dt} = \frac{Q(x_0,y_0,z_0)}{dy_0/dt} \ne \frac{R(x_0,y_0,z_0)}{dz_0/dt}$$
\end{enumerate}
\section*{Exercise}
\begin{enumerate}
	\item The Cauchy problem $yp-xq = 0$, $x_0(t) = \cos t$, $y_0(t) = \sin t$ and $z_0(t) = 1$ has infinitely many solution.
\end{enumerate}

\section{Non Linear PDE}
\section*{Exercise}
\begin{enumerate}
	\item Find complete integral of $z = px + qy + p^2 + q^2$ ?\\ $f(x,y,z,p,q) = z - px - qy - p^2 - q^2 = 0$.
	$$\frac{dx}{-x-2p} = \frac{dy}{-y-2q} = \frac{dz}{p(-x-2p) + q(-y-2q)} = \frac{dp}{-(-p+p)} = \frac{dq}{-(-q+q)}$$
	Thus, $p = a$ and $q = b$. $dz = adx + bdy$.$z = ax + by + c$.
\item Find complete integral of $q = 3p^2$ ?\\ $f(x,y,z,p,q) = q-3p^2 = 0$.
	$$\frac{dx}{-6p} = \frac{dy}{1} = \frac{dz}{p(-6p)+q} = \frac{dp}{-(0+q0)} = \frac{dq}{-(0+q0)}$$
	Thus, $p = a$ and $q = b$. Then $dz = adx + bdy$. Therefore, $z = ax + by + c$.
\item Find complete integral of $zpq = p+q$ ?\\ $f(x,y,z,p,q) = zpq - p - q = 0$.
	$$\frac{dx}{zq - 1} = \frac{dy}{zp - 1} = \frac{dz}{p(zq-1)+q(zp-1)} = \frac{dp}{-(0+p(pq))} = \frac{dq}{-(0+q(pq))}$$
		$p = cq$ and $zpq = p + q$. $q = \frac{c + 1}{cz}$ and $p = \frac{c+1}{z}$. $zdz = \frac{c+1}{c} (cdx + dy)$.\\ $cz^2 = 2(c+1)(cx + y + d)$.
	\item Find the complete integral of $p^2 - y^2q = y^2 - x^2$ ?\\ $f(x,y,z,p,q) = p^2 - y^2q - y^2 + x^2 = 0$
	$$\frac{dx}{2p} = \frac{dy}{-y^2} = \frac{dz}{p(2p)+q(-y^2)} = \frac{dp}{-(2x+p(0))} = \frac{dq}{-(-2y+q(0))}$$
	$p = \sqrt{c^2-x^2}, q = c^2y^{-2}-1$. $dz = \sqrt{c^2-x^2} dx + (c^2y^{-2} -1)dy$.
		$$z = \frac{x}{2} \sqrt{c^2 - x^2} + \frac{c^2}{2} \sin^{-1} \frac{x}{c} - \frac{c^2}{y} - y + d$$
	\item Find complete integral of $z^2(p^2z^2+q^2) = 1$ ?\\
	$f(x,y,z,p,q) = z^2(p^2z^2+q^2)-1 = 0$
		$$\frac{dx}{2pz^4} = \frac{dy}{2qz^2} = \frac{dz}{4p^2z^3+2zq^2} = \frac{dp}{-p(4p^2z^3+2zq^2)} = \frac{dq}{-q(4p^2z^3+2zq^2)}$$
		$p = qc$. $q^2 = \frac{1}{z^2(c^2z^2+1)}$. $z\sqrt{c^2z^2+1}dz = c dx + dy$
		$$(c^2z^2+1)^3  = 9c^4(cx + y + d)^2$$
	\item Find complete integral of $px + qy = pq$ ?\\
	$f(x,y,z,p,q) = px + qy - pq = 0$.
		$$\frac{dx}{x-q} = \frac{dy}{y-p} = \frac{dz}{p(x-q)+q(y-p)} = \frac{dp}{-(p+p(0))} = \frac{dq}{-(q+q(0))}$$
		$p = qc$, $q = (cx+y)/c$. $p = cx+y$. $cdz = (cx+y)(cdx + dy) = udu$
		$$2cz = (cx+y)^2 + d$$
\end{enumerate}

\section{Special Forms of Charpit's Equation}
\begin{enumerate}
	\item $f(p,q) = 0$ \\
	Substitute $p = a$ and $q = b$, obtain $b = \phi(a)$ from $f(a,b) = 0$. Then complete integral is $z = ax + \phi(a)y + c$.
	\item Clairut's Equation : $z = px + qy + f(p,q)$. Then complete integral is $z = ax + by + f(a,b)$.
	\item $f(x,p) = g(y,q)$\\
	Write $f(x,p) = g(y,q) = a$. Obtain $p = \phi(x,a)$ and $q = \psi(y,a)$. Then integrate $z = pdx + qdy$ to find the complete integral.
	\item $f(p,q,z) = 0$ \\
	Substitute $u = x+ay$ and $p = dz/du$ and $q = adz/du$ in $f(p,q,z) = 0$ and solve.
\end{enumerate}

\section*{Exercise}
\begin{enumerate}
	\item Find complete integral of $pq = 1$ ?
	$z = ax + y/a + c$.
	\item Find complete integral of $p^2 + q^2 = n^2$ ?\\
	$a^2 + b^2 = n^2$  $b = \pm \sqrt{n^2-a^2}$
	$z = ax \pm \sqrt{n^2-a^2}y + c$.
	\item Find complete integral of $q = e^{-p/a}$ ?
	$z = \alpha x + e^{-\alpha/a} y + c$.
	\item Find complete integral of $p^2 - q^2 = \lambda$ ?
	$z = ax \pm \sqrt{a^2-\lambda}y + c$.
	\item Find complete integral of $pq = k$ ?
	$z = ax + ky/a + c$.
	\item Find complete integral of $\sqrt{p} + \sqrt{q} = 1$ ?
	$z = ax + (1-\sqrt{a})^2y + c$.
	\item Find complete integral of $p^2 + q^2 = npq$ ?
	$z = ax + a(n \pm \sqrt{n^2-4})y/2 + c$.
	\item Find complete integral of $q = 3p^2$ ?
	$z = ax + 3a^2y + c$.
	\item Find complete integral of $p^3q^3 = 1$ ?
	$z = ax + \omega y/a + c$.\\
	\hrule
	\item Find complete integral of $z = px + qy -2\sqrt{pq}$ ? $z = ax + by - 2\sqrt{ab}$.
	\item Find complete integral of $z = px + qy + \log(pq)$ ? $z = ax + by + \log(ab)$.
	\item Find complete integral of $z = px + qy + p/q$ ? $z = ax + by + a/b$.
	\item Find complete integral of $z = px + qy + p^2 + pq + q^2$ ? $z = ax + by + a^2 + ab + b^2$.
	\item Find complete integral of $z = px + qy + p^2 + q^2$ ? $z = ax + by + a^2 + b^2$.\\
	\hrule
	\item Find complete integral of $\sqrt{p} + \sqrt{q} = 2x$ ?
	$\sqrt{p}-2x = -\sqrt{q} = a$.
	$p = (a+2x)^2$, $q = a^2$.
	integrate $dz = (a+2x)^2 dx + a^2 dy$.
	$z = (a+2x)^3/6 + a^2y + b$.
	\item Find complete integral of $yp = 2xy+\log q$ ?
	$p = 2x +\frac{1}{y}\log q$. $p-2x = \frac{1}{y}\log q = a$.
	integrate $dz = (a+2x)dx + e^{ay}dy$.
	$z = (a+2x)^2/4 + \frac{e^{ay}}{a} + b$.
	\item Find complete integral of $q = 2yp^2$ ?
	$q/2y = p^2 = a$.
	integrate $dz = \pm \sqrt{a} dx + 2ay dy$.
	$z = \pm \sqrt{a}x + ay^2 + b$.
	\item Find complete integral of $x^2p^2 = yq^2$ ?
	$x^2p^2 = yq^2 = a^2$.
	integrate $dz = a \frac{dx}{x} + a \frac{dy}{\sqrt{y}}$.
	$z = a\log x + 2a\sqrt{y} + b$.\\
	\hrule
	\item Find complete integral of $9(p^2z + q^2) = 4$ ?
	$u = x+ay$, $p = dz/du$, $q = a dz/du$.
	$(\frac{dz}{du})^2 z + (a \frac{dz}{du})^2 = 4/9$.
	$(\frac{dz}{du})^2 (z + a^2) = 4/9$.
	$(\frac{dz}{du}) = \pm 2/3\sqrt{z+a^2}$.
	$3\sqrt{z+a^2}dz = \pm 2du$.
	$(z+a^2)^\frac{3}{2} = \pm u + c$.
	$(z+a^2)^3 = \pm (x+ay+c)^2$.
	\item Find complete integral of $p^2 = qz$ ?
	$u = x+ay$, $p = dz/du$, $q = a dz/du$.
	$(\frac{dz}{du})^2 = az \frac{dz}{du}$.
	$\frac{dz}{du} = az$.
	$\frac{dz}{z} = adu$.
	$\log z = au + \log c$.
	$z = ce^{x+ay}$.
	\item Find complete integral of $p(1+q^2) = q(z-\alpha)$ ?
	$u = x+ay$, $p = dz/du$, $q = a dz/du$.
	$\frac{dz}{du} (1 + (a\frac{dz}{du})^2) = a \frac{dz}{du} (z - \alpha)$.
	$\frac{dz}{du} = \pm \sqrt{a(z-\alpha)-1}/a$.
	$\pm \frac{adz}{\sqrt{a(z-\alpha)-1}} = du$.
	$2\sqrt{a(z-\alpha)-1} = u + c$.
	$4(a(z-\alpha)-1)  = (x+ay+c)^2$.
\end{enumerate}


\section{Group Theory}
\begin{definition}
	An \textbf{algebra} is $\entity{\mathcal{S},\mathcal{F}}$ where $\mathcal{S}$ is a collection of sets and $\mathcal{F}$ is a collection of functions/relations defined on them.
\end{definition}

\begin{definition}
	A \textbf{binary relation} on a set $A$ is a relation between $A \times A$ and $A$.
\end{definition}

\begin{definition}
	An \textbf{associative} binary relation $\ast$ on $A$ satisfies
	\begin{equation}
		(x \ast y),(y \ast z) \in A \implies (x \ast y) \ast z, x \ast (y \ast z) \in A,\ (x \ast y) \ast z = x \ast (y \ast z)
	\end{equation}
\end{definition}

\begin{definition}
	A \textbf{commutative} binary relation $\ast$ on $A$ satisfies
	\begin{equation}
		x \ast y \in A \implies y \ast x \in A,\ x \ast y = y \ast x
	\end{equation}
\end{definition}
	A commutative algebra is also called abelian.

\begin{definition}
	A binary \textbf{operation} on $A$ is a function $\ast : A \times A \to A$.
\end{definition}

\begin{definition}
	An \textbf{associative} binary operation $\ast$ on $A$ satisfies
	\begin{equation}
		(x \ast y) \ast z = x \ast (y \ast z)
	\end{equation}
\end{definition}

\begin{definition}
	A \textbf{commutative} binary operation $\ast$ on $A$ satisfies
	\begin{equation}
		x \ast y = y \ast x
	\end{equation}
\end{definition}

\begin{definition}
	A binary \textbf{algebra} $\entity{A,\ast}$ is an algebra with a set $A$ together with a binary operation $\ast$ on $A$.
\end{definition}

\begin{definition}
	A \textbf{magma} is a binary algebra $\entity{A,\ast}$ where $\ast$ is a binary operation on $A$.
	By the definition of binary operation, $\ast$ is well-defined(closed) on $A \times A$.
\end{definition}

\begin{definition}
	A \textbf{semigroup} is a magma $\entity{A,\ast}$ where $\ast$ is associative.
\end{definition}

\begin{definition}
	A \textbf{left identity} $e'$ of an algebra $\entity{A,\ast}$ satisfies $e' \ast x = x,\ \forall x \in A$.
	And \textbf{right identity} $e'$ satisfies $x \ast e' = x,\ \forall x \in A$.
	An \textbf{identity} element $e$ of $\entity{A,\ast}$ satisfies both.
\end{definition}
	A binary algebra has at most one identity element.
	Homomorphisms map identity elements into identity elements.

\begin{definition}
	A \textbf{monoid} is a semigroup $\entity{A,\ast}$ where $\ast$ has an identity $e \in A$.
\end{definition}

\begin{definition}
	Let $x \in A$.
	An \textbf{inverse} $x^{-1}$ of $x$ in an algebra $\entity{A,\ast}$ satisfies $xx^{-1} = x^{-1}x$.
	Let $e$ be the identity of a monoid $\entity{A,\ast}$.
	Then, $x^{-1}$ satisfies $xx^{-1} = x^{-1}x = e$.
\end{definition}

\begin{definition}
	A \textbf{group} is a monoid $\entity{A,\ast}$ where every element $x \in A$ has an inverse $x^{-1}$.
\end{definition}

\begin{definition}
	An algebra $\entity{R,+,\times}$ is a \textbf{ring} if
	\begin{enumerate}
		\item $\entity{R,+}$ is an abelian group.
		\item $\entity{R,\times}$ is a semigroup.
		\item $\times$ is distributive over $+$.
	\end{enumerate}
\end{definition}

\begin{definition}
	A commutative ring with unity $\entity{D,+,\times}$ is an \textbf{integral domain} if
	\begin{enumerate}
		\item $\entity{D^\ast,\times}$ has no zero divisors.
		\item $\times$ is distributive over $+$.
	\end{enumerate}
\end{definition}

\begin{definition}
	An integral domain $\entity{F,+,\times}$ is a \textbf{field} if
	\begin{enumerate}
		\item $\entity{F^\ast,\times}$ is an abelian group.
		\item $\times$ is distributive over $+$.
	\end{enumerate}
\end{definition}

\begin{definition}
	An algebra $\entity{V,F,+,\times}$ is a linear algebra if
	\begin{enumerate}
		\item $\entity{F}$ is a field.
		\item $\entity{V,+}$ is an abelian group.
		\item $\entity{V,\times}$ is a semigroup.
		\item $\times$ is distributive over $+$.
	\end{enumerate}
\end{definition}

\begin{figure}[h]
	\centering
\begin{tikzpicture}
	\node (a) at (0,0) {$Magma$};
	\node (b)[below=of a]{$Semigroup$};
	\node (c)[below=of b]{$Monoid$};
	\node (d)[below=of c]{$Group$};
	\node (e)[below=of d]{$Abelian\ Group$};
	\node (f)[left=of e]{$Cyclic\ Group$};

	\node (h)[right=of b]{$Semigroupoid$};
	\node (i)[below=of h]{$Small\ Category$};
	\node (j)[below=of i]{$Groupoid$};
	\node (g)[above=of h]{$Partial\ Algebra$};

	\draw (g) -- (h) -- (i) -- (j);
	\draw (g) -- (a) node[midway,above]{$+totality$}(a) -- (b) node[midway,right]{$+associative$} -- (c) node[midway,right]{$+identity$} -- (d) node[midway,right]{$+inverses$} -- (e) node[midway,right]{$+commutative$} -- (f);
	\draw (d) -- (f) node[midway,left]{$+generator$};
\end{tikzpicture}
	\caption{Binary Algebraic Structures}
\end{figure}

\begin{definition}
	The \textbf{sum} of two subsets $A$ and $B$ of a magma\footnote{Instead of magma, the name groupoid is used in many texts that don't study groupoid in detail} $\entity{X,+}$ is 
	$$A+B = \{ a+b : a \in A, b \in B \}$$
\end{definition}

\begin{definition}
	Let $\entity{R,+,\cdot},\entity{R',+',\cdot'}$ be two commutative rings with identity.
	A function $f : R \to R'$ is \textbf{linear} if $f(k \cdot x+y) = k \cdot' f(x) +' f(y)$.
\end{definition}

\begin{definition}
	A function $f : R^n \to R'$ is \textbf{$n$-linear} if for $1 \le k \le n$,
	\begin{equation*}
		f(a_1,a_2,\dots,ka_i+a'_i,\dots,a_n) = kf(a_1,a_2,\dots,a_i,\dots,a_n) + f(a_1,a_2,\dots,a'_i,\dots,a_n)
	\end{equation*}
\end{definition}

\begin{definition}
	Let $\entity{G,\ast_1,\ast_2,\dots,\ast_r}$ and $\entity{H,\star_1,\star_2,\dots,\star_r}$ be two algebraic structures.
	A function $f : G \to H$ is a \textbf{homomorphism} if $\forall \ast_k,\ f(x \ast_k y) = f(x) \star_k f(y)$.
\end{definition}

\begin{definition}
	An \textbf{isomorphism} is a bijective, homomorphism.
\end{definition}

\begin{enumerate}
	\item Number of relations on $A = 2^{n^2}$.
	\item Number of reflexive relations on $A = 2^{n^2-n}$.
	\item Number of symmetric relatons on $A = 2^{\frac{n(n+1)}{2}}$.
	\item \textcolor{red}{Number of equivalence relations on $A = B(n)$, $n^{th}$ Bell number\footnote{$B(n)= \sum S(n,k)$ where $S(n,k)$ are Stirling numbers of second kind.}}
	\item Number of total relations on $A = 2^n \textcolor{red}{3^{\frac{n(n-1)}{2}}}$.
	\begin{figure}[hbt]
	\centering
		\[ \begin{bmatrix} 0 & 1 & 2 & 3 \\ 4 & 0 & 5 & 6 \\ 7 & 8 & 0 & 9 \\ 10 & 11 & 12 & 0 \end{bmatrix} \qquad \begin{bmatrix} 1 & 2 & 4 & 7 \\ \bar{2} & 3 & 5 & 8 \\ \bar{4} & \bar{5} & 6 & 9 \\ \bar{7} & \bar{8} & \bar{9} & 10 \end{bmatrix} \qquad \begin{bmatrix} 1 & \textcolor{red}{1} & \textcolor{red}{2} & \textcolor{red}{3} \\ \textcolor{red}{\bar{1}} & 2 & \textcolor{red}{4} & \textcolor{red}{5} \\ \textcolor{red}{\bar{2}} & \textcolor{red}{\bar{4}} & 3 & \textcolor{red}{6} \\ \textcolor{red}{\bar{3}} & \textcolor{red}{\bar{4}} & \textcolor{red}{\bar{6}} & 4 \end{bmatrix} \]
	\caption{Enumerating Relations - Reflexive, Symmetric, and Total}
	\end{figure}
	\item Let $|A|=m$, $|B|=n$. Number of functions $f : A \to B $ = $n^m$.
	\item Number of injections $f : A \to B = \perms{n}{m} \qquad (n \geq m)$.
	\item \textcolor{red}{Number of surjections $\displaystyle f : A \to B = \sum_{r=0}^{n-1} (-1)^r \binom{n}{r} (n-r)^m \qquad (n \leq m)$}
	\item Number of bijections $f : A \to B = n! \qquad (n = m)$
	\begin{figure}[hbt]
	\centering
	\begin{tabular}{ccccc}
		1 & & & & \\ %1 from right of previous row
		1 & 2 & & & \\  % 2 = 1+1
		2 & 3 & 5 & & \\%3 = 2+1, 5 = 3+2
		5 & 7 & 10 & 15 & \\ %7 = 5+2, 10=7+3, 15= 10+5
		15 & 20 & 27 & 37 & 52\\
	\end{tabular}
	\caption{Bell Triangle}
	\end{figure}
	\item Number of binary operations on $A = n^{n^2}$ where $|A|=n$.
\end{enumerate}

%Fraleigh Part I, Chapters 1-7
\section{Groups and Subgroups}
\begin{definition}
	A \textbf{group} is a binary algebraic structure $\entity{G,\ast}$ which satisfies
	\begin{enumerate}
		\item $\ast$ is closed, $\forall x,y \in G,\ x \ast y \in G$
		\item $\ast$ is associative, $\forall x,y,z \in G,\ (x \ast y) \ast z = x \ast (y \ast z)$.
		\item $\ast$ has an identity element, $\exists e \in G,\ \forall x \in G,\ e \ast x = x = x \ast e$.
		\item $\ast$ has inverses for every element of $G$, $\forall x \in G,\ \exists x^{-1} \in G,\ x \ast x^{-1} = e = x^{-1} \ast x$
	\end{enumerate}
\end{definition}

\begin{definition}
	The \textbf{order} of a group is the number of elements in it.
	The \textbf{order} of an element $g \in G$ is the order of the smallest subgroup of $G$ containing $g$.
\end{definition}

\begin{definition}
	An element $g \in G$ is a \textbf{generator} if the smallest subgroup of $G$ containing $g$ is $G$ itself.
	A group $G$ is \textbf{cyclic} if it has a generator.
\end{definition}

\begin{definition}
	Let $H \le G$. $H$ is \textbf{normal} in $G$ if $gH = Hg$ for every element $g \in G$.
\end{definition}

\begin{definition}
	The \textbf{center} of a group, $Z(G)$ is the set of all elements that commutes with every element in $G$.
\end{definition}
\begin{definition}
	The \textbf{centralizer} of an element $g$, $C(g)$ is the set of all elements that commute with $g$.
\end{definition}
\begin{definition}
	Let $H \le G$. $H$ is \textbf{normal} in $G$ if $gH = Hg$ for every element $g \in G$.
\end{definition}
\section{Properties of Center}
\begin{enumerate}
	\item The center $Z(G)$ of a group $G$ is a normal subgroup of $G$. The centralizer of $g$, $C(g)$ is a subgroup of $G$.
	\item $Z(G) \le C(g) \le C(g^k)$.
	\item \textcolor{red}{$C(g) = C(g^k) \iff \gcd(k,n) = 1$ where $o(g)=n$.}
	\item $Z(S_n)$ is trivial for $n \ge 3$.
	\item $Z(D_n)$ is trivial when $n$ is odd.
	\item $Z(A_n)$ is trivial for $n \ge 4$.
	\item $Z(M_n(F)) = \{ aI : a \in F \}$.
	\item $Z(GL(n,F)) = \{ aI : a \in F,\ a \ne 0 \}$.
	\item $Z(SL(n,F)) = \{ aI : a \in F,\ a^n = 1 \}$.
	\item $Z(Q_8) = \{ 1,-1 \} \cong \mathbb{Z}_2$.
	\item Center of a direct product is the direct product of centers.
	\item Center of a simple group is either trivial(nonabelian) or the whole group(abelian).
	\item Gr\"un`s Lemma : If $G$ is perfect, then $Z(G/Z(G))$ is trivial.
\end{enumerate}

\section{Properties of Groups}
\begin{enumerate}
	\item $o(a)=o(a^{-1})$
	\begin{proof}
		$a^n = e \iff (a^{-1})^n a^n = (a^{-1})^n \iff e = (a^{-1})^n$
	\end{proof}
	\item $o(xax^{-1}) = o(a) = o(x^{-1}ax)$
	\begin{proof}
		$(xax^{-1})^n = e \iff  xa^nx^{-1} = e \iff a^n = x^{-1}x \iff a^n = e$
	\end{proof}
	\item $o(ab) = o(ba)$
	\begin{proof}
		$(ab)^n = e \iff b(ab)^nb^{-1} = e \iff (ba)^n = e$
	\end{proof}
	\item $\forall a \in G,\ a^{-1} = a \implies G$ is abelian.
	\begin{proof}
		$ab = a^{-1}b^{-1} = (ba)^{-1} = ba$
	\end{proof}
	\item $\forall a,b \in G,\ (ab)^2 = a^2b^2 \iff G$ is abelian.
	\begin{proof}
		$abab = aabb \iff bab = abb \iff ba = ab$
	\end{proof}
	\item $\forall a,b \in G,\ (ab)^{-1} = a^{-1}b^{-1} \iff G$ is abelian
	\begin{proof}
		$(ab)^{-1} = a^{-1}b^{-1} \iff (ab)^{-1} = (ba)^{-1} \iff ab = ba$
	\end{proof}
\item \textcolor{red}{If $\forall a,b \in G,\ a^3b^3 = (ab)^3$, then every commutator is of order $3$.}
	\begin{proof}
	$a^3b^3 = (ab)^3 \implies a^2b^2 = (ba)^2$.
	$$(aba^{-1}b^{-1})^2 = (a^{-1}b^{-1})^2(ab)^2 = b^{-2}(a^{-2}b^2)a^2 = b^{-2}(ba^{-1})^2a^2 = b^{-1}a^{-1}ba$$
		$(aba^{-1}b^{-1})^4 = (b^{-1}a^{-1}ba)^2 = aba^{-1}b^{-1} \implies (aba^{-1}b^{-1})^3 = e$
	\end{proof}
	\item \textcolor{red}{$a^n = 1,\ aba^{-1} = b^2 \implies b^{2^n-1} = e$.}
	\begin{proof}
		$(aba^{-1})^2 = ab^2a^{-1} = b^4 \implies a^2ba^{-2} = b^4 \implies a^nba^{-n}=b^{2^n}$.
	\end{proof}
	\item Let $a,b$ be elements of fintie order, then $ab$ is not necessarily of finite order.
	\item If $x$ commutes with $y$, then
		\subitem $x$ commutes with $y^{-1}$, since $y^{-1}(xy)y^{-1} = y^{-1}(yx)y^{-1}$
		\subitem $x^{-1}$ commutes with $y$, since $x^{-1}(xy)x^{-1} = x^{-1}(yx)x^{-1}$.
		\subitem $x^{-1}$ commutes with $y^{-1}$, since $(xy)^{-1} = (yx)^{-1}$.
	\item Group $G$ has precisely one element $g$ of order two, then $g$ commutes with every element of $G$.
	\begin{proof}
		Let $g \in G$ such that $o(g) = 2$.\\
		$\forall x \in G,\ o(xgx^{-1}) = o(g) = 2 \implies xgx^{-1} = g \implies xg = gx$
	\end{proof}
\end{enumerate}

\section{Subgroups}
\begin{enumerate}
	\item Subgroup Test : $a^{-1}b \in H,\ \forall a,b \in H \implies H \le G$.
	\item Finite Subgroup Test : $H$ is a subgroup of a finite group if $\ast$ is closed in $H$.
	\item Group $G$ has a element of order $n$ iff $G$ has a \texttt{cyclic} subgroup of order $n$. 
	\item Let $G$ be an \texttt{abelian} group. The set $\{ g \in G : g^p = e \}$ is a subgroup of $G$.
	However, it is not true for nonabelian groups.
	$\{ g \in D_4 : g^2 = e \}$ is not a subgroup of $D_4$.
	\item Let $G$ be an \texttt{abelian} group of order $n$. If $d|n$, then $G$ has a subgroup of order $d$. If $d$ is square-free, then $G$ has an element of order $d$.
	\item Every cyclic group of order $n$ has $\phi(n)$ elements of order $n$.
		Suppose $G$ has $n_m$ elements of order $m$, then $G$ has $n_m/\phi(m)$ cyclic subgroups of order $m$.
		\subitem If a finite abelian group $G$ has $24$ elements of order $6$, then $G$ has $24/\phi(6) = 12$ subgroups of order $6$ as abelian group of order $6$ are cyclic.
	\item The dihedral group $D_n$ has $\phi(d)$ elements of order $d$ for every divisor $d$ of $n$, except $d=2$. There are either $n$ or $n+1$ elements of order $2$ depending on the parity of $n$.
	The number of subgroup of $D_n = \tau(n)+\sigma(n)$.
	\item $H,K \le G \implies H \cap K \le G$. And $H \cup K \subset HK \le G$.
		\subitem $|HK| = |H| |K| / |H \cap K|$.
		\subitem $m\mathbb{Z} \cap n\mathbb{Z} = k\mathbb{Z}$ where $k = lcm(m,n)$.
		\subitem $m\mathbb{Z} + n\mathbb{Z} = k\mathbb{Z}$ where $k = \gcd(m,n)$.
\end{enumerate}

\section{Normal Subgroups}
\begin{enumerate}
	\item Normal subgroup Test : $\forall g \in G,\ \forall h \in H,\ ghg^-1 \in H \implies H \trianglelefteq G$.
	\item $$\Gamma p = \left\{ \begin{pmatrix} a & b \\ c & d \end{pmatrix} \in SL_2(\mathbb{Z}) : a \cong d \cong 1 \pmod{p}, b \cong c \cong 0 \pmod{p} \right\} \trianglelefteq SL_2(\mathbb{Z})$$
\end{enumerate}

\section{Strange Groups}
\begin{enumerate}
	\item Smallest non-abelian group is $S_3$.
	Smallest non-cyclic group is the Klein $4$-group, $V \cong \mathbb{Z}_2 \times \mathbb{Z}_2$.
	Smallest non-abelian simple group is $A_5$. Thus, $A_5$ is the smallest perfect group.
	\item \textcolor{red}{$D_p, D_4, Q_8, A_4,\dots$ are non-abelian groups with every proper subgroup abelian}. 
	\item $\mathbb{C}^\ast$ is a multiplicative group with identity $1$.
	Unit circle is a subgroup of $\mathbb{C}^\ast$.
	Unit circle has a unique cyclic subgroup for any order.
	The $n$th roots of unity is the cyclic subgroup of unit circle with order $n$.
	\item $\mathbb{Q}/\mathbb{Z}$ is torsion group which has a unique cyclic subgroup of any finite order.
	And every proper subgroup of $\mathbb{Q}/\mathbb{Z}$ is finite and cyclic.
	\item $ \entity{\left\{ \begin{bmatrix} a & a \\ a & a \end{bmatrix} : a \ne 0 \right\},\times}$ is a group with identity $\begin{bmatrix} \frac{1}{2} & \frac{1}{2} \\ \frac{1}{2} & \frac{1}{2} \end{bmatrix}$.
	\item $ \entity{\left\{ \begin{bmatrix} a & 0 \\ 0 & 0 \end{bmatrix} : a \ne 0 \right\},\times}$ is a group with identity $\begin{bmatrix} 1 & 0 \\ 0 & 0 \end{bmatrix}$.
		\item $\entity{\mathbb{Q}^+,\ast}$ where $a \ast b = \frac{ab}{5}$ is a group with idenity $5$.
	\item $\entity{\{5,15,20,25,30,35\},\times_{40}}$ is a group with identity $25$.
	\item \textcolor{red}{The multiplicative group $\mathbb{Z}_n^\times = \{ m \in \mathbb{Z}_n : \gcd(m,n)=1 \}$.\\
		If it is cyclic, then it has $\phi(\phi(n))$ generators.}
	\item Convergent sequences under addition is a group.
	\item Group of rigid motions(rotations) of the cube is a group of order $\binom{8}{1}\binom{3}{1}=24$ under permutation multiplication.
	This group is isomorphic to $S_4$.
\end{enumerate}

\section{Group Representations}
\begin{enumerate}
	\item The function $\phi : G \to S_G,\ \phi(x)=\lambda_x,\ \lambda_x(g) = xg$ is the \textbf{left regular representation} of $G$.
	\item Let $G$ be a finite group with a generating set $S$. The \textbf{Cayley digraph} of $G$ has elements of $G$ as its vertices and generators from $S$ as its arcs.\\
	The Cayley digraph for an abelian graph is symmetric.
	\item A \textbf{permutation matrix} is obtained by reordering rows of an identity matrix.\\
	The permutation matrices $P_{n \times n}$ under matrix multiplication forms a group which is isomorphic to $S_n$. By Cayley`s theorem, every group $G$ is isomorphic to a group of permutation matrices where left regular representation corresponds to left multiplication.
	\item The \textbf{set theoretic group representation} using generators and their relations.\\
	The dihedral group with generators $y=R_{2\pi/n}$, rotation by $2\pi/n$ radians and $x=\mu$, reflection (about the line through the center and a fixed vertex) of a regular $n$-gon.
	$$D_n = \{ x^iy^j : x^2=y^n=1, (xy)^2=1 \}$$
	The symmetric group with generators $x=(1,2)$ and $y=(1,2,\dots,n)$. 
	$$S_n = \{ x^iy^j : x^2=y^n=1,(yx)^{n-1}=1 \}$$
	The alternating group with the set of all three cycles of the form $x_j = (1,2,j)$ as generating set $S$.
	$$A_n = \left\{ \prod_{j=3}^n x_j^{n_j} : x_j^3=1,\ (x_ix_j)^2=1\right\}$$
\end{enumerate}

\section{Counter Examples}
\begin{enumerate}
	\item $\entity{\mathbb{R}^\ast,\ast}$ where $a \ast b = a/b$ is not associative.
	\item $\entity{\mathbb{C},\ast}$ where $a \ast b = |ab|$ has no identity element.
	%What is the nature of a inverses in a group over a piecewise connected subset of $\mathbb{R}$ in relation to the fixed points and boundaries ?
	\item $\entity{C[0,1]-\{0\},\times}$ is a not closed. There exists a pair of functions with product $0$.
	\item Let $G$ be a group and $\mathscr{P}(G)$ be the power set of $G$. Define $A \ast B = \{ ab : a \in A,\ b \in B \}$. Then $\entity{\mathscr{P}(G),\ast}$ is a monoid with identity $\{ e \}$. The units are the left cosets of the trivial subgroup.
	\item $\entity{GL(n,F),+}$ is not closed as $I_n+(-I_n) \notin GL(n,F)$.
\end{enumerate}

\section{Group Homomorphisms}
\begin{enumerate}
	\item $\phi : \mathbb{Z} \to \mathbb{Z}$ where $\phi(n)=2n$ with $\ker(\phi) = 0$ and $\phi[\mathbb{Z}] = 2\mathbb{Z}$.
	\item $\phi : \mathbb{Q} \to \mathbb{Q}$ where $\phi(x)=2x$ with $\ker(\phi) = 0$ and $\phi[\mathbb{Q}] = \mathbb{Q}$.
	\item $\phi : \mathbb{R} \to \entity{\mathbb{R}^+,\times}$ where $\phi(x)=0.5^x$ with $\ker(\phi) = 0$ and $\phi[\mathbb{R}] = \mathbb{R}^+$.
	\item $\phi : \mathbb{Z} \to \entity{\mathbb{Z},\ast}$ where $m \ast n = m+n-1 $ is a group with $\ker(\phi) = 0$ and $\phi[\mathbb{Z}] = \mathbb{Z}$.
		(hint : $\phi(n) = n+1$, $\phi(0) = 1,\ x^{-1} = -x-2$)
	\item $\phi : \mathbb{Q} \to \entity{\mathbb{Q},\ast}$ where $x \ast y =  x+y+1$ is a group with $\ker(\phi) = 0$ and $\phi[\mathbb{Q}] = \mathbb{Q}$.
		(hint : $\phi(x) = 3x-1,\ \phi(0) = -1,\ x^{-1} = -x-2$)
	\item $\phi : \mathbb{Q}^\ast \to \entity{\mathbb{Q}-\{-1\},\ast}$ where $x \ast y =  \frac{(x+1)(y+1)}{3}-1$ is a group with $\ker(\phi) = 1$ and $\phi[\mathbb{Q}^\ast] = \mathbb{Q}-\{-1\}$.
		(hint : $\phi(x) = 3x-1,\ \phi(1)=2$,\ $x^{-1} = \frac{8-x}{x+1}$)
\end{enumerate}

\section{Cyclic Groups}
\begin{enumerate}
	\item Every cyclic group is abelian.
	\begin{proof}
		$G = \gen{g} \implies \forall a,b \in G,\ ab = g^ng^m = g^mg^n = ba$.
	\end{proof}
	\item Subgroup of cyclic group is cyclic. Let $G$ be a cyclic group of order $n$. The order of the subgroup generated by $g^m$ is $n/\gcd(n,m)$. For each divisor $d$ of $n$, there exists unique cyclic subgroup of order $n/d$.\\

	The multiplicative group $\mathbb{Z}_{25}^\times \cong \mathbb{Z}_{20}$ has generator $3$.
	We have $\gcd(20,5) = \gcd(20,15)$.
	Clearly, $3^5 \cong 18 \pmod{25}$ and $3^{15} \cong 7 \pmod{25}$. Thus, $\gen{7} \cong \gen{18} \cong \mathbb{Z}_4$.
	\item Every proper subgroup of the Klein 4-group, $V \cong \mathbb{Z}_2 \times \mathbb{Z}$ is cyclic. However, $V$ is not cyclic.
	\item For any natural number $n$, there exists a cyclic group of order $n$. Two cyclic group of same order are isomorphic.
	\begin{proof}
		The finite group $\entity{\mathbb{Z}_n,+_n}$ is cyclic with order $n \in \mathbb{N}$ and the infinite group $\mathbb{Z}$ is cyclic.
		Let $G,H$ be cyclic groups of the same order with generators $g,h$ respectively.
		Then $\phi : G \to H,\ g \overset{\phi}{\to} h$ is an isomorphism.
	\end{proof}
	\item An automorphism of a cyclic group is well defined by the image of a generator.
		Clearly, $\mathbb{Z}_{12}$ has $\phi(12)=4$ generators and there are four distinct automorphisms.
	\item For finite cyclic group $\mathbb{Z}_n$, a generator is an element with the same order as the group. However, this is not the case for inifinite cyclic group $\mathbb{Z}$.
		$$o(g) = o(G) \nimplies \gen{g} \cong G$$
	\item Every finite cyclic group, $\mathbb{Z}_n$ has $\phi(n)$ generators which are relatively prime to $n$.
	Clearly, $\mathbb{Z}_{20}$ has a non-prime generator, say $9$.
	\item \textcolor{red}{The equation $x^m = e$ has $m$ solutions in any finite cyclic group $\mathbb{Z}_n$ where $m|n$.}
	\item Let $G$ be an abelian group and $H,K$ are cyclic subgroups of $G$ with generators $h,k$ respectively. Then $\gen{hk}$ is a cyclic subgroup of order $lcm(r,s)$.
	\item $\mathbb{Q}/\mathbb{Z}$ is not cyclic.\\
		proof : $o(\frac{1}{2} + \mathbb{Z}) = 2$, where the infinite cyclic group $\mathbb{Z}$ has no such element.
	\item $\mathbb{Q}^\ast$ is not cyclic.\\
		proof : $o(-1) = 2$, where $\mathbb{Z}$ don't have any element of order two.
	\item $\mathbb{Q},\mathbb{R},\mathbb{C}$ are not cyclic.\\
		proof : If $\mathbb{Q}$ is cyclic, then $\mathbb{Q}/\mathbb{Z}$ is a cyclic quotient group. But $\mathbb{Q}/\mathbb{Z}$ is not.
	\item The subgroup generated by $n$th primite root of unity is a cyclic subgroup of $\mathbb{C}^\ast$ isomorphic to $\mathbb{Z}_n$.
		Clearly, $\gen{(1+i)/\sqrt{2}} \cong \mathbb{Z}_8$.
	\item The subgroup generated by any complex number which is a non-root of unity is a cyclic subgroup of $\mathbb{C}^\ast$ isomorphic to $\mathbb{Z}$.
		Clearly $\gen{1+i} \cong \mathbb{Z}$.
\end{enumerate}

\section{Number Groups}
\begin{enumerate}
	\item $\mathbb{Z},\mathbb{Q},\mathbb{R},\mathbb{C}, n\mathbb{Z}, \mathbb{Z}_n, \mathbb{Q}_c, \mathbb{R}_c, \mathbb{Q}^+, \mathbb{R}^+, \mathbb{Q}^\ast, \mathbb{R}^\ast, \mathbb{C}^\ast, \mathbb{Z}_n^\times$ are groups with a suitable arithmetic operators from $\{ +,\times,+_c,\times_c,+_n,\times_n\}$.
	%$U(n)$ in Fraleigh is $\mathbb{Z}_n^\times$
	%$U_n$ in Fraleigh is primitive $n$th roots of unity 
	%$U$ in Fraleigh is the unit circle in $\mathbb{C}$
	\item Any nontrivial subgroup of $\mathbb{Q}$ is an infinite cyclic group.
	\item $\entity{\mathbb{R}-\{-1\},\ast}$ where $a \ast b = a+b+ab$ is a group with identity $0$ and $o(-2)=2$.
	\item The cyclic group, $\mathbb{Z}_n \cong \mathbb{Z}/n\mathbb{Z} = \{ g^n : n \in \mathbb{N}\}$. $\mathbb{Z}_n$ has $\phi(d)$ elements of order $d$ for every divisor $d$ of $n$.
	$$a^{-1}b \in Z_n \iff \gcd(a,n)|b$$
	\item Group $\mathbb{Z}_n^\times$ is the multiplicative group of natural numbers less than $n$ that are relatively prime to $n$. Thus $|\mathbb{Z}_n^\times| = \phi(n)$.
	Clearly, $\mathbb{Z}_n^\times$ are abelian.
\end{enumerate}

\section{Linear Groups}
\begin{enumerate}
	\item $M_{m \times n}(F)$ is the additive group of all matrices of order $m \times n$ with entries from the field $F$.
	When $m=n$, we may write $M_n(F)$.
	\item General Linear Group, $GL(n,F)$ is the multiplicative group of all invertible matrices of order $n$ with entries from field $F$.
	\item Special Linear Group, $SL(n,F)$ is the multiplicative group of all matrices of order $n$ and determinant $1$ with entries from field $F$.
\end{enumerate}

%Fraleigh Part II, Chapter 8-12
\section{Permutations, Cosets \& Direct Products}
\begin{definition}
	The \textbf{symmetric group} $S_n$ is the set of all permutation on a set $\{1,2,\dots,n\}$ together with the function composition operation.
\end{definition}

	The cycle $f: (1,2,3) \in S_5$ maps $1 \to 2 \to 3 \to 1$ and fixes $4,5$. And cycle $g:(1,2,5) \in S_5$ maps $1 \to 2 \to 5 \to 1$ and fixes $3,4$.
	For example $f(g(1)) = f(2) = 3$, and $f(g(3)) = f(3) = 5$..
	Thus by function composition $f \circ g : (1,2,3)(1,2,5) = (1,3)(2,5)$.
	$$ \begin{pmatrix} 1 & 2 & 3 & 4 & 5 \\ 2 & 3 & 1 & 4 & 5 \end{pmatrix} \begin{pmatrix} 1 & 2 & 3 & 4 & 5 \\ 2 & 5 & 3 & 4 & 1 \end{pmatrix} = \begin{pmatrix} 1 & 2 & 3 & 4 & 5 \\ 3 & 2 & 1 & 4 & 5 \end{pmatrix} \begin{pmatrix} 1 & 2 & 3 & 4 & 5 \\ 1 & 5 & 3 & 4 & 2 \end{pmatrix}$$

\begin{theorem}[Cayley]
	Every group is isomorphic to a subgroup of a symmetric group.
\end{theorem}
\begin{proof}
	The function $\phi : G \to S_G$ defined by $\phi(x) = \lambda_x$ where $g \overset{\lambda_x}{\to} xg$ is an homomorphism.
\end{proof}

\begin{definition}
	Let $\sigma$ be a bijection/permutation on a set $A$.
	The \textbf{orbits} of the permutation $\sigma$ are the equivalent classes of the relation 
	$$a \sim_\sigma b \iff \exists n \in \mathbb{N},\ a = \sigma^n(b)$$
\end{definition}

\begin{definition}
	A permutation $\sigma$ is a \textbf{cycle} if it has at most one orbit containing more than one element.
	The \textbf{length} of a cycle $\sigma$ is the number of elements in its largest orbit. 
\end{definition}
	The multiplication of disjoint cycles is commutative.

\begin{theorem}
	Every permutation of a finite set has a unique cycle decomposition.
\end{theorem}
\begin{proof}
	construct cycles corresponding to each orbit under the permutation
\end{proof}
$$ \begin{pmatrix} 1 & 2 & 3 & 4 & 5 & 6 & 7 \\ 3 & 5 & 2 & 4 & 1 & 7 & 6 \end{pmatrix} = \begin{pmatrix} 1 & 2 & 3 & 5 \\ 3 & 5 & 2 & 1 \end{pmatrix} \begin{pmatrix}4 \\ 4\end{pmatrix} \begin{pmatrix} 6 & 7 \\ 7 & 6 \end{pmatrix}$$
	In short, we may write $(1,3,2,5)(6,7)$ ignoring those which are left fixed by the permutation.
	And $(1,3,2,5)(6,7) = (1,5)(1,2)(1,3)(6,7)$ is an even permutation.	

\begin{definition}
	The alternating group $A_n$ is the subgroup of all even permutations in the symmetric group $S_n$.
\end{definition}

\begin{definition}
	Let $H$ be a subgroup of group $G$.
	The \textbf{left coset}, $gH$ of $H$ containing $g \in G$ is the set of all element of the form $gh$ where $h \in H$.
	The \textbf{right coset} $Hg$ of $H$ containing $g \in G$ is the set of all element of the form $hg$ where $h \in H$.
\end{definition}

\begin{theorem}[Lagrange]
	The order of a subgroup $H$ of a finite group $G$  divides the order of $G$.
\end{theorem}
\begin{proof}
	The left cosets of $H$ in $G$ are disjoint and covers $G$. Thus $|H|$ must divide $|G|$.
\end{proof}

\begin{definition}
	\textbf{Index} of $H$ in $G$, $(G:H)$ is the number of left cosets of $H$ in $G$.
\end{definition}
\begin{theorem}
	The number right cosets of $H$ in $G$ is same as the number of left cosets of $H$ in $G$.
\end{theorem}
\begin{proof}
	$aH = bH \iff ah_1 = bh_2 \iff (ah_1)^{-1} = (bh_2)^{-1} \iff h_1^{-1}a^{-1} = h_2^{-1}b^{-1} \iff Ha^{-1} = Hb^{-1}$.
	Thus, $aH \overset{\phi}{\to} Ha^{-1}$ is bijective.
\end{proof}

\begin{theorem}
	Let $K \le H \le G$.
	Then $(G:K) = (G:H)(H:K)$.
\end{theorem}

\begin{definition}
	Let $G,H$ be two groups.
	The \textbf{direct product} $G \times H$ is defined as the group $\entity{G \times H,\ast}$ where $\ast : (G \times H) \times (G \times H) \to (G \times H)$ such that $(g_1,h_1) \ast (g_2,h_2) = (g_1g_2,h_1h_2)$.
\end{definition}

\begin{theorem}
	$\mathbb{Z}_n \times \mathbb{Z}_m \cong \mathbb{Z}_{n \times m} \iff \gcd(m,n)=1$.
\end{theorem}
\begin{proof}
	$(1,1) \in Z_n \times Z_m$ has order $mn$.
	Thus, $Z_n \times Z_m$ is cyclic.
\end{proof}
	Suppose $\gcd(m,n) = 1$.
	The canonical isomorphism $\phi : \mathbb{Z}_{mn} \to \mathbb{Z}_m \times \mathbb{Z}_n$ is given by
	$$ a\!\!\!\!\! \pmod{mn} \overset{\phi}{\to} \left(a\!\!\!\!\! \pmod{m}, a\!\!\!\!\! \pmod{n}\right) $$

\begin{theorem}
	Let $(a_1,\dots,a_n) \in G_1 \times \dots G_n$ and $o(a_i) = r_i$.\\
	Then $o((a_1,\dots,a_n)) = lcm(r_1,\dots,r_n)$.
\end{theorem}

\begin{theorem}
	Let $G$ be a finitely generated group.
	Then $G \cong \mathbb{Z}_{p_1^{r_1}} \times \mathbb{Z}_{p_2^{r_2}} \times \dots \times \mathbb{Z}_{p_k^{r_k}} \times \mathbb{Z} \times \dots \times \mathbb{Z}$ where the number of $\mathbb{Z}$ is its \text{Betti number}.
\end{theorem}

\begin{theorem}
	Let $G$ be a finite abelian group with order $n$. If $m|n$, then $G$ has a subgroup $H$ of order $m$.
\end{theorem}
\begin{proof}
	We have, $n = \prod P_j^{r_j}$ and $m = \prod P_j^{s_j}$ where $0 \le s_j \le r_j$.
	From the structure of finitely generated abelian group $G$, we may derive the structure of its subgroup $H$ of order $m$ by diminishing the powers of primes as required.
\end{proof}
	%What is the weakest sufficient condition for the converse of Lagrange`s theorem to hold ?


\section{Decomposable Group}
\begin{definition}
	Let $H,K \le G$.
	The equivalent classes of the equivalence relation $aRb \iff a = hbk,\ h \in H,\ k \in K$ are the \textbf{double cosets} of $G$.
\end{definition}

\begin{definition}
	A group $G$ is \textbf{decomposable} if $G \cong H \times K$ where $H,K$ are proper, nontrivial subgroups of $G$.
	Otherwise, $G$ is indecomposable.
\end{definition}
	Finite indecomposable groups are $\mathbb{Z}_p$.

\section{Consequences of Lagrange`s theorem}
\begin{enumerate}
	\item By Lagrange`s theorem, every group of prime order is cyclic.
	\item If $|G|=pq$, then every proper subgroup of $G$ is cyclic.
	\item The quotient group $\mathbb{Z}_n/\gen{g} \cong \mathbb{Z}_\frac{n}{m}$ where $o(g)=m$.
\end{enumerate}

\section{Finite Abelian Groups}
\begin{enumerate}
	\item Finite abelian groups are finitely generated.
	\item Number of abelian groups of order $n = p_1^{r_1}p_2^{r_2}\dots p_k^{r_k}$ is $\prod_k B(r_k)$.
	\item Order of an abelian group $G$ is square free, then $G$ is cyclic.
	\item Order of an element in a cyclic group\\
	Let $m \in \mathbb{Z}_n$.
	Then it has order $$o(m) = \frac{n}{\gcd(n,m)}$$
	\item Order of an element in a product of Cyclic groups\\
	Let $(g_1,g_2,\dots,g_k) \in G_1 \times G_2 \times \dots \times G_k$. Then 
	$$o(g_1,g_2,\dots,g_k) = lcm(o(g_1),o(g_2),\dots,o(g_k))$$
	\item Enumerating the elements of same order in a finite abelian group.
	\subitem Enumerate elements of order $4$ in $\mathbb{Z}_{12} \times \mathbb{Z}_{10}$ ? 

	Let $(g,h) \in \mathbb{Z}_{12} \times \mathbb{Z}_{10}$ has order $o(g,h)= 4 \iff o(g)=4,\ o(h) = 1 \text{ or } 2$. Clearly, an element $k \in \mathbb{Z}_{12}$ is of order $4$ iff $\frac{12}{\gcd(12,k)}=4$. For $\gcd(12,k)=3$, we have $k = 3$ or $9$. For $\gcd(10,k)=5$, we have $k=5$. For $\gcd(10,k)=10$, we have $k=0$. Thus, the elements are $(3,0),(3,5),(9,0)$ and $(9,5)$.  
	In other words, $\phi(4)\phi(2)+\phi(4)\phi(1)=4$ elements of order four in $\mathbb{Z}_{12} \times \mathbb{Z}_{10}$.
	\subitem Enumerate elements of order $9$ in $\mathbb{Z}_{12} \times \mathbb{Z}_{18} \times \mathbb{Z}_{27}$?

	There are $\phi(1),\phi(3),\phi(9)$ elements of order $1,3,9$ respectively(if any\footnote{We know that, $\mathbb{Z}_{12}$ don't have any element of order $9$.}).
	There are $1+2+6$ elements of order either $1,3$ or $9$ in both $\mathbb{Z}_{18}$ and $\mathbb{Z}_{27}$.
	There are $3 \times 9 \times 9$ elements out of which precisely $3 \times 3 \times 3$ of them are of order either $1$ or $3$.
	Thus, there are $216$ elements of order $9$.
	\item Let $g \in \mathbb{Z}_n$ with $o(g) = m$ where $n=p_1^{r_1}p_2^{r_2}\dots p_k^{r_k}$ and $m = p_1^{s_1}p_2^{s_2}\dots p_k^{s_k}$ such that $0 \le s_j \le r_j$.
	Then $g = (g_1,g_2,\dots,g_k) \in \mathbb{Z}_{p_1^{r_1}} \times \mathbb{Z}_{p_2^{r_2}} \times \dots \mathbb{Z}_{p_k^{r_k}}$ with $o(g_j) = s_j$.
		For example, $o(15) = 12$ in $\mathbb{Z}_{36}$. The isomorphism $\phi : \mathbb{Z}_{36} \to \mathbb{Z}_4 \times \mathbb{Z}_9$ where $\phi(a) = (a\pmod{4},a\pmod{9})$.
	Clearly $15 \to (3,6)$. And $o(3) = 4$ and $o(6)=3$.
	\item Let $(g,h) \in \mathbb{Z}_{p^{r_1}} \times \mathbb{Z}_{p^{r_2}}$ with $o(g,h) = p^{r_3}$ where $r_1 \ge r_2$. Then, $(\mathbb{Z}_{p^{r_1}} \times \mathbb{Z}_{p^{r_2}})/\gen{(g,h)} \cong $
		\subitem $\mathbb{Z}_{p^{r_1}} \times \mathbb{Z}_{p^{r_2-r_3}}$ when $o(h) = o(g,h)$.
		\subitem $\mathbb{Z}_{p^{r_1-r_3}} \times \mathbb{Z}_{p^{r_2}}$ when $o(h) < o(g,h)$.

		For example, $(\mathbb{Z}_8 \times \mathbb{Z}_4)/\gen{(2,1)} \cong \mathbb{Z}_8$ and $(\mathbb{Z}_8 \times \mathbb{Z}_4)/\gen{(2,2)} \cong \mathbb{Z}_2 \times \mathbb{Z}_4$.
	\item Order of an element in $S_n$
		Let $\sigma \in S_n$ be a permutation with structure $1^{n_1}2^{n_2}\dots r^{n_r}$.
		Then $o(\sigma) = lcm(\{k : n_k \ge 1 \})$.
		Order of an element in $A_n$ can be found using the same rule as above. Parity of permutation is the parity of $\sum (j-1)n_j$.\\
		Maximum order of an element in $A_{10}$ is $3 \times 7 = 21$. And maximum order of an element in $S_{10}$ is $2 \times 3 \times 5 = 30$ where $2^13^15^1$ is an odd permutation, $\because (1+2+4)$.
		\subitem $A_7$ has a element of order $6$ with structure $2^2 3^1$, since $2+2 = 4$ is even parity.
	\item Maximal abelian subgroup of $S_n$\\
		$S_{10}$ has maximal abelian subgroup of order $36$ which is isomorphic to $\mathbb{Z}_6 \times \mathbb{Z}_6$ and is generated by $\{ (1,2),(3,4,5),(6,7),(8,9,10)\}$. It is abelian as the cycles are disjoint.
	\item \textcolor{red}{Direct product form of the multiplicative group of units, $\mathbb{Z}_n^\times$}\\
	$\mathbb{Z}_{10}^\times=\{1,3,7,9\}$ and $\phi(10)=\phi(2)\phi(5)=4$.
		And $\mathbb{Z}_{10}^\times \cong \mathbb{Z}_4$ as $\gen{3} = \mathbb{Z}_{10}^\times$.
		$$\mathbb{Z}_{mn}^\times \cong \mathbb{Z}_m^\times \times \mathbb{Z}_n^\times \iff \gcd(m,n) = 1$$
		$$\forall n \in \mathbb{N},\ \mathbb{Z}_{2^{n+2}}^\times \cong \mathbb{Z}_2 \times \mathbb{Z}_{2^n}$$
		$$\forall p > 2,\ \forall n \in \mathbb{N},\ \mathbb{Z}_{p^n}^\times \cong \mathbb{Z}_{p^n-p^{n-1}}$$

		Thus, $\mathbb{Z}_4^\times = \mathbb{Z}_2,\ \mathbb{Z}_8^\times = \mathbb{Z}_2 \times \mathbb{Z}_2,\ \mathbb{Z}_{16}^\times \cong \mathbb{Z}_2 \times \mathbb{Z}_4,\ \dots$
		Clearly, $\phi(40) = \phi(8)\phi(5)$ and $\mathbb{Z}_{40}^\times \cong (\mathbb{Z}_2 \times \mathbb{Z}_2) \times \mathbb{Z}_4$.
		And $\mathbb{Z}_{1000}^\times \cong \mathbb{Z}_2 \times \mathbb{Z}_2 \times \mathbb{Z}_{100}$.
\end{enumerate}

\section{Structure of a Permutation}
\begin{definition}
	The \textbf{structure} of a permutation $\sigma \in S_n$ is $1^{n_1} 2^{n_2} \dots r^{n_r}$ where $n_j$ is the number of cycles of length $j$.
\end{definition}
	The number of permutations of the structure $1^{n_1} 2^{n_2} \dots r^{n_r}$ in $S_n$ is
	$$ \frac{n!}{\prod_{k=1}^r n_k!\ k^{n_k}}$$

	There are $\frac{10!}{3!\ 2!\ 1!\ 2^2\ 3}$ elements of the structure $1^3 2^2 3^1$.

	%To enumerate elements of order $12$ in $S_7 \times D_4 \times A_5 \times Z_{30}$.

\section{Finitely Generated Permutation Group}
How to find the order of the group generated by finite number of permutations from a symmetric group ?
\begin{definition}
	The set of all elements of an abelian group $G$ of finite order forms a normal subgroup called \textbf{torsion} subgroup of $G$.
\end{definition}

\begin{definition}
	A \textbf{torsion free} group has only one element of finite order in it.
\end{definition}

\section{Torsion and Torsion Free Groups}
\begin{enumerate}
	\item The torsion subgroup of $\mathbb{C}^\ast$ is the set of all roots of unity.
	The cyclic group generated by $z$ where $|z| \ne 1$ is a torsion free subgroup of $\mathbb{C}^\ast$.
	The cyclic group generated by $e^{2 \pi i x},\ x \in \mathbb{R}-\mathbb{Q}$ is a torsion free subgroup of the unit circle.
	\item Any finite group is a torsion group. The subgroups and quotient groups of any torsion group is also a torsion group. 
	\item Every infinite group has a nontrivial torsion free subgroup. The subgroups of a torsion free group is always torsion free.
	\item Let $T$ be the torsion subgroup of an abelian group $G$. Then the quotient group $G/T$ is torsion free.\\

	The group $\mathbb{Q}^\ast$ has only two elements of finite order, say $1$ and $-1$.
	The torsion subgroup of $\mathbb{Q}^\ast \cong \mathbb{Z}_2$.
	Thus $\mathbb{Q}^+ \cong \mathbb{Q}^\ast / \{ 1,-1 \}$ is torsion free.
	Similarly, $\mathbb{R}^+$ is torsion free.
	\item Suppose normal subgroup $H$ contains the torsion subgroup of a group $G$. Then $G/H$ is torsion free.
	Thus $\mathbb{C}^\ast/U \cong \mathbb{R}^+$ is torsion free.
	\item There is no bound for the order of elements in this torsion group.
		\subitem $\mathbb{Q}/\mathbb{Z} \cong \mathbb{Q}_1$ is a torsion group and $o(p/q+\mathbb{Z}) = q$.
		\subitem $\mathbb{Q}_\pi$ is torsion free.
\end{enumerate}

%Fraleigh Part III, Chapters 13-17 
\section{Homomorphisms \& Factor Groups}
%Also contains Group Action and its Applications
\begin{definition}
	Let $\phi : G \to G'$ be a homomorphism.
	Then $\phi[G]$ is the range of $\phi$.
\end{definition}
	Compositions of group homomorphisms is again a group homomorphism.

\begin{definition}
	Let $\phi : G \to G'$ be a group homomorphism.
	Then, the \textbf{kernel} of $\phi$, 
	$$ \ker(\phi) = \phi^{-1}[e'] = \{ g \in G : \phi(g) = e' \} $$
\end{definition}

\section{Properties of Homomorphisms}
Let $\phi : G \to G'$.
\begin{enumerate}
	\item $\phi(e) = e'$.
	\item $\phi(a^{-1}) = \phi(a)^{-1}$.
	\item $H \le G \implies \phi[H] \le \phi[G] \le G'$.
	\item $K' \le \phi[G] \implies \phi^{-1}[K'] \le G$.
	\item Let $N = \ker(\phi)$. Then $\phi^{-1}(\phi(a)) = aN$. And $\phi$ is injective iff $N$ is trivial.
	\item Let $\phi : G \to G'$ with $\ker(\phi)=N$.
		\subitem Rule for Kernel : $G/N \cong \phi[G] \implies o(G)/o(N) = o(\phi[G]) \implies o(G) | o(N)o(G')$
		\subitem Rule for Generators : $(gh)^n = e \implies \phi(gh^n) = e' \implies o(\phi(g) \phi(h))|o(G)$,
	\item $T : \mathbb{Z}_8 \to \mathbb{Z}_{12}$ where $T(x)=4x$ is not a homomorphism (by Rule of generators).
		\subitem \textcolor{red}{Number of surjection homomorphisms $\phi : \mathbb{Z}_n \to \mathbb{Z}_m$ is $\phi(m)$ where $m|n$.}
	\item Given $G,G'$ and normal subgroup $N$. The homomorphism $\phi : G \to G'$ with $\ker(\phi)=N$ exists only if $o(G)/o(N) < o(G')$. (Rule of Kernel)\\
	proof : $\not\!\exists \phi : S_4 \to S_3$ with $\ker(\phi) = \mathbb{Z}_2$ as $S_4/\mathbb{Z}_2$ is too big to be a subgroup of $S_3$.
	\item If $\phi : G \to G'$ is surjective and $G$ is cyclic(abelian), then $G'$ is cyclic(abelian).
	\item If $\phi : G \to G'$ is injective, then $G \cong \phi[G] \le G'$.
		\subitem There does not exists an injective homomorphism, $\phi : S_n \to \mathbb{C}^\ast$ as $\phi : S_n \to \phi[S_n]$ where $\phi[S_n] \le \mathbb{C}^ast$ is an isomorphism. However, subgroups of $\mathbb{C}^\ast$ is abelian.
	\item $\phi : G \to G$ where $\phi(x)=x^m$ is an automorphism iff $\gcd(m,n) = 1$.
\end{enumerate}

\begin{definition}
	Let $H \le G$. $H$ is a \textbf{characteristic subgroup} if $\phi[H] \subset H$ for every automorphism $\phi$ on $G$.
\end{definition}

\section{Properties of normal subgroups}
\begin{enumerate}
	\item Intersection of normal subgroups are again normal.
	\item For every subset $S$ of a group $G$, there exists a minimal normal subgroup of $G$ containing $S$.
	\item Subgroup of index two is normal (if exists).
	\item Subgroups of the center $Z(G)$ are normal.\\
		$H = \{ I_3,2I_3,4I_3 \} \trianglelefteq GL(3,F_{11})$ as $H \le Z(GL(3,F_{11})) = \{ aI_3 : a \in F^\ast_{11} \}$
	\item \textcolor{red}{$\forall k|n,\ \{ m \in \mathbb{Z}_n^\times : m \cong 1 \pmod{k} \} \trianglelefteq \mathbb{Z}_n^\times$}\\
		$\{ 1,7,13,19 \} \trianglelefteq \mathbb{Z}_{30}^\times$ where $k = 6$.
		%$U_k(n) \cong \{ m \in \mathbb{Z}_n^\times : m \cong 1 \pmod{k} \} $ where $k|n$ in AIM Notes
	\item \textcolor{red}{Characteristic subgroups are normal.}
	\item Let $\phi : G \to G'$ be a homomorphism. Then $\ker(\phi) = N$ is normal subgroup of $G$.
	\item Let $\phi : G \to G'$.  If $N \trianglelefteq G$, then $\phi[N] \trianglelefteq \phi[G]$.  If $N' \trianglelefteq G'$, then $\phi^{-1}(N') \trianglelefteq G$.
	\item Intermediate subgroup condition : Let $K \le H \le G$ and $K \trianglelefteq G$ then $K \trianglelefteq H$.
	\item Let $K \le H \le G$. If $H,K$ are normal subgroups of $G$, then $G/H \trianglelefteq G/K$.
	\item $K \trianglelefteq H \trianglelefteq G \nimplies K \trianglelefteq G$
		\begin{proof}
			$D_5 \trianglelefteq D_{10} \trianglelefteq D_{20}$. But $D_5 \not\trianglelefteq D_{20}$.
		\end{proof}
	\item Let $H \le G$ and $N \trianglelefteq G$. Then $HN = \{ hn : h \in H, n \in N\}$ is the smallest subgroup of $G$ containing both $N$ and $H$.
	\item Let $H,K$ be normal subgroups of $G$, then $HK$ is normal in $G$.
	\item Let $H,K$ be normal subgroups of $G$ such that $H \cap K = \{ e \}$. Then $hk = kh$.
	\item $Z(G) \trianglelefteq G$ and $Z(G/Z(G)) \trianglelefteq G/Z(G)$.
	\item Let $\gamma : G \to G/Z(G),\ \gamma(g) = gZ(G)$. Then $\gamma^{-1}(Z(G/Z(G))) \trianglelefteq G$.
\end{enumerate}

\begin{definition}
	Let $N$ be a normal subgroup of $G$.
	The \textbf{quotient group} $G/N$ is the set of all left cosets of $N$ with binary operation $g_1N \ast g_2N = (g_1g_2)N$.
\end{definition}

\begin{theorem}
	Let $N \trianglelefteq G$. $\gamma : G \to G/N$ where $\gamma(g) = gN$ is canonical homomorphism with $\ker(\gamma)=N$.
\end{theorem}

\begin{theorem}
	Let $\phi : G \to G'$ be a homomorphism with $\ker(\phi)=N$. Then there exists a canonical homomorphism $\gamma : G \to G/N$ where $\gamma(g) = gN$ such that $G/N \cong \phi[G]$.
\end{theorem}

\begin{theorem}
	Let $G,G'$ be groups with normal subgroups $H,H'$. Let $\phi : G \to G'$ be a homomorphism with $\phi[H] \le H'$.
	Then there exists an induced canonical homomorphism $\phi_\ast : G/H \to G'/H'$ where $\phi_\ast(gH) = \phi(g)H'$.
\end{theorem}

\begin{definition}
	The map $x \to gxg^{-1}$ is the \textbf{inner automorphism} of $G$ by $g$.
\end{definition}
\begin{enumerate}
	\item The set of all inner automorphisms on $G$ is a group, say $Inn(G)$.
	\item \textcolor{red}{$Inn(G) \cong G/Z(G)$.}
	\item $Inn(G) \trianglelefteq Aut(G)$.
	\item Let $G$ be a finite cyclic group of order $n$. Then $Aut(G) \cong \mathbb{Z}_n^\times$.
		\subitem $Aut(V) \cong S_3$.
		\subitem $Aut(Q_8) \cong S_4$.
		\subitem $Aut(F \times F \times \dots F) \cong GL(n,F)$.
		\subitem $Aut(A_n) \cong Aut(S_n) \cong S_n$, $n \ne 6,\ n > 2$
		\subitem $Aut(A_6) \cong Aut(S_6) \cong S_6 \rtimes Z_2$
	\item Outer automorphism group is the quotient group, $Out(G) \cong Aut(G)/Inn(G)$.
	\item A group $G$ is complete if both center $Z(G)$ and outer automorphism group $Out(G)$ are trivial.
		\subitem $S_n$ is complete, $n \ge 3,\ n \ne 6$.
		\subitem If $G$ is a nonabelian simple group, then $Aut(G)$ is complete.
	\item $G \cong Aut(G) \nimplies G$ is complete.
	\begin{proof}
		$D_4 \cong Aut(D_4)$, $D_4$ is not complete.
	\end{proof}
\end{enumerate}

\begin{definition}
	The \textbf{conjugacy class} of $x$, $Cl(x) = \{ gxg^{-1} : g \in G\}$.
\end{definition}

\begin{definition}
	Let $H,K \le G$. The subgroups are conjugates if $\exists g \in G,\ K=i_g[H]$.
\end{definition}
\begin{enumerate}
	\item Conjugacy is an equivalence relation on the set of all subgroups of $G$.
	\item Normal subgroups are alone in their conjugacy equivalence class.
\end{enumerate}

\begin{definition}
	A group $G$ is simple if it does not have a proper, nontrivial, normal subgroup.
\end{definition}

\begin{enumerate}
	\item $M$ is a maximal normal subgroup of $G$ iff $G/M$ is simple.
	\item Abelian simple groups are cyclic groups of prime order, say $\mathbb{Z}_p$.
	\item $G/Z(G)$ is cyclic iff $G$ is abelian.
	\begin{proof}
		Let $gZ(G)$ be a generator of $G/Z(G)$. Let $g_1,g_2 \in G$. Then $g_1 = g^{n_1}z_1$ and $g_2 = g^{n_2}z_2$ where $z_1,z_2 \in Z(G)$.
		Thus, $g_1g_2 = g_2g_1$. Therefore, $G$ is abelian.
		If $G$ is abelian, then $Z(G) \cong G$ and $G/Z(G)$ is trivial, thus cyclic.
	\end{proof}
\end{enumerate}

\begin{definition}
	An element $g \in G$ is a \textbf{commutator} if $g=aba^{-1}b^{-1}$ for some $a,b \in G$.
\end{definition}
\begin{enumerate}
	\item The set of all commutators in a group $G$ is a subgroup of $G$, say \textbf{commutator subgroup} $C$.
	\item Commutator subgroup $C$ is the smallest normal subgroup of $G$ such that $G/C$ is abelian.
	\item Let $N \trianglelefteq G$. $G/N$ is abelian iff $C \le N$.
	\item Commutator subgroup of a simple group is either trivial(abelian) or the whole group(nonabelian).
	\item Commutator subgroup of $S_n$ is $A_n$.
\end{enumerate}
\begin{definition}
	A group is \textbf{perfect} if the commutator subgroup is the whole group.
\end{definition}
\begin{enumerate}
	\item Any nonabelian, simple group is perfect. 
	\item Direct product of nonabelian simple groups in perfect but not simple.
	\item $SL(2,F_5)$ is a perfect group which is not simple.
\end{enumerate}

\begin{definition}
	An \textbf{action} of group $G$ on a set $X$ is a function $\ast : G \times X \to X$ where
	\begin{enumerate}
		\item $\forall x \in X,\ ex = x$
		\item $\forall x \in X,\ \forall g_1,g_2 \in G,\ (g_1g_2)x = g_1(g_2x)$
	\end{enumerate}
\end{definition}
	The set $X$ is $G$-set if $G$ acts on $X$.
	Let $S \subset G$ such that $\forall s \in S,\ Gs \subset S$. Then $S$ is a sub $G$-set.

\begin{theorem}
	Let $X$ be a $G$-set.
	Then $\phi : G \to S_X$ where $\phi(g) = \sigma_g,\ \sigma_g(x) = gx$ is the group action induced homomorphism.
	% with $\ker(\phi) = ?$.
\end{theorem}

\begin{enumerate}
	\item $\phi$ is the permutation representation of $G$ induced by the group action of $G$ on $X$.
	\item Group action is \textbf{faithful} if $e \in G$ is the only element that fixes every $x \in X$.
		\subitem For a faithful group action, the kernel of the induced homomorphism is trivial.
	\item Group action is \textbf{transitive} if $\forall x_1,x_2 \in X,\ \exists g \in G,\ gx_1 = x_2$.
	\item Every group $G$ is a $G$-set where the action is both faithful and transitive. 
	\item Let $H \le G$. 
		\subitem Conjugation is an action of $G$ on $H$, say $(g,h) \to ghg^{-1}$. 
		\subitem Left multiplication is an action of $G$ on $H$, say $(g,h) \to gh$. 
	\item Let $H \le G$ and $L_H$ be the set of left cosets of $H$. 
		\subitem $L_H$ is a $G$-set under conjugation, say $(g,aH) \to g(aH)g^{-1}$.
	\item Let $V(F)$ be a vector space. Then $V$ is an $F^\ast$-set.
	\item Disjoint union of $G$-sets is also a $G$-set.
	\item $G_x$ is the \textbf{isotropy subgroup} of $G$ containing all elements that fix $x$.
	\item $X_g$ is the subset of $X$ fixed by $g \in G$.
	\item The relation $x_1 \sim_g x_2 \iff gx_1=x_2$ is an equivalence relation on $X$.
	\item The equivalence classes of the above relation, $Gx$ is the \textbf{orbit} of $x$ in a $G$-set $X$, 
	\item Orbit Stabiliser theorem : $|Gx| = (G:G_x)$
	\item Burnside's Formula, $r |G| = \sum_{g \in G}|X_g|$
\end{enumerate}

\section{Group Homomorphisms}
\begin{enumerate}
	\item $\phi : S_n \to \mathbb{Z}_2$ where $\phi(\sigma) = 1$ if the $\sigma$ is an odd permutation and $\phi(\sigma) = 2$ otherwise. Then $\ker(\phi)=A_n$.
	\item Evaluation Homomorphism, $\phi_c : F \to \mathbb{R}$ where $\phi_c(f) = f(c)$ where $F$ is the additive group of all functions $f : \mathbb{R} \to \mathbb{R}$.
	\item $\phi : \mathbb{R}^n \to \mathbb{R}^m$ where $\phi(v) = Av,\ A \in M_{m \times n}(\mathbb{R})$.
	\item The trace, $tr : M_n(\mathbb{R}) \to \mathbb{R}$.
	\item The trace, $tr : M(n,F) \to F$. Then $\ker(tr)$ is $n^2-1$ dimensional over $F$.
	\item Determinant $\det : GL(n,\mathbb{R}) \to \mathbb{R}^\ast$ where $\det(A)=|A|$ with $\ker(\det) = SL(n,\mathbb{R})$ and $\det[GL(n,\mathbb{R})] \cong \mathbb{R}^\ast$.
	\item Determinant $\det : GL(n,F_q) \to F_q^\ast$ where $\det(A)=|A|$ with $\ker(\det) = SL(n,F_q)$ and $\det[GL(n,F_q)] \cong F_q^\ast$.
	$$|GL(n,F_q)| = \prod_{r=0}^{n-1} (q^n-q^r)$$
	$$|SL(n,F_q)| = \frac{|GL(n,F_q)|}{q-1} \text{ since } GL(n,F_q)/SL(n,F_q) \cong F_q^\ast$$
	\item $\phi : \mathbb{Z}_n^\times \to \mathbb{Z}_k^\times$ with $\ker(\phi) = \{ m \in \mathbb{Z}_n^\times : m \cong 1 \pmod{k} \}$.
	\item $\phi_r : \mathbb{Z} \to \mathbb{Z}$ where $\phi_r(n) = rn$. $\phi_0$ is trivial, $\phi_1$ is identity, $\phi_{-1}$ is surjective.
	\item Projection map $\pi_i : \prod G_j \to G_i$ where $\pi_i(g_1,g_2,\dots,g_n) = g_i$.
	\item $\sigma : F \to \mathbb{R}$ where $\sigma(f) = \int_0^1 f(x)\ dx$ and $F$ is the additive group of all continuous functions $f : [0,1] \to \mathbb{R}$.
	\item $\gamma : \mathbb{Z} \to \mathbb{Z}_n$ where $\gamma(m) = r,\ m =qn+r,\ 0 \le r < n$.
	\item $\phi : \mathbb{C}^\ast \to \mathbb{R}^\ast$ where $\phi(z) = |z|$.
	Left cosets $aN$ are circles of radius $a$ about origin.
	\item Let $D$ be the set of all differentiable function. Define $\phi : D \to F$ where $\phi(f) = f'$.
	Left cosets $fN$ are $f(x)+C$.
	\item $\phi : \mathbb{Z} \to \mathbb{R}$ where $\phi(n) = n$.
	\item $\phi : \mathbb{R} \to \mathbb{Z}$ where $\phi(x) = [x]$ with $\ker(\phi)=[0,1)$.
	\item $\phi : \mathbb{R}^\ast \to \mathbb{R}^\ast$ where $\phi(x) = |x|$ with $\ker(\phi) = \{1,-1\} \cong \mathbb{Z}_2$.
	\item $\phi : \mathbb{Z}_6 \to \mathbb{Z}_2$ where $\phi(n) \cong n \pmod{2}$ with $\ker(\phi)=\{0,2,4\} \cong \mathbb{Z}_3$.
	\item $\phi : \mathbb{R} \to \mathbb{R}^\ast$ where $\phi(x) = 2^x$ with $\ker(\phi) = \{0\}$.
	\item Injection map, $\phi_i : G_i \to \prod G_j$ where $\phi_i(g)=(e_1,e_2,\dots,ge_i,\dots,e_n)$ with $\ker(\phi)=\{e_i\}$.
	\item $\phi : G \to G$ where $\phi(g) = g^{-1}$ with $\ker(\phi)=\{e\}$.
	\item $\phi : F \to F$ where $\phi(f) = f^{\prime\prime}$ where $F$ is the set of all functions $f$ having derivatives of all orders with $\ker(\phi)=\{ax+b : a,b \in \mathbb{R}\}$.
	\item $\phi : F \to F$ where $\phi(f) = \int_0^4 f(x)\ dx$ where $F$ is the set of all continuous functions $f : \mathbb{R} \to \mathbb{R}$.
	\item $\phi : F \to F$ where $\phi(f) = 3f$ with $\ker(\phi)=\{0\}$.
	\item $\phi : F \to \mathbb{R}^\ast$ wherer $\phi(f) = \int_0^1 f(x)\ dx$ where $F$ is the multiplicative group of continuous functions $f : \mathbb{R} \to \mathbb{R}$ such that $f(x) \ne 0$.
	\item $\phi : \mathbb{Z} \to \mathbb{Z}_7$ where $\phi(1) = 4$ with $\ker(\phi) = 7\mathbb{Z}$.
	\item $\phi : \mathbb{Z} \to \mathbb{Z}_{10}$ where $\phi(1) = 6$ with $\ker(\phi) = 5\mathbb{Z}$.
	\item $\phi : \mathbb{Z} \to S_8$ where $\phi(1) = (1,4,2,6)(2,5,7)$ with $\ker(\phi) = 12\mathbb{Z}$.
	\item $\phi : \mathbb{Z}_{10} \to \mathbb{Z}_{20}$ where $\phi(1) = 8$ with $\ker(\phi) = \{0,5\} \cong \mathbb{Z}_2$.
	\item $\phi : \mathbb{Z}_{24} \to S_8$ where $\phi(1) = (1,4,6,7)(2,5)$ with $\ker(\phi) = \{0,4,8,12,16,20\} \cong \mathbb{Z}_6$.
	\item $\phi : \mathbb{Z} \times \mathbb{Z} \to \mathbb{Z}$ where $\phi(1,0)=3$, $\phi(0,1)=-5$ with $\ker(\phi)=\gen{(5,3)} \cong \mathbb{Z}$.
	\item $\phi : \mathbb{Z} \times \mathbb{Z} \to \mathbb{Z} \times \mathbb{Z}$ where $\phi(1,0)=(2,-3)$ and $\phi(0,1) = (-1,5)$ with $\ker(\phi) = \{ (0,0) \}$.
	\item $\phi : \mathbb{Z} \times \mathbb{Z} \to S_{10}$ where $\phi(1,0) = (3,5)(2,4)$ and $\phi(0,1) = (1,7)(6,10,8,9)$ with $\ker(\phi)= \gen{(2,4)} \cong \mathbb{Z}$.
	 %Remember : $2\mathbb{Z} \times 4\mathbb{Z} \not\cong \gen{(2,4)} \cong \mathbb{Z}$
	\item $\phi : \mathbb{Z}_{12} \to \mathbb{Z}_5$ where $\phi(1)=0$ with $\ker{\phi}=\mathbb{Z}_{12}$.
	\item $\phi : \mathbb{Z}_{12} \to \mathbb{Z}_4$ where
		\subitem $\phi(1)=0$ with $\ker(\phi)=\mathbb{Z}_{12}$
		\subitem $\phi(1)=1$ with $\ker(\phi)=\{0,4,8\} \cong \mathbb{Z}_3$
		\subitem $\phi(1)=2$ with $\ker(\phi)=\{ 0,6 \} \cong \mathbb{Z}_2$
		\subitem $\phi(1)=3$ with $\ker(\phi)=\{ 0,4,8\} \cong \mathbb{Z}_3$
	\item $\phi : \mathbb{Z}_2 \times \mathbb{Z}_4 \to \mathbb{Z}_2 \times \mathbb{Z}_5$ where
		\subitem $\phi(1,0) = (0,0) ,\ \phi(0,1) = (0,0)$ with $\ker(\phi) = \mathbb{Z}_2 \times \mathbb{Z}_4$
		\subitem $\phi(1,0) = (1,0) ,\ \phi(0,1) = (0,0)$ with $\ker(\phi) = \{ 0 \} \times \mathbb{Z}_4$
		\subitem $\phi(1,0) = (0,0) ,\ \phi(0,1) = (1,0)$ with $\ker(\phi) = \mathbb{Z}_2 \times \{ 0,2 \} \cong V$
		\subitem $\phi(1,0) = (1,0) ,\ \phi(0,1) = (1,0)$ with $\ker(\phi) = \{ 0 \} \times \{ 0,2 \}$
	\item $\phi : \mathbb{Z}_3 \to \mathbb{Z}$ where $\phi(1) = 0$
	\item $\phi : \mathbb{Z}_3 \to S_3$ where 
		\subitem $\phi(1) = ()$ with $\ker(\phi) = \mathbb{Z}_3$
		\subitem $\phi(1) = (1,2,3)$ with $\ker(\phi) = \{ 0 \}$
		\subitem $\phi(1) = (1,3,2)$ with $\ker(\phi) = \{ 0 \}$
	\item $\phi : \mathbb{Z} \to S_3$ where $\phi(1) = ()$ with $\ker(\phi) = \mathbb{Z}$.
	\item $\phi : \mathbb{Z} \times \mathbb{Z} \to 2\mathbb{Z}$ where $\phi(1,0) = 2s ,\ \phi(0,1)= 2t $ with $\ker(\phi) = \{ 0 \},\ s,t \ne 0 $.
	\item $\phi : 2\mathbb{Z} \to \mathbb{Z} \times \mathbb{Z}$ where $\phi(2) = (s,t)$ with $\ker(\phi)= \{ 0 \},\ s,t \ne 0$.
	\item $\phi : D_4 \to S_3$ where 
		\subitem $\phi(R_{90}) =(),\ \phi(\mu) =()$ with $\ker(\phi) = D_4$.
		\subitem $\phi(R_{90}) =(i,j),\ \phi(\mu) =()$ with $\ker(\phi) = \{ 0,R_{180},\mu, R_{180}\mu \}$.
		\subitem $\phi(R_{90}) =()$ or $\phi(\mu) =(i,j)$ with $\ker(\phi) = \{ 0,R_{90},R_{180},R_{270} \}$.
		\subitem $\phi(R_{90}) =(i,j)$ or $\phi(\mu) =(i,j)$ with $\ker(\phi) = \{ 0,R_{90}\mu,R_{180},R_{270}\mu \}$.
		\subitem $\phi : D_4 \to S_3,\ \ker(\phi) \not\cong \mathbb{Z}_2$ since $S_3$ don't have a subgroup isomorphic to $D_4/\mathbb{Z}_2$
	\item $\phi : S_3 \to S_4$ where
		\subitem $\phi(1,2)=(),\ \phi(1,2,3)= () $ with $\ker(\phi) = S_3$.
		\subitem $\phi(1,2)=(i,j),\ \phi(1,2,3)= () $ with $\ker(\phi) = \{ (),(1,2,3),(1,3,2) \}$.
		\subitem $\phi(1,2)=(),\ \phi(1,2,3)= (i,j,k) $ with $\ker(\phi) =k \{ (),(1,2) \} $.
		\subitem $\phi(1,2)=(i,j),\ \phi(1,2,3)= (i,j,k) $ with $\ker(\phi) = \{ () \} $.
		\subitem $\phi(1,2)=(i,j)(k,l),\ \phi(1,2,3)= () $ with $\ker(\phi) = \{(),(1,2,3),(1,3,2)\} $.
	\item $\phi : S_4 \to S_3$ where
		\subitem $\phi(1,2)=(),\ \phi(1,2,3,4)= () $ with $\ker(\phi) = S_4$.
		\subitem $\phi(1,2)=(i,j),\ \phi(1,2,3,4)= (i,j) $ with $\ker(\phi) = A_4$.
		\subitem $\phi(1,2)=(i,j),\ \phi(1,2,3,4)= (i,k) $ is surjective with \\$\ker(\phi) = \{ (), (1,3)(2,4), (1,2)(3,4), (1,4)(2,3) \} \cong \mathbb{Z}_2 \times \mathbb{Z}_2 \cong V$.
\end{enumerate}

\section{Counter Examples}
\begin{enumerate}
	\item $\phi : \mathbb{Z}_9 \to \mathbb{Z}_2$ where $\phi(n) \cong n \pmod{2}$. But, $\phi(2+8) \ne \phi(2)+\phi(8)$.
	\item $\phi : M_n(\mathbb{R}) \to \mathbb{R}$ where $\phi(A) = \det(A)$. However, $\det(A+B) \ne \det(A) + \det(B)$.
	\item $\phi : GL(n,\mathbb{R}) \to \mathbb{R}^\ast$ where $\phi(A) = tr(A)$. However, $tr(AB) \ne tr(A) tr(B)$.
	\item $\phi : S_3 \to S_4$ where $\phi(1,2)=(1,2),\ \phi(1,2,3)= (1,3,4)$ is not a homomorphism.\\ Let $\sigma = (1,2)(1,2,3) = (2,3)$, $\phi(\sigma) = \phi(1,2)\phi(1,2,3)=(1,3,4,2)$ and $\phi(\sigma^2) \ne ()$.
	\item $\phi : S_3 \to S_4$ where $\phi(1,2)=(1,2)(3,4),\ \phi(1,2,3)= (1,2,3)$ is not as well.\\ Let $\sigma = (1,2)(1,2,3) = (2,3)$, $\phi(\sigma) = \phi(1,2)\phi(1,2,3)=(2,4,3)$ and $\phi(\sigma^2) \ne ()$.
	\item $\phi(1,2)=(i,j),\ \phi(1,2,3,4)= () $.\\
	Let $\sigma = (2,3,4) = (1,2)(1,2,3,4)$. Then $\phi(\sigma)=(i,j)$ and $\phi(\sigma^3) \ne ()$.
	\item $\phi(1,2)=(),\ \phi(1,2,3,4)= (1,2)$.\\
	Let $\sigma = (1,2)(1,2,3,4) = (2,3,4)$. $\phi(\sigma) = (1,2)$ and $\phi(\sigma^3) \ne ()$.
\end{enumerate}

\section{Special Homomorphisms}
\begin{enumerate}
	\item There are  two homomorphisms of $\mathbb{Z}$ onto $\mathbb{Z}$. $\phi_1(n) = n$ and $\phi_2(n) = -n$.
	%$\phi : \mathbb{Z} \to \mathbb{Z}_2$ where $\phi(n) = n^2 \pmod{3}$ is not well-defined as $1 \in \mathbb{Z}_2 \ne 1 \in \mathbb{Z}_3$.
	\item There are countably many homomorphisms of $\mathbb{Z}$ into $\mathbb{Z}$. $\phi_r(n) = rn,\ r \in \mathbb{Z}$.
	\item There is a unique homomorphisms of $\mathbb{Z}$ into $\mathbb{Z}_2$. $\phi(n) \cong n \pmod{2}$.
	\item $\phi_g : G \to G$ where $\phi_g(x) = gx$ is a homomorphism only when $g = e$.
	\item $\phi_g : G \to G$ where $\phi_g(x) = gxg^{-1}$ is a homomorphism with $\ker(\phi_g) = \{ e \}$.
	\item There exists exactly 24 surjective homomorphisms from $S_4$ onto $S_3$. However, the $\ker(\phi) = \mathbb{Z}_2 \times \mathbb{Z}_2$ as it is the only normal subgroup of $S_4$ with order $4$.
	\item The field $\entity{\left\{\begin{bmatrix} a & -b \\ b & a \end{bmatrix} : a,b \in \mathbb{R} \right\},+,\times} \cong \mathbb{C}$ where $\phi\left(\begin{bmatrix} a & -b \\ b & a \end{bmatrix}\right) = a+ib$.
\end{enumerate}

\section{Quotient Groups}
\begin{enumerate}
	\item $\mathbb{R}/n\mathbb{R} \cong \{ e \}$ where $n\mathbb{R} = \{ nr : r \in \mathbb{R} \}$.
	\item $S_n/A_n \cong \mathbb{Z}_2$, $n>1$.
	\item $A_4/V  = \left\{ [V],\ (1,2)[V],\ (1,2,3,4)[V] \right\} \cong \mathbb{Z}_3$.
	\item $(\mathbb{Z}_4 \times \mathbb{Z}_6)/\gen{(0,1)} \cong \mathbb{Z}_4$.
	\item $(\mathbb{Z}_4 \times \mathbb{Z}_6)/\gen{(0,2)} \cong \mathbb{Z}_4 \times \mathbb{Z}_2$.
	\item $(\mathbb{Z}_4 \times \mathbb{Z}_6)/\gen{(2,3)} \cong \mathbb{Z}_4 \times \mathbb{Z}_3$.
	\item $D_n/\mathbb{Z}_n \cong \mathbb{Z}_2$, $n>2$. And $D_n \cong \mathbb{Z}_n \rtimes \mathbb{Z}_2$.
	\item \textcolor{red}{$\mathbb{Z}_n^\times/N \cong \mathbb{Z}_k^\times$ where $N = \{ m \in \mathbb{Z}_n^\times : m \cong 1 \pmod{k} \}$.}
	\item Factor groups of cyclic groups are cyclic. $\mathbb{Z}_n/\mathbb{Z}_d \cong \mathbb{Z}_{n/d}$, $d|n$.
	\item $F/K \le F$ where $F$ is the additive group of all continuous functions $f : \mathbb{R} \to \mathbb{R}$ and $K$ is the subgroup of all constant functions.
	\item $F^\ast/K^\ast \le F^\ast$ where $F^\ast$ is the multiplicative group of all continuous functions $f : \mathbb{R} \to \mathbb{R}$ such that $f(x) \ne 0$ and $K^\ast$ is the subgroup of all nonzero constant functions.
\end{enumerate}

\section{Maximal \textcolor{blue}{Normal} Subgroups}
\begin{enumerate}
	\item $S_n$ : \textcolor{blue}{$A_n$}, $n > 5$
		\subitem $S_4$ : \textcolor{blue}{$A_4,\ \mathbb{Z}_2 \times \mathbb{Z}_2$}
	\item $A_4$ : \textcolor{blue}{$\mathbb{Z}_2 \times \mathbb{Z}_2$}
		\subitem $A_n$ is simple, $n > 4$.
	\item $D_n$ : \textcolor{blue}{$D_{n/2},\ \mathbb{Z}_n$}, $D_d$ where $d|n,\ n>2$.
		\subitem $D_4$ is the only dihedral group in which $\mathbb{Z}_2 \times \mathbb{Z}_2$ is normal. (index $2$)
\end{enumerate}

\section{Order of Quotient Groups}
\begin{enumerate}
	\item $\mathbb{Z}_6/\gen{3}$.
	We have $|H| = o(3) = 6/\gcd(6,3) = 2$ and $|G/H| = |G|/|H| = 6/2 = 3$
	\item $(\mathbb{Z}_4 \times \mathbb{Z}_{12})/(\gen{2} \times \gen{2})$.
	We have, $o(2)=4/\gcd(4,2) = 2$ and $o(2) = 12/\gcd(12,2) = 6$. And $|G|/|H| = 48/12 = 4$.
	\item $(\mathbb{Z}_4 \times \mathbb{Z}_2)/\gen{(2,1)}$.
	We have, $o(2,1) = lcm(o(2),o(1)) = lcm(2,2) = 2$. And $|G/H| = 8/2 = 4$.
	\item $(\mathbb{Z}_3 \times \mathbb{Z}_5)/\{0\} \times \mathbb{Z}_5$.
	Clearly, $|G/H| = 15/5 = 3$.
	\item $(\mathbb{Z}_2 \times \mathbb{Z}_4)/\gen{(1,1)}$.
	We have, $o(1,1) = lcm(o(1),o(1)) = lcm(2,4) = 4$. And $|G/H| = 8/4 = 2$.
	\item $(\mathbb{Z}_{12} \times \mathbb{Z}_{18})/\gen{(4,3)}$.
	We have $o(4,3) = lcm(o(4),o(3)) = lcm(3,6) = 6$. And $|G/H| = 12 \times 18/6 = 36$.
	\item $(\mathbb{Z}_2 \times S_3)/\gen{(1,\rho_1)}$ where $\rho_1 = (1,2,3)$.
	We have $o(1,\rho_1) = lcm(o(1),o(\rho_1)) = lcm(2,3) = 6$. And $|G/H| = 12/6 = 2$.
	\item $(\mathbb{Z}_{11} \times \mathbb{Z}_{15})/\gen{(1,1)}$.
	Clearly $o(1,1) = 11 \times 15$. And $|G/G| = 1$.
\end{enumerate}

\section{Order of an element in the quotient group}
\begin{enumerate}
	\item $5+\gen{4} \in \mathbb{Z}_{12}/\gen{4}$.
	$4 \times 5+\gen{4} = 0+\gen{4}$.
	\item $26+\gen{12} \in \mathbb{Z}_{60}/\gen{12}$.
	$6 \times (2+24)+\gen{12} = 0 + \gen{12}$.	
	\item $(2,1)+\gen{(1,1)} \in (\mathbb{Z}_3 \times \mathbb{Z}_6)/\gen{(1,1)}$.
	$3 \times [(1,0)+(1,1)+\gen{(1,1)}] = (0,0) + \gen{(1,1)}$.
	\item $(3,1)+\gen{(1,1)} \in (\mathbb{Z}_4 \times \mathbb{Z}_4)/\gen{(1,1)}$.
	$2 \times [(2,0)+(1,1)+\gen{(1,1)} = (0,0) + \gen{(1,1)}$.
	\item $(3,3) + \gen{(1,2)} \in (\mathbb{Z}_4 \times \mathbb{Z}_8)/\gen{(1,2)}$.
	$8 \times [(2,1) + (1,2)+\gen{(1,2)}] = (0,0) + \gen{(1,2)}$.
	\item $(2,0) + \gen{(4,4)} \in (\mathbb{Z}_6 \times \mathbb{Z}_8)/\gen{(4,4)}$.
	$3 \times [(2,0)+\gen{(4,4)}] = (0,0) + \gen{(4,4)}$.
\end{enumerate}

\section{Conjugate Subgroups}
\begin{enumerate}
	\item $i_{\rho_1}[H]$ where $H = \{ \rho_0, \mu_1 \}$ and $\mu_1 = (2,3)$.\\
	We have, $i_{\rho_1}(\mu) = (1,2,3)(2,3)(1,3,2)=(1,3) = \mu_2$.
	Thus, $i_{\rho_1}[H] = \{ \rho_0, \mu_u \}$.
\end{enumerate}

\section{Group $G$ characterised by $G/Z(G)$}
\begin{enumerate}
	\item If $G$ is non-abelian, finite group then $|Z(G)| \le \frac{1}{4}|G|$. Otherwise $G/Z(G)$ is a group of order $1,2$ or $3$. And groups of order $1,2,3$ are cyclic.
	\item If $G$ is non-abelian, then $Z(G)$ is not a maximal subgroup of $G$.
	\begin{proof}
		Suppose $Z(G)$ is a maximal subgroup of $G$.
		Then $G/Z(G)$ has no nontrivial subgroups.
		That is, $G/Z(G)$ is of prime order and thus cyclic which is not possible as $G$ is non-abelian.
	\end{proof}
	\item For $A_5,S_3,\dots$, the group $G/Z(G)$ is non-abelian.
\end{enumerate}

\section{Group Actions}
\begin{enumerate}
	\item 
\end{enumerate}

\begin{definition}
	Let $G$ be a group. The dual group of $G$, $\hat{G}$ is the abelian group of all homomorphisms $\phi : G \to \mathbb{C}^\ast$.
\end{definition}
	$\widehat{A \times B} \cong \hat{A} \times \hat{B}$

%Fraleigh Part VII - Chapters 34-40
\section{Advanced Group Theory}
\section{Isomorphism Theorems}
\begin{enumerate}
	\item $\forall \phi : G \to G',\ \exists \gamma_N : G \to G/N,\ \phi = \mu \gamma$ where $N = \ker(\phi)$ and $\phi[G] \overset{\mu}{\to} G/N$.
	\item Let $H \le G$ and $N \trianglelefteq G$. Then $(HN)/N \cong H/(H \cap N)$.
		\subitem $|HN| = |H| |N| / |H \cap N|$.
		\subitem If $H \cap N = \{ e \}$, then $|HN| = |H| |N|$.
	\item Let $K \le H \le G$ and $H,K$ are normal subgroups of $G$. Then $G/H \cong (G/K)/(H/K)$.
\end{enumerate}

\begin{definition}
	A \textbf{subnormal series} of a group $G$ is a finite sequence $\{ H_i \}_{i=0}^n$ such that $H_i \trianglelefteq H_{i+1},\ H_0 = \{ e \}$ and $H_n = G$.
\end{definition}

\begin{definition}
	A \textbf{normal series} of a group $G$ is a finite sequence $\{ H_i \}_{i=0}^n$ such that $H_i \trianglelefteq G,\ H_0 = \{ e \}$ and $H_n = G$.
\end{definition}

\begin{definition}
	A subnormal(normal) series of a group $G$ is a \textbf{composition(principal) series} of group $G$ if every quotient group $H_{i+1}/H_i$ is simple.
\end{definition}

\begin{definition}
	A composition series of a group $G$ is \textbf{solvable} if every quotient group $H_{i+1}/H_i$ is abelian.
\end{definition}

\begin{definition}
	The ascending central series of the group $G$ is $\{ e \} \le Z(G) \le Z_1(G) \le Z_2(G) \dots$ where $Z_1(G) = \gamma^{-1}(Z(G/Z(G)))$, $Z_i(G) = \gamma_1^{-1}(Z(G/Z_1(G))) \dots$ and $\gamma : G \to G/Z(G),\ \gamma(g) = gZ(G)$ and $\gamma_1 : G \to G/Z_1(G),\ \gamma_1(g) = gZ_1(G),\dots$.
\end{definition}

\begin{enumerate}
	\item Zassenhaus Lemma (Butterfly Lemma) : Let $H^\ast \trianglelefteq H$ and $K^\ast \trianglelefteq K$. Then 
	\subitem $H^\ast(H \cap K^\ast) \trianglelefteq H^\ast(H \cap K)$,
	\subitem $K^\ast(H^\ast \cap K) \trianglelefteq K^\ast(H \cap K)$,
	\subitem $(H^\ast \cap K)(H \cap K^\ast) \trianglelefteq (H \cap K)$, and 
		$$ H^\ast(H \cap K)/H^\ast(H \cap K^\ast) \cong K^\ast(H \cap K)/K^\ast(H^\ast \cap K) \cong (H \cap K)/(H^\ast \cap K)(H \cap K^\ast) $$
	\item Schreier Theorem : Any two subnormal series of a group $G$ have isomorphic refinements.
	\item Jordan-H\"older Theorem : Any two composition(principal) series of a group $G$ are isomorphic.
	\item Every normal subgroup $N$ of $G$ belongs to some composition series of the group $G$.
	\item Finite product of solvable groups is solvable.
\end{enumerate}

\begin{definition}
	If every element of $G$ has order a power of prime $p$, then $G$ is a \textbf{$p$-group}.
	Let $H \le G$ and $H$ is a $p$-group, then $H$ is a \textbf{$p$-subgroup} of $G$.
\end{definition}
\begin{definition}
	Let $G$ be a group and $H \le G$. The \textbf{normaliser} $N[H]$ of $H$ is the largest subgroup of $G$ such that $H \trianglelefteq N[H]$.
\end{definition}
\begin{definition}
	Maximal $p$-subgroup is a \textbf{Sylow $p$-subgroup} of $G$.
\end{definition}
\begin{definition}
	The \textbf{class equation} of $G$ is $|G| = c + n_{c+1} + \dots + n_r$ where $n_j$ is the length of $j$th orbit in the partition of $G$ under conjugation and $c = |Z(G)|$ is the number of element that are alone in their conjugacy class.
\end{definition}
\begin{enumerate}
	\item The set of all Sylow $p$-subgroups of $G$, $Syl_p(G)$ is a $G$-set with conjugation action.
	\item Let $X$ be a finite $G$-set and $|G|=p^n$. Then $|X| \cong |X_G| \pmod{p}$.
	\item Cauchy's theorem : Let $G$ be a finite group and $p$ divides the order of $G$, then $G$ has element $g$ of order $p$.
	\item Let $H$ be a $p$-subgroup of a finite group $G$. Then $(N[H]:H) \cong (G:H) \pmod{p}$.
		\subitem If $p$ divides the index of $H$ in $G$, $(G:H)$, then $N[H] \ne H$.
		\subitem $N[H]$ is isomorphic to the group of all inner automorphisms $G$ that map $H$ onto itself.
	\item The class equation of various groups,
		\subitem $G :  n = n$, if $G$ is abelian.
		\subitem $G : p^3 = p + p + \dots + p$, if $G$ non-abelian.
		\subitem $S_3 : 6 = 1 + 2 + 3$.
		\subitem $S_4 : 24 = 1 + 3 + 8 + 6 + 6$.
		\subitem $S_5 : 120 = 1+ 10 + 15 + 20 + 20 + 24 + 30$.
		\subitem $A_4 : 12 = 1 + 3 + 4 + 4$.
		\subitem $A_5 : 60 = 1 + 20 + 12 + 12 + 15$.
		\subitem $D_4 : 8 = 2 + 2 + 2 + 2$.
		\subitem $D_5 : 10 = 1 + 2 + 2 + 5$.
		\subitem $D_6 : 12 = 2 + 2 + 2 + 3 + 3$.
		\subitem $Q_8 : 8 = 2 + 2 + 2 + 2$.
	\item Distinct groups can have the same class equation.
\end{enumerate}	

\section{Sylow Theorems}
\begin{enumerate}
	\item If $|G|=p^nm$, then $\{ H_i \}_{i=0}^n$ is a subnormal series such that $|H_i| = p^i$ and $H_i \le G$.
	\item Let $P_1,P_2$ be Sylow $p$-subgroups of a finite group $G$. Then $P_1,P_2$ are conjugate subgroups of $G$.
	\item Let $G$ be a finite group and $p$ divides the order of $G$.
	Then the number of Sylow $p$-subgroups, $n_p \cong 1 \pmod{p}$ and $n_p|o(G)$.
\end{enumerate}

\section{Applications of Sylow theorems}
\begin{enumerate}
	\item Wilson`s theorem : $(p-1)! \cong -1 \pmod{p}$.\\
	$S_p$ has $(p-2)!$ Sylow $p$-subgroups.
	Clearly, $(p-2)! \cong 1 \pmod{p}$ and theorem holds.
	\item Nonabelian group of order $pq$ is isomorphic to $\mathbb{Z}_q \rtimes \mathbb{Z}_p$. It has $q$ Sylow-$p$ subgroups.
	\item Sylow $p$-subgroups are conjugates.
		Suppose $|G|=36$ with four Sylow $3$-subgroups (of order $9$). Then either they are isomorphic to $\mathbb{Z}_9$ or $\mathbb{Z}_3 \times \mathbb{Z}_3$.
\end{enumerate}

\section{$HN$ subgroups}
\begin{enumerate}
	\item $G = \mathbb{Z}_{24},\ H = \gen{4},\ N = \gen{6}$. $HN = \gen{2}$.
	\item $G = \mathbb{Z}_{36},\ H = \gen{6},\ N = \gen{9}$. $HN = \gen{3}$.
\end{enumerate}

\section{Third Isomorphism Theorem}
\begin{enumerate}
	\item $G = \mathbb{Z}_{24},\ H = \gen{4},\ N = \gen{8}$. 
	$G/K = \{ \gen{8}, 1 + \gen{8}, \dots, 7+\gen{8} \}$.\\
	$H/K = \{ \gen{8}, 4+\gen{8} \}$.
	$G/H = \{ \gen{4}, 1+\gen{4}, 2+\gen{4}, 3+\gen{4} \}$.
\end{enumerate}

\section{Non-abelian Groups}
	There are a few classes of non-abelian groups which has every proper subgroup abelian : 
	1) every nonabelian group of order $pq$ where $p|q$, and
	2) two non-abelian groups of order $p^3$.

\section{Semidirect Product}
\begin{definition}
	Let $\phi : H \to Aut(N)$ be a group homomorphism where $N,H$ are two group.
	Then the \textbf{semidirect product} $N \rtimes H$ is defined as the group $\entity{N \rtimes H,\ast}$ where $\ast : (N \times H) \times (N \times H) \to (N \times H)$ such that $(n_1,h_1) \ast (n_2,h_2) = (n_1 \phi_{h_1}(n_2),h_1h_2)$.
\end{definition}

	Let $G$ be a group with nontrivial normal subgroups $N,H \le G$ such that $N \cap H = \{ 1 \}$ and $N \vee H = G$. Then $G/N \cong H$ and $G/H \cong N$. Thus $G \cong N \times H$.\\

	We can extend the notion direct product as follows.
	Let $G$ be a group with nontrivial subgroups $N,H$ such that $N$ is normal and $N \cap H = \{ 1 \}$. Then $G \cong N \rtimes H$ except for $G \cong \mathbb{Z}_4$ and $Q_8$.
	
\begin{definition}
	The \textbf{fundamental group} of a topological space is the group of equivalent classes under homotopy of the loops contained in the space.
\end{definition}

\section{Semidirect Products}
\begin{enumerate}
	\item The dihedral group, $D_n \cong \mathbb{Z}_n \rtimes \mathbb{Z}_2$.
	\item No simple group $G$ can be expressed as a semidirect/direct product.
		\subitem Simple groups are indecomposable.
	\item The fundamental group of the Klein bottle is $\mathbb{Z} \rtimes \mathbb{Z}$.\\
\end{enumerate}


\section{The converse of Lagrange`s theorem} Finite group $G$ not necessarity have subgroups for each divisor of its order.
	For example, the alternating group $A_5$ of order $12$ does not have a subgroup of order $6$.

\section{Classification of Finite Groups}
\begin{enumerate}
	\item By Burnside`s theorem, $p$-Groups have non-trivial center. And $Q_8$ is the smallest non-abelian $p$-group.
	\item By Sylow first theorem, no group of prime power order is simple.
	\item Every group of prime power order is solvable.
	\item Every group $G$ of order $p$ is cyclic and $G \cong Z_p$. The number of generators is $\phi(n)$.
	\item Every group $G$ of order $p^2$ is abelian. There are two groups $Z_{p^2}$ and $Z_p \times Z_p$.
	\item There are exactly five groups of order $p^3$.
	\begin{proof}
		Three abelian groups -- $Z_{p^3}, Z_{p^2} \times Z_p, \text{ and } Z_p \times Z_p \times Z_p$ and two non-abelian groups -- $(Z_p \times Z_p) \rtimes Z_p, \text{ and } Z_{p^2} \rtimes Z_p$ except for $p =2$. For $p=2$, $Z_4 \rtimes Z_2 \cong (Z_2 \times Z_2) \rtimes Z_2 \cong D_4$. However we have $Q_8$, which is another nonabelian group of order $8$.
	\end{proof}
	\item Every non-abelian group $G$ of order $p^3$ has center $Z(G)$ of order $p$.
	\begin{proof}
		Since $G$ is a $p$-group, $G$ has nontrivial center. Suppose $|Z(G)| = p^2$, then $G/Z(G)$ is a cyclic group of order $p$. But $G$ is non-abelian.
	\end{proof}
	\item Every non-abelian group $G$ of order $p^3$ has $p^2+p-1$ distinct conjugacy classes.
	\item Abelian group of order $pq$ is cyclic. Non-abelian group of order $pq$ exists and is isomorphic to $\mathbb{Z}_q \rtimes \mathbb{Z}_p$ provided $q \cong 1 \pmod{p}$.
	\item Every non-abelian group $G$ of order $pq$ has trivial center.
	\begin{proof}
		Suppose nonabelian group $G$ has a nontrivial center of order $p$ (wlog), then $G/Z(G)$ is a cyclic group of order $q$.
		But $G$ is non-abelian. Thus $Z(G)$ is trivial.
	\end{proof}
	\item Every group of square free order is supersolvable. And thus solvable.
	\begin{proof}
		Suppose $|G| = p_1 p_2 \dots p_k$ where $p_1 > p_2 > \dots p_k$. Then there exists a normal series $G_1 \trianglelefteq G_2 \trianglelefteq \dots \trianglelefteq G_k \trianglelefteq G$ such that $|G_1| = p_1$, $|G_2| = p_1p_2$ and $|G_k|=p_1p_2\dots p_k$.
	\end{proof}
	%\item Every group $G$ of order $pqr$ is
	%\item Every group $G$ of order $p^2q$ is	 
	%\item Every group $G$ of order $p^3q$ is 
	\item Every group $G$ of order $p^n$ has normal subgroups of order $p^k$ where $k = 1,2,\dots,n$. Every nonabelian group of order $p^n$ is non-solvable.
\end{enumerate}

\section{Assorted}
\begin{enumerate}
	\item Hopf Property : $\exists G,\ G \times G \cong G$.\\
	However, $\mathbb{Q} \times \mathbb{Q} \not\cong \mathbb{Q}$ since $\mathbb{Z} \times \mathbb{Z} \not\cong \mathbb{Z}$.
	\item Every finitely generated subgroups of $\mathbb{Q}$ is cyclic. (hint : find generator given a finite set of generators)
	\item Each element of $GL_2(\mathbb{Z}_p)$ is conjugate to a strictly upper triangular matrix $\begin{pmatrix} 1 & a \\ 0 & 1 \end{pmatrix}$ where $a \in \mathbb{Z}_p^\ast$.
\end{enumerate}

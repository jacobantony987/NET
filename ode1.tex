\chapter{Basic Calculus}
\section{Differentiation}
\begin{enumerate}
	\item Linearity : $[f(x)+g(x)]' = f'(x) \pm g'(x)$ and $[cf(x)]' = cf'(x)$.
	\item Product rule : $[f(x)g(x)]' = f(x)g'(x) + f'(x)g(x)$.
	\item Quotient rule : $[f(x)/g(x)]' = [f'(x)g(x) - f(x)g'(x)]/g^2(x)$.
	\item Chain rule : $[f(g(x)]' = f'(g(x)) g'(x)$.
	\item $[x^r]' = rx^{r-1}$ where $r \in \mathbb{R}$.
	\item $[a^x]' = a^x \ln a$ where $a \in \mathbb{R}^+$.
	\item $[\sin x]' = \cos x$, $[\cos x]' = -\sin x$, $[\tan x]' = \sec^2 x$, $[\csc x]' = -\csc x \cot x$,\\ $[\sec x]' = \sec x \tan x$ and $[\cot x]' = -\csc^2 x$.
	\item $[\sin^{-1}x]' = \frac{1}{\sqrt{1-x^2}}$, $[\tan^{-1}x]' = \frac{1}{1+x^2}$, and $[\sec^{-1}x]' = \frac{1}{x\sqrt{x^2-1}}$.\\ Hint : $y = f^{-1}(x) \implies f(y) = x \implies f'(y) = 1$.
\end{enumerate}

\section{Integration}
\begin{enumerate}
	\item Linearity : $\int [f(x) \pm g(x)]\ dx = \int f(x)\ dx \pm \int g(x)\ dx$ and $\int cf(x)\ dx = c \int f(x)\ dx$.
	\item Product rule : $\int [f(x)g(x)]\ dx = f(x)\int g(x)\ dx - \int f'(x) \left[\int g(x)\ dx\right]\ dx$.
		{\color{blue}
		$$\int fg\ dx = f\int g - f'\iint g + f''\iiint g + \dotsb $$
		}
	\item $\int \tan x\ dx = -\log \cos x$ and $\int \cot x\ dx = \log \sin x$.
	\item \textcolor{red}{$\int \csc x\ dx = \log (\csc x - \cot x)$ and $\int \sec x\ dx = \log (\sec x + \tan x)$.}
	\item \textcolor{red}{$\int \sqrt{a^2 - x^2}\ dx = \frac{x}{2} \sqrt{a^2 - x^2} + \frac{a^2}{2} \sin^{-1} \frac{x}{a}$}
\end{enumerate}

\section{Ordinary Differential Equation}
\begin{enumerate}
	\item An equation involving derivatives with respect to an independent variable and involving dependent variable is called an \textbf{ordinary differential equation}(ODE).
	\item  The \textbf{order} and \textbf{degree} of an ODE is the order and degree of its highest derivative.
	\item An ODE is \textbf{linear} if it does not contain product of dependent variable and its derivatives.
	\item A \textbf{solution} of a differential equationis a relation between the dependent variable and the independent variable. Solution has the general form : $f(x,y) = 0$.
		\subitem A \textbf{general solution} is of the form $\sum c_jy_j(x)$ where $c_j$s are arbitrary constants and the number of arbitrary constants is equal to the order of the differential equation.
		\subitem A \textbf{particular solution} is obtained from general solution by giving particular values to its arbitrary constants.
		\subitem A \textbf{singular solution} is a solution which cannot be obtained from a general solution by a choice of arbitrary constants.
	\item There are two major type of problems :
		\subitem An \textbf{initial value problem} is a differential equation together with values of dependent variable and its derivatives for a particular value of independent variable. 
		\subitem A \textbf{boundary value problem} is a differential equation together with functions of dependent variable and its derivatives at different values of independent variable.
\end{enumerate}

%	Existence, Uniqueness of Solutions
\subsection{Solving first order ordinary differential equations}
\begin{enumerate}
	\item Variable Separable : $f(x)dx = g(y)dy$\\ 
	\textcolor{blue}{$\int f(x) dx = \int g(y) dy$.}
	\item Homogeneous : $x^k f(y/x,y')$ \\ 
	\textcolor{blue}{$y= vx \implies dy = vdx + xdv$. Then $g(x)dx = h(v)dv$.}
	\item Exact : $Mdx + Ndy = 0$ where $M_y = N_x$ and $M,N,M_y,N_x$ are continuous.\\
	\textcolor{blue}{$\int M\ dx + \int N^\star\ dy = C$ where $N^\star$ is the part of $N(x,y)$ not containing $x$.}
	\item Almost Exact : $Mdx + Ndy = 0$ but $M_y \ne N_x$. \\
		\textcolor{blue}{Case 1 : $(M_y - N_x)/N = f(x)$, Case 2 : $(M_y-N_x)/-M = g(y)$ and\\ Case 3 : $(M_y - N_x)/(N_y - M_x) = h(z)$ where $z = xy$.
		Suppose Case 1 is true,\\ then $I\!F = e^{\int f(x)\ dx}$ and $\int M\ I\!F\ dx + \int (N\ I\!F)^\star\ dy = C$.}
	\item Inspection Method - Use known results to simply the ODE.
		{\color{blue}
		\subitem $[y/x]' = (xdy - ydx)/x^2$.
		\subitem $[x/y]' = (ydx - y^2dx)/y^2$.
		\subitem $[y^2/x]' = (2xydy - y^2dx)/x^2$.
		\subitem $[\ln(xy)]' = (xdy+ydx)/xy$.
		\subitem $[xy]' = xdy + ydx$.
		\subitem $[x^2+y^2]' = 2(xdx + ydy)$.
		\subitem $[\tan^{-1} (x/y)]' = (ydx-xdy)/(x^2+y^2)$.
		\subitem $[\sin^{-1} (x/y)]' = (ydx-xdy)/y\sqrt{y^2-x^2}$.
		\subitem $[\sec^{-1} (x/y)]' = (ydx-xdy)/\sqrt{x^2-y^2}$.
		\subitem $[\ln(x/y)]' = (ydx-xdy)/xy$.
		}
	\item Leibnitz`s Method : $y'+P(x)y = Q(x)$.\\ 
	\textcolor{blue}{The solution is : $y\ I\!F = \int I\!F\ Q(x)\ dx$ where $I\!F = e^{\int P(x)\ dx}$.}
	\item Bernouli`s Method : $y'+P(x)y = Q(x)y^n$ where $n \ne 0,1$.\\ 
	\textcolor{blue}{The solution is : $y^{1-n}\ I\!F = \int I\!F\ Q(x)(1-n)\ dx$ where $I\!F = e^{\int P(x)(1-n)\ dx}$.\footnote{In an intermediate step, we replace $y^{1-n}$ with $u$ and solve using Leibnitz`s method.}}
\end{enumerate}

\subsubsection*{Exercise}
\begin{enumerate}
	\item Computing $M$ from $N$ in an exact differential equation\\
		Suppose $g(x,y)dx + (x+y)dy = 0$ is exact and $g(x,0) = x^2$.\\
		Exact $\implies g_y = N_x = 1 \implies g(x,y) = y+f(x)$\\ And $g(x,0) = f(x) = x^2 \implies g(x,y) = x^2+y$.
	\item Set $S = \{ \frac{2}{x+1} : x \in (-1,1) \}$.\\
		$-1 < x < 1 \implies 0 < x+1 < 2 \implies \infty > 1/(x+1) > 1/2$\\
		$\implies \infty > 2/(x+1) > 1 \implies S = (1,\infty) \implies S' = [1,\infty)$.
	\item $S$ is union of disjoint bounded intervals.\\
		$S$ is compact only if each interval is closed. $\sup S \in S$ if right most interval is right closed and $\inf S \in S$ if left most interval is left closed. If $S$ has more than one interval in it, then $S$ being compact is a different story.
	\item Let $A \subset \mathbb{R}$. Then $I(A)$ is an open set. Thus, either $I(A)$ is empty or uncountable.
\end{enumerate}

\subsection{Existence \& Uniqueness}
\begin{enumerate}
	\item A function $f(x,y)$ such that $|f(x,y_1)-f(x,y_2)| \le k|y_1-y_2|$ is a \textbf{Lipschitz} function with Lipschitz constant $k$. If the function is differentiable, then condition reduces to the form $|\partial f/\partial y| \le k$.
	\item Peano`s Theorem : Consider an initial value problem $y' = f(x,y),\ y(x_0) = y_0$. If $f(x,y)$ is continuous and is bounded, say $|f(x,y)| \le M$, in the rectangle $|x-x_0| \le h$ and $|y-y_0|\le k$. Then there exists at least one solution $\phi$ such that $\frac{d\phi}{dx} = f(x,y)$ on the interval $|x-x_0| \le \min\{ h,k/M \}$. %Existence, Uniqueness of Solutions
	\item Picard`s Theorem : Consider an initial value problem $y' = f(x,y),\ y(x_0) = y_0$. If $f(x,y)$ is continous and is bounded in the rectangle $|x-x_0| \le h$ and $|y-y_0| \le k$ and $f(x,y)$ satisfies Lipschitz condition, then the there exists a unique solution.
	\item Types of IVP,
	\begin{enumerate} 
		\item No Solution. The general solution reduces to an contradictory statement with given initial values. Or Peano's theorem hypotheses donot hold.
		\item Unique Solution. Unique particular solution is obtained. Or Picard's theorem hypotheses hold.
		\item Uncountably many solutions. Particular solutions together with zero function and other variants.
	\end{enumerate}
\end{enumerate}

\subsection{Solving First Order ODEs of Degree $n > 1$}
\begin{enumerate}
	\item Solutions are of the form \begin{enumerate*} \item Cartesian Form (Equation containing $x,y$ and constants.) \item Parametric Form, $x = f_1(P,c)$ and $y = f_2(P,c)$. \item $x = g(x,P)G(x,P,c)$ and $y = f(x,P)F(x,P,c)$. \end{enumerate*}
	\item General Form : $p_0P^n + p_1 P^{n-1} + \dots + p_{n-1}P + p_n = 0$ where $P = y'$ and $p_k$`s are functions of $x$ and $y$. If we can factorise it into linear factors, say $(P-f_1)(P-f_2)\dotsm(P-f_n) = 0$. Then we can solve each one of those factor $P-f_k = 0$ into some $F_k(x,y,c_k) = 0$. And the general solution is $F_1(x,y,c)F_2(x,y,c)\dotsm F_n(x,y,c) = 0$.
	\item Solvable for $x$. That is, $x = f(y,P)$ where $P = dy/dx$.\\
		$x = f(y,P) \implies 1/P = F(y,P,dP/dy) \implies \psi(y,P,c) = 0 \implies y = g(P,c)$.
	\begin{enumerate}
		\item Case 1: $x = f(P) \implies 1/P = F(P,dP/dy) \implies y = g(P,c)$.
	\end{enumerate}
	\item Solvable for $y$.\\
		$y = f(x,P) \implies P = F(x,P,dP/dx) \implies \psi(x,P,c) = 0 \implies x = g(P,c)$.
	\begin{enumerate}
		\item Case 1: $y = f(P) \implies P = F(P,dP/dx) \implies x = g(P,c)$.
		\item Case 2: Lagrange`s Equation : $y = xF(P) + f(P)$.\\
			$y = xF(P)+f(P) \implies P = \psi(x,y,P,dP/dx) \implies dx/dP + g(P)x = h(p)$.\\
		Solve Leibnitz Equation.
		\item Case 3 : Clairut`s Equation : $y = xP + f(P)$.\\
			$y = xc + f(c)$.
	\end{enumerate}
\end{enumerate}

\subsection{Orthogonal Trajectory}
\begin{enumerate}
	\item If a family of curves $f(x,y,c) = 0$ satisfies differential equation $F(x,y,P) = 0$. Then the differential equation of their orthogonal trajectory is $F(x,y,-1/P) = 0$.
\end{enumerate}

\section{First Order Ordinary Differntial Equations}
\subsection{Solving ordinary differential equations for a singular solution}
\begin{definition}
	If a family of curves $f(x,y,c) = 0$ represented by $F(x,y,P) = 0$ and it has an envelope. Then the envelope is the\footnote{It is possible to have multiple singular solutions ?} singular solution of $F(x,y,P) = 0$.
\end{definition}
\begin{enumerate}
	\item Method 1 : $P$ discriminant.\\
		Let $f(x,y,P) = 0$.
		From $\frac{\partial f}{\partial P} = 0$ obtain a $P$-discriminant\footnote{relation not containing $P$} relation, $F(x,y) = 0$. Then $F(x,y)$ or its factors satisfying $f(x,y,P) = 0$ are the singular solutions.
	\item Method 2 : $c$-discriminant.\\
		Let $\phi(x,y,c) = 0$ be a solution for $f(x,y,P) = 0$.
		From $\frac{\partial \phi}{\partial c}$ obtain a $c$-discriminant relation $F(x,y) = 0$.
		Then $F(x,y)$ or its factors satisfying $f(x,y,P) = 0$ are the singular solutions.
	\item Method 3 : Quadratic Relation in $P$.\\
		Let $AP^2 + BP + C = 0$.
		Then $F(x,y) = B^2-4AC$ is the respective $P$-discriminant relation.
		And $F(x,y)$ or its factors satisfying $f(x,y,P) = 0$ are the singular solutions.
\end{enumerate}

\section{Second Order Ordinary Differntial Equations}
\subsection{Solving second order ordinary differential equaitons}
\begin{enumerate}
	\item Linear Differential Equations with Constant Coefficients
	\begin{equation} D^n y + a_1 D^{n-1} y + \dots + a_n y = R(x) \end{equation}
		Solution is of the form : Complementary function + Particular Integral where Complementary function is the solution of the respective homogenous equation.
	\item We may write $f(D)y = R(x)$ where $f(D) = D^n + a_1 D^{n-1} + \dots + a_n$ is the respective auxiliary equation.
		Let $m_1,m_2,\dots$ be solutions of the auxiliary equation. 
		Then $e^{m_i x}$ is solution of the homogenous equation. 
		If $m_i$ is a root of multiplicity $n$ then $x^ke^{m_i},\ k = 0,1,2,\dots, n-1$ are the respective solutions.
	\begin{enumerate}
		\item Case 1 : Real Distinct Roots.\\
			Let $m = m_1, m_2$. Then $y = c_1 e^{m_1}x + c_2 e^{m_2 x}$ is the complementary function.
		\item Case 2 : Real, Multiple Roots.\\
			Let $m$ be a real root of multiplicity $4$. Then $y = (c_1+c_2x+c_3x^2+c_4x^3) e^{m x}$ is the complementary function.
		\item Case 3 : Complex, Conjugate Roots.\\
			Let $m = \alpha \pm i \beta$. Then $y = e^{\alpha x} (c_1 \cos \beta x + c_2 \sin \beta x)$ is the complementary function.
		\item Case 4 : Complex, Conjugate, Multiple Roots.\\
			Let $\alpha \pm i \beta$ be conjugate roots of multiplicity $4$. Then $y = e^{\alpha x} ((c_1 + c_2 x + c_3 x^2 + c_4 x^3) \cos \beta x + (c_5 + c_6 x + c_7 x^2 + c_8 x^3) \sin \beta x)$ is the complementary function.
		\item {\color{red} Case 5 : Conjugate Surds.\\
			Let $m = \alpha \pm \sqrt{\beta}$. Then $y = e^{\alpha x}(c_1 \cosh \beta x + c_2 \sinh \beta x)$ is the complementary function. }\footnote{Why ?}
	\end{enumerate}
	\item Particular Integral $y_p$
	\begin{enumerate}
		\item Case 1 : $R(x) = e^{\alpha x}$.
			$$ y_p = \begin{cases} \frac{e^{\alpha x}}{f(\alpha)} & f(\alpha) \ne 0 \\ \frac{1}{\phi(\alpha)} \frac{x^r}{r!} e^{\alpha x} & f(\alpha) = 0 \end{cases} $$
		\item Case 2 : $R(x) = \sin x$.
			$$ y_p = \begin{cases} \frac{1}{f(D)} \sin \alpha x & f(D) \ne 0, D^2 = -\alpha^2\\ \frac{x}{2} \int \sin \alpha x & f(D) = 0 \end{cases} $$
		\item Case 3 : $R(x) = x^m$.
			$$ y_p = \frac{1}{f(D)} x^m  \text{ where } (1-D)^{-n} = \sum_{r = 0}^\infty \binom{-n}{r} D^r $$
		\item Case 4 : $R(x) = e^{\alpha x} v(x)$.
			$$ y_p = e^{\alpha x} \frac{1}{f(D+\alpha)} v(x)$$
	\end{enumerate}
\item Cauchy-Euler Equations
\begin{equation}
	a_n x^n D^n y + a_{n-1} x^{n-1} D^{n-1} y + \dots + a_1 x Dy + a_0 y = R(x)
\end{equation}
		Put $x = e^t$. Then $t = \log x$, $x Dy = Dy$, $x^2 D^2 y = D(D-1)y$, \dots. 
		The Cauchy-Euler equation reduces to a linear differential equation with constant coefficient.
	\item Legendre`s Linear Differential Equation
\begin{equation}
	a_n (\alpha x + \beta)^n D^n y + a_{n-1} (\alpha x + \beta)^{n-1} D^{n-1} y + \dots + a_1 (\alpha x + \beta) Dy + a_0 y = R(x)
\end{equation}
		Put $\alpha x + \beta = e^t$. Then $t = \log (\alpha x + \beta)$, $(\alpha x + \beta) Dy = \alpha Dy$, $(\alpha x + \beta)^2 D^2y = \alpha^2 D(D-1)y$, \dots.
		The Legendre`s linear differential equation reduces to a linear differential equation with constant coefficient.
	\item Finding general solution from a fundamental solution.
\end{enumerate}

\section{Fields}

%Fraleigh Part VI - Chapters 29-33
\section{Extension Fields}
\begin{definition}
	There exists a unique \textbf{Galois field} $GF(p^n)$ of order $p^n$.
\end{definition}

\begin{theorem}[Kronecker]
	Let $F$ be a field and $f(x)$ be a nonconstant polynomial in $F[x]$.
	Then there exists an extension field $E$ of $F$ and an $\alpha \in E$ such that $f(\alpha) = 0$.
\end{theorem}

\begin{definition}
	A field $E$ is an \textbf{extension field} of field $F$ if $F$ is containined in $E$.
\end{definition}

\begin{definition}
	A field $E$ is a \textbf{simple extension} of field $F$ if there exists some $\alpha \in E$ such that $E$ is the minimal extension field of $F$ containing $\alpha$.
\end{definition}

\begin{definition}
	Let field $E$ be an extension of field $F$.
	A number $\alpha \in E$ is \textbf{algebraic over $F$} if there exists $f(x) \in F[x]$ such that $f(\alpha) = 0$.
\end{definition}

	Then $\alpha$ is \textbf{algebraic over the field $F$.}
	Otherwise $\alpha$ is \textbf{transcendental over the field $F$}.
	If $F = \mathbb{Q}$, then $\alpha$ is an \textbf{algebraic number}.

\begin{definition}
	An extension $E$ of a field $F$ is \textbf{algebraic} if $E \cong F(\alpha)$ for some $\alpha$ algebraic over $F$.
\end{definition}
	The field $\mathbb{Q}(\pi)$ is a simple, transcendental extension of $\mathbb{Q}$.
	And $\mathbb{Q}(i)$ is a simple, algebraic extension of $\mathbb{Q}$ as $f(x):\ x^2+1 \in \mathbb{Q}[x]$ and $f(i) = 0$.

\begin{definition}
	Let field $E$ be an $n$-dimensional vector space over field $F$.
	Then $E$ is a \textbf{finite extension} of $F$.
	And $[E:F] = n$.
\end{definition}

\begin{theorem}[Fundamental Theorem of Algebra]
	The field $\mathbb{C}$ is algebraically closed.
\end{theorem}
\begin{proof}
	Every non-constant polynomial has a linear factorisation.
	Let $f(z)$ be a non-constant polynomial which has no zero in $\mathbb{C}$.
	Then $1/f(z)$ is entire.
	Clearly $f(z) \to \infty$ as $z \to \infty$.
	Thus, $1/f(z) \to 0$ as $z \to \infty$.
	Therefore, $f$ is bounded.
	However, by Liouville`s theorem, the bounded, entire function $1/f(z)$ is constant.
\end{proof}

	Field $\mathbb{C}$ does not have any algebraic extensions. However, the field of all rational functions $\mathbb{C}(x)$ is a transcendental extension of $\mathbb{C}$.

\section{Properties of the $Z_n$ Ring}
	\par The binary algebra, $\entity{\mathbb{Z}_n,+_n,\times_n}$ is a commutative ring with unity. 

	\begin{theorem}
	$\entity{\mathbb{Z}_n,+_n,\times_n}$ is a field iff $n$ is a prime. 
	\end{theorem}
	\begin{proof}
		A number $a \in \mathbb{Z}_n$ is not a zero divisor(and has an inverse) iff $\gcd(a,n) = 1$.
	\end{proof}

\section{Simple Extensions of $\mathbb{Q}$}
	Let $\alpha$ be an algebraic number. Then there exists a polynomial $f(x) \in F[x]$ such that $f(\alpha) = 0$.
	From $f(x)$, we may obtain a monic polynomial $p(x) \in \mathbb{Q}[x]$ such that $p(\alpha) = 0$.
	By division algorithm, such monic irreducible polynomials are unique.
	Thus, we may refer $p(x) = irr(\alpha,\mathbb{Q})$.
	By Kronecker`s theorem, field $\mathbb{Q}$ has an algebraic extension $\mathbb{Q}(\alpha)$.
	
\begin{definition}[cyclotomic field]
	The $n$th \textbf{cyclotomic field} is $\mathbb{Q}(\alpha)$ where $\alpha$ is a primitive $n$th root of unity.
\end{definition}

\begin{definition}[cyclotomic polynomial]
	The $n$th \textbf{cyclotomic polynomial} $\Phi_n(x)$ is the monic irreducible polynomial with primitive $n$th roots of unity as its zeroes.
	$$ \Phi_n(x) = \prod_{\substack{1 \le k \le n \\ \gcd(k,n)=1}} \!\!\!\!\!\left(x-\zeta_k\right)$$
\end{definition}

\begin{definition}
	A number $\alpha$ is \textbf{constructible} if you can draw a line of $\alpha$ length in a finite number of steps using a straightedge and a compass (given a line of unit length).
\end{definition}

\begin{enumerate}
	\item The $n$th cyclotomic polynomial has degree $\phi(n)$.
	\item The constructible numbers form a field.
	\item A number $\alpha$ is constructible iff the degree of the monic, irreducible polynomial of $\alpha$ over $\mathbb{Q}$ is a power of the prime $2$.
	\item The constructible numbers field is an infinite extension of $\mathbb{Q}$.
\end{enumerate}
	The classical problems like trisecting an angle, squaring a circle and doubling a cube are thus impossible.

%\begin{enumerate}
%	\item $[E:F]$
%	\item $\{E:F\}$
%	\item $(E:F)$
%\end{enumerate}

%Fraleigh Part X - Chapters 48-56
\section{Automorphisms \& Galois Theory}

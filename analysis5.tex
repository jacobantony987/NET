\chapter{Integration}
\begin{definition}
	A function $f$ is Riemann integrable $f \in \mathscr{R}(a,b)$ iff $U(P,f) = L(P,f)$ where upper Riemann sum, $U(P,f) = \sum_P \max f(x)(x_i-x_{i-1})$ and lower Riemann sum, $L(P,f) = \sum_P \min f(x)(x_i-x_{i-1})$ . And $\upint_a^b f(x) = \inf_P U(P,f)$ and $\lowint_a^b f(x)\ dx = \sup_P L(P,f)$.
\end{definition}

\section{Properties of Riemann Integration}
\begin{enumerate}
	\item $m(b-a) \le L(P,f) \le U(P,f) \le M(b-a)$.
	\item If $f,g \in \mathscr{R}(a,b)$, then $f \pm g, cf \in \mathscr{R}(a,b)$.
	\item If $f \in \mathscr{R}(a,b)$, then $f$ is bounded.
	\item If $f \in \mathscr{R}(a,b)$ and $f(x) = g(x)$ for all except finitely many points in $[a,b]$, then $g \in \mathscr{R}(a,b)$.
\end{enumerate}

\section*{Exercise}
\begin{enumerate}
	\item $$\sum_{k=0}^{n-1} \sin (\alpha+k\beta) = \frac{\sin (\alpha + \frac{n-1}{2}\beta) \sin \frac{n\beta}{2}}{\sin \frac{\beta}{2}}$$
	\item $$\sum_{k=0}^{n-1} \cos (\alpha+k\beta) = \frac{\cos (\alpha + \frac{n-1}{2}\beta) \sin \frac{n\beta}{2}}{\sin \frac{\beta}{2}}$$
	\item $\int_0^{\pi/2} \sin x\ dx = 1$
	\item $\int_0^{\pi/2} \cos x\ dx = 1$
\end{enumerate}

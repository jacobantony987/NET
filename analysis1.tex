\chapter{Set Theory}
	\textbf{Set} is a collection of points provided it satisfies ZFC-axioms.\\
	The points are the elements of $A$ deonted by $x \in A$.

\section{Point Set Theory}
\begin{enumerate}
	\item \textbf{Cardinality} $|A|$ is the number\footnote{We adopt Cantor`s notion of number of elements when the set is infinite.} of elements of the set $A$.
	\item Let $n \in \mathbb{N}$, then there exists a finite set of cardinality $n$ given by \index{$\mathbb{N}_n$} $\mathbb{N}_n = \{ 1,2,\dots, n \}$. 
	\item A set $B$ is a \textbf{subset} of a set $A$, $B \subset A$ if $x \in B \implies x \in A$.
	\item The \textbf{power set} $\mathcal{P}(A)$ of a set $A$ is the family of all subsets of $A$.
	\item Two sets $A,B$ are \textbf{equal}, $A = B$ if $A \subset B$ and $B \subset A$.
	\item Set Operations
		\subitem \textbf{union} of two sets $A,B$ is the set $A \cup B = \{ x : x \in A \text{ or } x \in B\}$.
		\subitem \textbf{intersection} of two sets $A,B$ is the set $A \cap B = \{ x : x \in A \text{ and } x \in B\}$.
		\subitem \textbf{complement} of a set $A$ wrt a set $B$ is the set $A-B = \{ x \in A : x \notin B\}$.
		\subitem \textbf{symmetric difference} of two sets $A,B$ is the set $A \Delta B = (A-B) \cup (B-A)$.
		\subitem cartesian \textbf{product} of $A$ and $B$, $A \times B = \{ (a,b) : a \in A,\ b \in B\}$.
	\item A \textbf{relation} from $A$ to $B$ is a subset of $A \times B$.
	And $xRy \implies (x,y) \in R \subset A \times B$.
	\item A relation on $A$ is $R \subset A \times A$.
		\subitem \textbf{reflexive} relation $R$ on $A$ satisfies $xRx,\ \forall x \in A$.
		\subitem \textbf{symmetric} relation $R$ on $A$ satisfies $xRy \iff yRx$.
		\subitem \textbf{antisymmetric} relation $R$ on $A$ satisfies $(x,y) \in R \implies (y,x) \notin R$.
		\subitem \textbf{transitive} relation $R$ on $A$ satisfies $xRy,\ yRz \implies xRz,\ \forall x,y,z \in A$.
		\subitem \textbf{total} relation $R$ on $A$ satisfies either $xRy \text{ or } yRx,\ \forall x,y \in A,\ (x \ne y)$.  \item \textbf{equivalence} relation $R$ on $A$ is a reflexive, symmetric, and trasitive relation.
		\subitem An \textbf{equivalence class} of a set $A$ containining $x$ is the subset $\hat{x} = \{ y \in A : xRy \}$ where the relation $R$ is an equivalence relation.
	\item A \textbf{partition} $\{\hat{x},\hat{y},\dots\}$ of $A$ is a family of subsets $\hat{x}$ of $A$ which satisfies
		\subitem $x \in \hat{x},\ \forall x \in A$.
		\subitem $\hat{x} \cap \hat{y} \iff \hat{x} = \hat{y}$.
		\subitem $A = \cup \{ \hat{x} : x \in A\}$.
	\item A \textbf{function} from $A$ to $B$ is relation which has a unique element $(a,b),\ \forall a \in A$.
		\subitem A function $f : A \to B$ is an \textbf{injection} if it satisfies $f(x) = f(y) \implies x = y$. 
		\subitem A function $f : A \to B$ is a \textbf{surjection} if it satisfies $y = f(x),\ \forall y \in B$.
	\item Function $id_X : X \to X$  is the \textbf{identity} function on $X$ where $id_X(x) = x,\ \forall x \in X$.
	\item A function $f : X \to X$ is \textbf{idempotent} if $f \circ f = f$.
		\subitem Let $f:X \to Y$, $g:Y \to X$ and $g \circ f = id_X$. Then $f \circ g$ is idempotent.
	\item A function $f : A \to B$ is a \textbf{bijection} if $f$ is both injective and surjective.
	Then $A,B$ are of the same cardinality $A \sim B$.
		\subitem If $f : A \to B$ is an injection, then $\exists C \subset B$ such that $f : A \to C$ is a bijection.
		Then $A \sim C \subset B \implies |A| \le |B|$.
		If $A$ is uncountable, then $B$ is uncountable. If $B$ is countable, then $A$ is countable.
		\subitem If $f : A \to B$ is an surjection, then $\exists C \subset A$ such that $f : C \to B$ is a bijection.
		Then $B \sim C \subset A \implies |B| \le |A|$.
		If $A$ is countable, then $B$ is countable, then $A$ is uncountable.
		If $B$ is uncountable, then $A$ is uncountable.
	\item \textcolor{blue}{There exists a bijection from the set of all equivalence relations on $A$ to the set of all partitions of $A$.}
	\item A set $A$ is \textbf{finite} if there exists a natural number $n$ and a bijection $f : A \to \mathbb{N}_n$.
	\item \textcolor{blue}{A set $A$ is finite if and only if there does not exist a bijection from $A$ into any proper subset of $A$. A set $A$ is infinite if $A$ has a proper subset $B$ and there exists a bijection $f : A \to B$.}
	\item A set $A$ is \textbf{countably infinite} if there exists a bijection $f : A \to \mathbb{N}$. 
		\subitem A subset of a countably infinite set is at most countably infinite.
		\subitem If $A$ is uncountable and $B$ is countable, then $A-B$ is uncountable. 
		\subitem Non-degenerate intervals are uncountable.
	\item \textcolor{blue}{The finite cartesian product of countable sets are countable.}\\
		Proof : cantor diagonalisation process and induction.
	\item \textcolor{blue}{Countable union of countable sets is countable.}\\
		Let $A_j = \{ a_{i,j} : (i,j) \in \mathbb{N} \times \mathbb{N} \}$ and $S = \displaystyle\bigcup_{j \in \mathbb{N}} A_j$. Then $S \sim \mathbb{N} \times \mathbb{N} \implies |S| = \aleph_0$ \index{Aleph Naught $\aleph_0$}.
	\item \textbf{Continuum Hypothesis} : Let $\aleph_0,\aleph_1,\dots$ where $2^{\aleph_k} = \aleph_{k+1}$. Then there does not exists a set $A$ such that $\aleph_k < |A| < \aleph_{k+1}$.
		\subitem For any set $A$, there does not exists a bijection from $A$ to power set of $\mathcal{P}(A)$.
	\item $\aleph_0^{\aleph_0} = \aleph_1$, $ \aleph_0^n = \aleph_0$, and $n\aleph_0 = \aleph_0$.
		\subitem Set of all polynomials of degree less than $n$ with rational coefficients is countable. That is, $S \sim \mathbb{Q}^n \implies |S| = \aleph_0$.
		\subitem The set of all circles with rational radii and center with rational co-ordinates is countable. That is, $S \sim \mathbb{Q}^3 \implies |S| = \aleph_0$.
		\subitem The collection of function, $F = \{ f : \mathbb{R} \to \mathbb{R} \}$ is uncountable. $|F| = |\mathbb{R}|^{|\mathbb{R}|} = \aleph_2$.
\end{enumerate}

\section{Analysis}
\begin{enumerate}
	\item An \textbf{interval} $I$ is a subset of $\mathbb{R}$ such that it contain all the real numbers between any two real numbers of it.
		\subitem An interval $I$ is open if it does not have an extreme value.
	\item A \textbf{neighbourhood} of $x \in S$ is an open interval \footnote{$N$ is a neighbourhood of $x$ if there exists an set $G$ containing $x$ which is open in $S$.} containing $x$ contained in $S$.
	\item A point $x \in S$ is an \textbf{interior point} of $S$ if there exists $\varepsilon > 0$ such that the open interval $(x-\varepsilon, x+\varepsilon)$ is contained in $S$.
	\item The set of all interior points of $S$ is the \textbf{interior} of $S$ denoted by $S^0$. 
		\subitem \textcolor{blue}{The interior of a set $S$ is the largest open set contained in it.}
		\subitem Boundary points of an interval is not its interior points. That is, $[a,b]^0 = (a,b)$.
	\item A set $G$ is \textbf{open} if and only if $G^0 = G$.
		\subitem \textcolor{blue}{Open sets are countable union of disjoint open intervals.}
	\item \textcolor{blue}{Arbitrary union of open sets is open. Finite intersection of open sets is open.}
	\item A set $C$ is closed if $\mathbb{R}-C$ is open.
		\subitem \textcolor{blue}{Closure of a set $S$, is the smallest closed set $\bar{S}$ containing $S$.}
		\subitem The \textbf{exterior} of a set is the interior of its complement.
	\item The \textbf{boundary} of a set denoted by $\partial S$ is the intersection of its closure and closure of its exterior.
	\item A point $x$ is a \textbf{limit point} of $S$ if every neighbourhood of $x$ has infinitely many points of $S$.
		\subitem A point $x$ is a limit point of $S$ if there exists an eventually nonconstant sequence $\sequence{x_n}$ in $S$ converging $x$.
		\subitem $S = \{ \frac{1}{n} : n \in \mathbb{N} \}$ has limit point $0$.
		\subitem The set of limit points of a set $S$ is the \textbf{derived set} $S'$.
		\subitem $\bar{S} = S \cup S'$.
	\item A set $S$ is \textbf{rare}(nowhere dense) if its interior is empty.\\
	A set $S$ is \textbf{meagre}(Baire first category) if it is a countable union of rare sets.\\
	A set $S$ is \textbf{non-meagre}(second category) if it is not meagre.
	\item Notions of smallness : $Countable < Zero\ Measure < Rare$.
		\subitem The set of rational numbers is rare.
		\subitem The set of irrationals numbers is rare.
		\subitem Cantor set is rare. Fat Cantor Set\footnote{The Smith-Voltera cantor set is a rare set with measure $\frac{1}{2}$, constructed by removing $\frac{1}{4}$th from middle.} with nonzero measure is rare.
	\item Counter Examples
		\subitem Cantor function is uniformly continuous, but not absolutely continuous.
		\subitem Voltera function is differentiable, its derivative is bounded everywhere. Thus it is Lebesgue integrable, but not Riemann integrable. 
		\subitem Weierstrass function\footnote{Weierstrass` monster function, $f(x) = \sum_{k=1}^\infty a^k \cos (b^k\pi x)$ } is continuous everywhere, it is integrable. But nowhere differentiable.
	\item \textbf{Dedekind Cut} is the partition of rational numbers into two disjoint sets to construct an irrational number. That is, $\mathbb{Q} = [A:B]$ where $\mathbb{Q} = A \cup B$, $A$ does not have a maximum and $B$ does not have a minimum.
	For example, $A = \{ q \in \mathbb{Q} : q^2 < 2 \}$ and $B = \{ q \in \mathbb{Q} : q^2 > 2 \}$. 
	\item A set $S$ is bounded above if there exists $m \in \mathbb{R}$ such that $\forall x \in S,\ x \le m$.
		\subitem If $S$ is bounded above, there exists infinitely many upperbounds.
	\item The least upperbound is the \textbf{supremum} of $S$, say $\sup(S)$.
		\subitem If $S$ is not bounded above, then $\sup(S) = +\infty$. 
		\subitem $\sup(S)$ need not be an element of $S$. If $\sup(S) \in S$, then $\sup(S) = \max(S)$.
	\item The greatest lowerbound is the \textbf{infimum} of $S$, say $\inf(S)$.
		\subitem If $S$ is not bounded below, then $\inf(S) = -\infty$.
\end{enumerate}

\section{Real Number System}
	The set of \textbf{real numbers} $\mathbb{R}$ is a complete,  ordered field in which the set of rational numbers $Q$ is dense.
\begin{enumerate}
	\item Greatest integer function
		$ \forall x \in \mathbb{R},\quad x-1 < \lfloor x \rfloor < x $ %greatest integer/floor
	\item Arithmetic mean vs Geometric mean
		$ \forall a,b \in \mathbb{R},\quad \frac{a+b}{2} \ge \sqrt{ab} $ 
	\item Exponential function
		$ \displaystyle \lim_{n \to \infty} \left(1+\frac{x}{n}\right)^n = e^x $
	\item Archimedian Property
		$ \forall x \in \mathbb{R},\ \exists n \in \mathbb{N} : x < n $ 
	\item Dense Subset
		$ \forall x,y \in \mathbb{R},\ \exists r \in \mathbb{Q} : x < r < y \quad (x<y) $
	\item $ \left| |a|-|b| \right|  \le |a-b| $
\end{enumerate}

\section*{Exercise}
\begin{enumerate}
	\item $\exists a,b,c,d \in \mathbb{R}$ such that $\lambda^2 x^2 + 2xy + y^2 = (ax+by)^2 + (cx+dy)^2,\forall x,y$
	\begin{proof}[Solution]
	$a^2+c^2 = \lambda$, $ab+cd = 1$, and
		$b^2+d^2 = 1$.
	Thus, $\lambda \ge 0$.\\
	$(ab+cd)^2 = a^2b^2 + 2abcd + c^2d^2 = 1$. \\
	$(a^2+c^2)(b^2+d^2) = 1-2abcd + a^2d^2 + b^2c^2 = 1-(ad-bc)^2 = \lambda$.
	Thus $1-\lambda \ge 0$.\\
		Clearly, $0 \le \lambda \le 1$.
	\end{proof}
	\item $f(x) = x^5 - 5x + 2$ Find number of real roots ?
	\begin{proof}[Solution]\footnote{For the number of distinct real roots use Sturm Theorem.}
		It has critical points at $f'(x) = 5x^4 - 5 = 0$. $x^4 = 1$.
		Then $x = 1,-1$. $f(1) = -2$ and $f(-1) = 6$. Clearly $f$ has exactly three real roots.
		Just need to confirm sign changes before $-1$ and after $1$.
		$f(-2) < 0$ and $f(3) > 0$.

		\textbf{Descartes Rule of Signs} : The number of positive real roots of a polynomial $p(x)$ if at most the number of sign changes in $p(x)$ (or an even number less than that). The number of negative real roots of a polynomail $p(x)$ is at most the number of sign changes in $p(-x)$ (or an even number less than that). 

		Number of sign changes in $f(x)  = 2$. Number of sign changes in $f(-x) = 1$. Then $f(x)$ has at most $2+1$ real roots.
	\end{proof}
	\item $\alpha = \int_0^\infty \frac{1}{1+t^2} dt$. Then $\alpha = \pi/2$ and $\frac{d\alpha}{dt} = 0$. \textcolor{blue}{Derivative of Integral was deceiving !!}
\end{enumerate}

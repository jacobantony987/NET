\section{Real Set Theory}
\begin{enumerate}
	\item A \textbf{neighbourhood} of $x \in S$ is an open interval \footnote{$N$ is a neighbourhood of $x$ if there exists an set $G$ containing $x$ which is open in $S$.} containing $x$ contained in $S$.
	\item A point $x \in S$ is an \textbf{interior point} of $S$ if there exists $\varepsilon > 0$ such that $(x-\varepsilon, x+\varepsilon)$ is contained in $S$. The set of all interior points of $S$ is the \textbf{interior} of $S$, $S^0$. 
		\subitem \textcolor{blue}{The interior of a set $S$ is the largest open set contained in it.}
		\subitem Boundary points of an interval is not its interior points. That is, $[a,b]^0 = (a,b)$.
	\item A set $G$ is \textbf{open} if and only if $G^0 = G$.
		\subitem \textcolor{blue}{Open sets are countable union of disjoint open intervals.}
	\item \textcolor{blue}{Arbitrary union of open sets is open. Finite intersection of open sets is open.}
	\item A set $C$ is closed if $\mathbb{R}-C$ is open.
		\subitem \textcolor{blue}{Closure of a set $S$, is the smallest closed set $\bar{S}$ containing $S$.}
		\subitem The \textbf{exterior} of a set is the interior of its complement. The \textbf{boundary} of a set $\partial S$ is the intersection of its closure and closure of its exterior.
	\item A point $x$ is a \textbf{limit point} of $S$ if every neighbourhood of $x$ has infinitely many points of $S$.
		\subitem A point $x$ is a limit point of $S$ if there exists an eventually nonconstant sequence $\sequence{x_n}$ in $S$ converging $x$.
		\subitem $S = \{ \frac{1}{n} : n \in \mathbb{N} \}$ has limit point $0$.
		\subitem The set of limit points of a set $S$ is the \textbf{derived set} $S'$.
		\subitem $\bar{S} = S \cup S'$.
	\item A set $S$ is \textbf{rare}(nowhere dense) if its interior is empty. A set $S$ is \textbf{meagre}(Baire first category) if it is a countable union of rare sets. A set $S$ is \textbf{non-meagre}(second category) if it is not meagre.
		\subitem The set of rational numbers is rare.
		\subitem The set of irrationals numbers is rare.
		\subitem Cantor set is rare.  \subitem Notions of smallness : $Countable > Zero\ Measure > Rare$.\footnote{The Smith-Voltera cantor set is a rare set with measure $\frac{1}{2}$, constructed by removing $\frac{1}{4}$th from middle.} \item \subitem Cantor function is uniformly continuous, but not absolutely continuous.
		\subitem Voltera function is differentiable, but its derivative is not integrable.
		\subitem Weierstrass function\footnote{Weierstrass` monster function, $f(x) = \sum_{k=1}^\infty a^k \cos (b^k\pi x)$ } is continuous everywhere but nowhere differentiable.
	\item \textbf{Dedekind Cut} : $\mathbb{Q} = [A:B]$ where $A = \{ q \in \mathbb{Q} : q < \sqrt{2} \}$ and $B = \{ q \in \mathbb{Q} : q > \sqrt{2} \}$. Clearly, $\mathbb{Q} = A \cup B$, $A$ does not have a maximum and $B$ does not have a minimum.
	\item A set $S$ is bounded above if there exists $m \in \mathbb{R}$ such that $\forall x \in S,\ x \le m$. If $S$ is bounded above, there exists infinitely many upperbounds. The least upperbound is the \textbf{supremum} of $S$, say $\sup(S)$. If $S$ is not bounded above, then $\sup(S) = +\infty$.
		\subitem $\sup(S) \notin S$
		\subitem If $\sup(S) \in S$, then $\sup(S) = \max(S)$.
	\item The greatest lowerbound is the \textbf{infimum} of $S$, say $\inf(S)$. If $S$ is not bounded below, then $\inf(S) = -\infty$.
\end{enumerate}

\section{Properties of Numbers}
\begin{enumerate}
	\item Greatest integer function
		$ \forall x \in \mathbb{R},\quad x-1 < \lfloor x \rfloor < x $ %greatest integer/floor
	\item Arithmetic vs Geometric mean
		$ \forall a,b \in \mathbb{R},\quad \frac{a+b}{2} \ge \sqrt{ab} $ 
	\item Exponential function
		$ \displaystyle \lim_{n \to \infty} \left(1+\frac{x}{n}\right)^n = e^x $
	\item Archimedian Property
		$ \forall x \in \mathbb{R},\ \exists n \in \mathbb{N} : x < n $ 
	\item Dense Subset
		$ \forall x,y \in \mathbb{R},\ \exists r \in \mathbb{Q} : x < r < y \quad (x<y) $
	\item $ \left| |a|-|b| \right|  \le |a-b| $
	\item Derived Set $A' = \{ x \in X : \forall N \in \mathcal{N}_x, N-\{x\} \cap A \ne \phi \}$.
	\item Every function on $\mathbb{N}$ is continuous as the induced topology on $\mathbb{N}$ is discrete.
\end{enumerate}

\section{Sequence}
\begin{enumerate}
	\item A \textbf{sequence} $x_n$ in a set $X$ is a function $x : \mathbb{N} \to X$ where $x_n = x(n)$.
	\item A \textbf{subsequence} $x_{n_k}$ of a sequence $x_n$ is a function $x \circ n$ where $n : \mathbb{N}\to \mathbb{N},\ n_k = n(k)$ is a strictly increasing sequence.
	\item A sequence $\sequence{x_n}$ is \textbf{convergent} if there exists $x \in \mathbb{R},\ \forall \varepsilon > 0,\ \exists N \in \mathbb{N}$ such that $\forall n > N, |x_n-x| < \varepsilon$.
	 Then $x$ is a \textbf{limit} of the sequence $\sequence{x_n}$ and $x_n \to x$.
	\item If space $X$ is $T_2$, then limit of convergent sequence in $X$ is unique.
		\subitem \textcolor{blue}{In $\mathbb{R}$, limit of a convergent sequence is unique.}
	\item \textcolor{blue}{A sequence $\sequence{x_n}$ converges if and only if every subsequence $\sequence{x_{n_k}}$ converges.}
	\item A sequence $\sequence{x_n}$ is \textbf{bounded} if $|x_n| \le k$.
		\subitem \textcolor{blue}{Every convergent sequence is bounded.}
		\subitem \textbf{Bolzano-Weierstrass Theorem} : Every bounded sequence has a convergent subsequence.
	\item A point $x$ is a \textbf{limit point}(cluster point) of the sequence $\sequence{x_n}$ if every neighbourhood of $x$ contains infinitely many terms of the sequence.
		\subitem \textcolor{blue}{$x$ is a limit point of $\sequence{x_n}$ if and only if it has a subsequence converging to $x$.}
		\subitem \textcolor{blue}{Every convergent sequence has a unique limit point.}
		\subitem \textcolor{blue}{A bounded sequence with unique limit point is convergent.}
	\item A sequence $x_n$ is \textbf{Cauchy} if $\forall \varepsilon > 0,\ \exists N \in \mathbb{N}$ such that $\forall n,m >N,\ |x_n-x_m| < \varepsilon$.
		\subitem \textcolor{blue}{Every Cauchy sequence is bounded.}
	\item A space is \textbf{complete} if every Cauchy sequence in it converges.
		\subitem In $\mathbb{R}$, sequence is convergent if and only if Cauchy.
		\subitem $\mathbb{R}^n,\mathbb{C}^n,l^2,C[a,b]$ are complete.
		\subitem Sequence space $l^p$ is complete if and only if $p=2$.
	\item A sequence $\sequence{x_n}$ is \textbf{monotonically increasing} if $\forall n \in \mathbb{N},\ a_{n+1} \ge a_n$.
		\subitem \textcolor{blue}{Every sequence has a monotone subsequence.}
		\subitem \textcolor{blue}{Every monotonically increasing(decreasing) sequence which is bounded above(below) is convergent. And the limit is its supremum(infimum).}
	\item A sequence $\sequence{x_n}$ is \textbf{contractive} if there exists $c \in (0,1)$ such that $|a_{n+2}-a_{n+1}| \le c|a_{n+1}-a_n|$ for sufficiently large values of $n$.
		\subitem \textcolor{blue}{Every contractive sequence is Cauchy.}
	\item $\forall x \in \mathbb{R}$, there exist a rational sequence and an irrational sequence converging to $x$.
		$\left[ \frac{10^n x_n}{10^n} \right] \to x$ and $ x_n + \frac{\sqrt{2}}{n} \to x$.
	\item Logarithm function is continuous.
		That is, $x_n \to x \implies \ln{x_n} \to \ln{x}, \quad (x_n > 0)$.
	\item Square root function is continuous.
		That is, $x_n \to x \implies \sqrt{x_n} \to \sqrt{x}, \quad (x_n > 0)$.
	\item Properties of Convergent Sequences,
		\subitem $x_n \to x \implies kx_n \to kx$.
		\subitem $x_n \to x,\ y_n \to y \implies x_n \pm y_n \to x \pm y$.
		\subitem $x_n \to x,\ y_n \to y \implies x_n y_n \to xy$
		\subitem $x_n \to x,\ y_n \to y,\ y_n \ne 0,\ y \ne 0 \implies x_n/y_n \to x/y$ 
	\item $x_n \to x,\ y_n \to y,\ x_n \le y_n \implies x \le y$
		\subitem $x_n \to x, x_n \le k \implies x \le k$.
	\item \textcolor{blue}{\textbf{Squeeze theorem} : $x_n \le y_n \le z_n,\ x_n \to l,\ z_n \to l \implies y_n \to l$.}
	\item \textcolor{blue}{Every convergent sequence is absolute convergent.}
	\subitem $|x_n| \to |x| \nimplies x_n \to x$.
	\subitem $x_n \to 0 \iff |x_n| \to 0$.
	\item $x_ny_n \to xy, \ x_n \to x \nimplies y_n \to y$
	\item $x_n \to \pm \infty \implies x_{n_k} \to \pm \infty$.
	\item Tests for non-convergence,
		\subitem Unbounded sequences are non-convergent.
		\subitem If sequence has two convergent subsequence with distinct limits.
		\subitem If it has a non-convergent subsequence.
	\item A few popular convergent sequences,
		\subitem $x^n \to 0$ where $(|x| < 1)$.
		\subitem $\frac{1}{n^p} \to 0$ provided $p > 0$.
		\subitem $p^\frac{1}{n} \to 1$ provided $p > 0$.
		\subitem $n^\frac{1}{n} \to 1$.
		\subitem \textcolor{blue}{$(1+\frac{1}{n})^n \to e$.}
	\item $(1+\frac{2}{n})^n \to e^2$\\
		Let $x_n = (1+\frac{2}{n})^n$. Suppose sequence $\sequence{x_n}$ converges, then subsequence $\sequence{x_{2n}}$ converges to the same limit and $x_{2n} = \left((1+\frac{1}{n})^n\right)^2 \to e^2$.
	\item A sequence $\sequence{x_n}$ is \textbf{Cesaro summable} if the sequence of arithmetic means is convergent.
\item \textcolor{blue}{\textbf{Cauchy`s First Theorem on Limits} : Every convergent sequence is Cesaro summable and has the same limit.} That is, $x_n \to x \implies \frac{x_1+x_2+\dots+x_n}{n} \to x$.
	\subitem Let sequence $\sequence{p_n}$ be a sequence of positive real numbers with $\frac{1}{p_1+p_2+\dots+p_n} \to 0$. Then sequence of weighted arithmetic means also converges to the same limit.\\ That is, $x_n \to x \implies \frac{p_1x_1 + p_2x_2 + \dots + p_nx_n}{p_1+p_2+\dots+p_n} \to x$.
		\subitem The sequence of geometric means also converges to the same limit.\\ That is, $x_n \to x \implies (x_1x_2\dots x_n)^\frac{1}{n} \to x$ provided $x_n \ge 0$.
	\item \textcolor{blue}{\textbf{Cauchy`s Second Theorem} : $ \frac{x_{n+1}}{x_n} \to l \implies x_n^\frac{1}{n} \to l$.}
		\subitem D'Alembert`s \textbf{Ratio Test} : Suppose \textcolor{red}{$x_n > 0$} and let $\frac{x_{n+1}}{x_n} \to l$.
		If $l < 1,\ x_n \to 0$. If $l > 1,\ x_n \to +\infty$. If $l = 1$, test fails.
		\subitem Cauchy`s \textbf{Root test} : Suppose $x_n \ge 0$ and let $ (x_n)^{\frac{1}{n}} \to l$.
		If $l < 1,\ x_n \to 0$. If $l > 1,\ x_n \to +\infty$. If $l = 1$, test fails.
	\item \textcolor{blue}{\textbf{Cesaro`s theorem} : The Cauchy product of two convergent sequences is Cesaro summable.} That is, $x_n \to x,\ y_n \to y \implies \frac{x_1y_n + x_2y_{n-1} + \dots + x_ny_1}{n} \to xy$.
	\item \textbf{Stolz-Cesaro Theorem} : $\frac{x_n-x_{n-1}}{y_n-y_{n-1}} \to l \implies \frac{x_n}{y_n} \to l$ provided \textcolor{red}{$\sequence{y_n}$ is strictly monotone and diverges to $\pm \infty$.}
		\subitem $\frac{x_n-x_{n-1}}{y_n-y_{n-1}} \to l \implies \frac{x_n}{y_n} \to l \implies \frac{x_1+x_2+\dots+x_n}{y_1+y_2+\dots+y_n} \to l$ provided \textcolor{red}{$\sequence{y_n}$ is strictly increasing to $+\infty$.}\footnote{Why the corollary of Stolz-Cesaro theorem is not applicable when $y_n$ is strictly monotone and diverges to $\pm \infty$.}
	\item \textbf{Riemann Sum}
		\[ \lim_{n \to \infty} \frac{1}{n} \sum_{k=0}^\infty f(k/n) = \int_0^1 f(x)\ dx \]
\end{enumerate}

\section*{Problems}
\begin{enumerate}
	\item $\frac{(n+1)!}{(n+1)^{n+1}} \frac{n^n}{n!} \to \frac{1}{e} \implies \left(\frac{n!}{n^n}\right)^\frac{1}{n} = \frac{\sqrt[n]{n!}}{n} \to \frac{1}{e}$
\end{enumerate}

\section{Series}
\begin{enumerate}
	\item A \textbf{series} $\sum a_n$ is a sequence of the form $\sequence{b_n}$ where $b_n = \sum_{k=1}^n a_n$, the sequence of partial sums. If the sequence of partial sums converges to $s$, then the \textbf{sum} of the series $\sum a_n = s$. If the sequence of partial sums diverges, the series also diverges.
	\item $n$th term test : If $\sum a_n$ converges, then $a_n \to 0$. And if $a_n \not\to 0$ then $\sum a_n$ diverges.
	\item Suppose $\sum a_n, \sum b_n$ converges, then $\sum a_n+b_n$, $\sum \alpha a_n$ converges.
		\begin{enumerate}
			\item Abel`s test : if $\sum a_n$ is monotonic and $\sum a_n,\sum b_n$ converges, then $\sum a_nb_n$ converges
			\item Dirichlet`s test : if $\sum a_n$ is decreasing \& converges and sequence of partial sums of $\sum b_n$ is bounded, then $\sum a_nb_n$ converges.
		\end{enumerate}
	\item Power Series test : $\sum 1/n^p$ converges if $p>1$ and diverges if $p \le 1$.
	\item Geometric Series test : $\sum a^n$ converges if $|a| < 1$ and diverges if $|a| \ge 1$.
	\item Ratio test : Let $a_n > 0$ \footnote{The condition $a_n > 0$ can be relaxed a bit, to eventually positive as eventuality is all that matters.} and $a_{n+1}/a_n \to l$.\\
		If $l<1$, $\sum a_n$ converges. If $l>1$, $\sum a_n$ diverges. If $l = 1$, test fails.
	\item Comparison test : Suppose $0 \le a_n \le b_n$.\\
		If $\sum b_n$ converges, then $\sum a_n$ converges. If $\sum a_n$ diverges, then $\sum b_n$ diverges.
	\item { \color{red}Limit Comparison Test : Suppose $a_n > 0$ and $b_n > 0$\footnote{In this case, eventuality is not sufficient.} and $a_n/b_n \to l$.\\
		If $l = 0$ and $\sum b_n$ converges, then $\sum a_n$ converges.
		If $l \ne 0$, then both behaves alike.}
	\item Cauchy's $n$th root test : If $a_n > 0$ and $a^\frac{1}{n} \to l$.\\
		If $l < 1$, then $\sum a_n$ converges. If $l > 1$, then $\sum a_n$ diverges. If $l = 1$, test fails.
	\item \textcolor{blue}{Condensation test : Suppose sequence $a_n$ is decreasing and positive.\\
		Then $\sum a_n$ and $\sum 2^na_n$ behaves similar. \textcolor{red}{Tailor-made for logarithmic functions.}}
	\item \textcolor{blue}{Rabee`s test : Suppose $a_n > 0$ and $n\left(\frac{a_n}{a_{n+1}} -1 \right) \to l$.\\
		If $l < 1$, then $\sum a_n$ converges. If $l > 1$, then $\sum a_n$ diverges. If $l = 1$, test fails.}
	\item \textcolor{blue}{Logarithmic test : Suppose $a_n > 0$ and $n\log (a/a_{n+1}) \to l$.\\
		If $l > 1$, then $\sum a_n$ converges. If $l <1$, then $\sum a_n$ diverges.}
	\item \textcolor{blue}{Lebinitz test : Suppose sequence $a_n$ is decreasing and converges to zero. ($a_n \downarrow_0$)\\
		Then the \textbf{alternating series} $\sum (-1)^n a_n$ converges.}
	\item A series $\sum a_n$ is \textbf{absolutely convergent} if $\sum |a_n|$ converges. In the case of series, absolute convergence implies convergence. A sequence which is convergent, but not absolutely convergent is \textbf{conditionally convergent}.
\end{enumerate}
